\documentclass[12pt,brazil,oneside]{book}
\usepackage{lmodern}
\usepackage{amssymb,amsmath}
\usepackage{ifxetex,ifluatex}
\usepackage{fixltx2e} % provides \textsubscript
\ifnum 0\ifxetex 1\fi\ifluatex 1\fi=0 % if pdftex
  \usepackage[T1]{fontenc}
  \usepackage[utf8]{inputenc}
\else % if luatex or xelatex
  \ifxetex
    \usepackage{mathspec}
  \else
    \usepackage{fontspec}
  \fi
  \defaultfontfeatures{Ligatures=TeX,Scale=MatchLowercase}
\fi
% use upquote if available, for straight quotes in verbatim environments
\IfFileExists{upquote.sty}{\usepackage{upquote}}{}
% use microtype if available
\IfFileExists{microtype.sty}{%
\usepackage{microtype}
\UseMicrotypeSet[protrusion]{basicmath} % disable protrusion for tt fonts
}{}
\usepackage[margin=1in]{geometry}
\usepackage{hyperref}
\hypersetup{unicode=true,
            pdftitle={Software R: Análise estatística de dados utilizando um programa livre},
            pdfauthor={Felipe Micail da Silva Smolski; Iara Denise Endruweit Battisti},
            pdfborder={0 0 0},
            breaklinks=true}
\urlstyle{same}  % don't use monospace font for urls
\ifnum 0\ifxetex 1\fi\ifluatex 1\fi=0 % if pdftex
  \usepackage[shorthands=off,main=brazil]{babel}
\else
  \usepackage{polyglossia}
  \setmainlanguage[]{brazil}
\fi
\usepackage{color}
\usepackage{fancyvrb}
\newcommand{\VerbBar}{|}
\newcommand{\VERB}{\Verb[commandchars=\\\{\}]}
\DefineVerbatimEnvironment{Highlighting}{Verbatim}{commandchars=\\\{\}}
% Add ',fontsize=\small' for more characters per line
\usepackage{framed}
\definecolor{shadecolor}{RGB}{248,248,248}
\newenvironment{Shaded}{\begin{snugshade}}{\end{snugshade}}
\newcommand{\AlertTok}[1]{\textcolor[rgb]{0.94,0.16,0.16}{#1}}
\newcommand{\AnnotationTok}[1]{\textcolor[rgb]{0.56,0.35,0.01}{\textbf{\textit{#1}}}}
\newcommand{\AttributeTok}[1]{\textcolor[rgb]{0.77,0.63,0.00}{#1}}
\newcommand{\BaseNTok}[1]{\textcolor[rgb]{0.00,0.00,0.81}{#1}}
\newcommand{\BuiltInTok}[1]{#1}
\newcommand{\CharTok}[1]{\textcolor[rgb]{0.31,0.60,0.02}{#1}}
\newcommand{\CommentTok}[1]{\textcolor[rgb]{0.56,0.35,0.01}{\textit{#1}}}
\newcommand{\CommentVarTok}[1]{\textcolor[rgb]{0.56,0.35,0.01}{\textbf{\textit{#1}}}}
\newcommand{\ConstantTok}[1]{\textcolor[rgb]{0.00,0.00,0.00}{#1}}
\newcommand{\ControlFlowTok}[1]{\textcolor[rgb]{0.13,0.29,0.53}{\textbf{#1}}}
\newcommand{\DataTypeTok}[1]{\textcolor[rgb]{0.13,0.29,0.53}{#1}}
\newcommand{\DecValTok}[1]{\textcolor[rgb]{0.00,0.00,0.81}{#1}}
\newcommand{\DocumentationTok}[1]{\textcolor[rgb]{0.56,0.35,0.01}{\textbf{\textit{#1}}}}
\newcommand{\ErrorTok}[1]{\textcolor[rgb]{0.64,0.00,0.00}{\textbf{#1}}}
\newcommand{\ExtensionTok}[1]{#1}
\newcommand{\FloatTok}[1]{\textcolor[rgb]{0.00,0.00,0.81}{#1}}
\newcommand{\FunctionTok}[1]{\textcolor[rgb]{0.00,0.00,0.00}{#1}}
\newcommand{\ImportTok}[1]{#1}
\newcommand{\InformationTok}[1]{\textcolor[rgb]{0.56,0.35,0.01}{\textbf{\textit{#1}}}}
\newcommand{\KeywordTok}[1]{\textcolor[rgb]{0.13,0.29,0.53}{\textbf{#1}}}
\newcommand{\NormalTok}[1]{#1}
\newcommand{\OperatorTok}[1]{\textcolor[rgb]{0.81,0.36,0.00}{\textbf{#1}}}
\newcommand{\OtherTok}[1]{\textcolor[rgb]{0.56,0.35,0.01}{#1}}
\newcommand{\PreprocessorTok}[1]{\textcolor[rgb]{0.56,0.35,0.01}{\textit{#1}}}
\newcommand{\RegionMarkerTok}[1]{#1}
\newcommand{\SpecialCharTok}[1]{\textcolor[rgb]{0.00,0.00,0.00}{#1}}
\newcommand{\SpecialStringTok}[1]{\textcolor[rgb]{0.31,0.60,0.02}{#1}}
\newcommand{\StringTok}[1]{\textcolor[rgb]{0.31,0.60,0.02}{#1}}
\newcommand{\VariableTok}[1]{\textcolor[rgb]{0.00,0.00,0.00}{#1}}
\newcommand{\VerbatimStringTok}[1]{\textcolor[rgb]{0.31,0.60,0.02}{#1}}
\newcommand{\WarningTok}[1]{\textcolor[rgb]{0.56,0.35,0.01}{\textbf{\textit{#1}}}}
\usepackage{longtable,booktabs}
\usepackage{graphicx,grffile}
\makeatletter
\def\maxwidth{\ifdim\Gin@nat@width>\linewidth\linewidth\else\Gin@nat@width\fi}
\def\maxheight{\ifdim\Gin@nat@height>\textheight\textheight\else\Gin@nat@height\fi}
\makeatother
% Scale images if necessary, so that they will not overflow the page
% margins by default, and it is still possible to overwrite the defaults
% using explicit options in \includegraphics[width, height, ...]{}
\setkeys{Gin}{width=\maxwidth,height=\maxheight,keepaspectratio}
\IfFileExists{parskip.sty}{%
\usepackage{parskip}
}{% else
\setlength{\parindent}{0pt}
\setlength{\parskip}{6pt plus 2pt minus 1pt}
}
\setlength{\emergencystretch}{3em}  % prevent overfull lines
\providecommand{\tightlist}{%
  \setlength{\itemsep}{0pt}\setlength{\parskip}{0pt}}
\setcounter{secnumdepth}{5}
% Redefines (sub)paragraphs to behave more like sections
\ifx\paragraph\undefined\else
\let\oldparagraph\paragraph
\renewcommand{\paragraph}[1]{\oldparagraph{#1}\mbox{}}
\fi
\ifx\subparagraph\undefined\else
\let\oldsubparagraph\subparagraph
\renewcommand{\subparagraph}[1]{\oldsubparagraph{#1}\mbox{}}
\fi

%%% Use protect on footnotes to avoid problems with footnotes in titles
\let\rmarkdownfootnote\footnote%
\def\footnote{\protect\rmarkdownfootnote}

%%% Change title format to be more compact
\usepackage{titling}

% Create subtitle command for use in maketitle
\newcommand{\subtitle}[1]{
  \posttitle{
    \begin{center}\large#1\end{center}
    }
}

\setlength{\droptitle}{-2em}

  \title{Software R: Análise estatística de dados utilizando um programa livre}
    \pretitle{\vspace{\droptitle}\centering\huge}
  \posttitle{\par}
    \author{Felipe Micail da Silva Smolski \\ Iara Denise Endruweit Battisti}
    \preauthor{\centering\large\emph}
  \postauthor{\par}
      \predate{\centering\large\emph}
  \postdate{\par}
    \date{2018-09-08}

\usepackage{booktabs}
\usepackage{multirow}
\usepackage{longtable,ltcaption}
\usepackage{pdfpages}
\usepackage[T1]{fontenc}		% Selecao de codigos de fonte.
\usepackage[utf8]{inputenc}		% Codificacao do documento (conversão automática dos acentos)
\usepackage{lastpage}			% Usado pela Ficha catalográfica
\usepackage{indentfirst}		% Indenta o primeiro parágrafo de cada seção.
\usepackage{color}				% Controle das cores
\usepackage{graphicx}

%\setlength{\parskip}{1em}
\renewcommand{\baselinestretch}{1.0}
\setlength{\parindent}{1.25cm}
\selectlanguage{brazil}

\usepackage{float}
\usepackage{booktabs}
\usepackage{longtable}
\usepackage{array}
\usepackage{multirow}
\usepackage[table]{xcolor}
\usepackage{wrapfig}
\usepackage{float}
\usepackage{colortbl}
\usepackage{pdflscape}
\usepackage{tabu}
\usepackage{threeparttable}
\usepackage{threeparttablex}
\usepackage[normalem]{ulem}
\usepackage{makecell}

\begin{document}
\maketitle

{
\setcounter{tocdepth}{1}
\tableofcontents
}
\hypertarget{apresentacao}{%
\chapter*{Apresentação}\label{apresentacao}}
\addcontentsline{toc}{chapter}{Apresentação}

A necessidade de flexibilidade e robustez para a análise estatística fez
com que fosse criado, na década de 1990, a linguagem de programação R.
Capitaneado pelos desenvolvedores Ross Ihaka e Robert Gentleman, dois
estatísticos da Universidade de Auckland na Nova Zelândia, o projeto foi
uma grande evolução para a análise de dados. A partir de então, a ideia
inicial de proporcionar autonomia ao pesquisador, viu na expansão do
acesso à internet uma oportunidade para que a pesquisa científica se
tornasse cada vez mais colaborativa. Ao mesmo tempo, os códigos e
rotinas se tornaram facilmente disponibilizáveis na rede, aumentando a
reprodução e replicação dos estudos, práticas estas que podem tornar as
análises mais confiáveis.

A linguagem de programação R trouxe consigo inúmeras vantagens aos
pesquisadores. Dentre elas, pode-se dizer primeiramente que, basicamente
o R trabalha com uma extensa relação de modelos estatísticos, que vão
desde a modelagem linear e não-linear, a análise de séries temporais, os
testes estatísticos clássicos, análise de grupamento e classificação,
etc. Não bastasse este fato, é possível a apresentação gráfica dos
resultados contando com variadas técnicas, passando também pela criação
e manipulação de mapas.

Outra questão importante é que o R possui uma comunidade ativa de
desenvolvedores, que se expande regularmente. Isto faz com que as
técnicas de análise de dados atinjam pesquisadores de variadas
disciplinas ao longo do planeta. Inclusive, concebe que o
desenvolvimento dos pacotes melhorem constantemente. No ano de 2018, já
haviam mais de 12.700 pacotes disponibilizados. Não menos importante,
talvez o essencial: o programa é livre, ao passo que entrega o estado da
arte da estatística ao usuário.

Outro progresso significativo na utilização do R foi a criação do
\emph{software} RStudio, a partir de 2010. Este, por sua vez, se
configura em um ambiente integrado com o R e com inúmeras linguagens de
marcação de texto (exemplos LaTeX, Markdown, HTML). Possui igualmente
versão livre que disponibiliza ao pesquisador a execução, guarda,
retomada e manipulação dos códigos de programação diretamente em seu
console, bem como a administração de diretórios de trabalhos e projetos.

O material aqui criado é destinado não somente a alunos de graduação,
pós-graduação, professores e pesquisadores acadêmicos, mas também para
qualquer indivíduo interessado no aprendizado inicial sobre a utilização
de técnicas estatísticas com o R. Inclusive, com o objetivo de alcançar
um público das mais variadas áreas do conhecimento, esta obra foi
elaborada com exemplos gerais, a serem absorvidos em um momento inicial
do estudante. Assim, possui a base para continuar estudos posteriores em
estatística e no \emph{software} RStudio. O sistema operacional aqui
utilizado é o Windows 10.

Este livro está organizado da seguinte maneira: no capítulo
\protect\hyperlink{intro}{1} {[}\textbf{Primeiros Passos com o R}{]},
busca-se instruir o pesquisador para a instalação dos programas
necessários para acessar o ambiente de programação, bem como orientar
sobre a usabilidade do programa em suas funções básicas de carregamento
de bases de dados, criação de objetos e princípios de manipulação.

Já no capítulo \protect\hyperlink{desc}{2} {[}\textbf{Estatística
Descritiva}{]}, leva o leitor ao encontro das técnicas básicas para
descrever as variáveis em bancos de dados, como exemplos a média,
mínima, máxima, desvio padrão, os quartis e também, apresentar os
princípios dos elementos gráficos de apresentação dos dados.

O capítulo \protect\hyperlink{inf}{3} {[}\textbf{Estatística
Inferencial}{]} tratará dos métodos de determinação de intervalos de
confiança (média e proporção), testes de hipóteses (verificar a
normalidade dos dados) e das comparações entre médias de amostras
dependentes e independentes.

No capítulo \protect\hyperlink{qui}{4} {[}\textbf{Teste de
Qui-Quadrado}{]}, serão abordadas as referidas técnicas para verificação
de asssociação entre duas variáveis qualitativas e de aderência a uma
distribuição.

No capítulo \protect\hyperlink{reg}{5} {[}\textbf{Modelos de
Regressão}{]} serão introduzidos os conhecimentos sobre as técnicas de
análise de correlação e regressão linear simples, bem como sobre o
diagrama de dispersão, método dos mínimos quadrados, análise de
variância, coeficiente e intervalo de predição, da análise dos resíduos
e dos princípios de regressão múltipla.

A criação de documentos dinâmicos utilizando o RStudio será tratada no
capítulo \protect\hyperlink{rmark}{6} {[}\textbf{RMarkdown}{]}. O
pesquisador poderá conhecer as formas de integrar a programação no R e a
manipulação de bases de dados, criando, compilando e configurando
relatórios finais em diversos formatos (HTML, PDF e Word/Libre/Open
Office).

Boa leitura!

\hypertarget{intro}{%
\chapter{Introdução}\label{intro}}

\emph{Felipe Micail da Silva Smolski}

\emph{Djaina Sibiani Rieger}

\begin{flushright}
\emph{}

\emph{}
\end{flushright}

O R é um ambiente voltado para análise de dados com o uso de uma
linguagem de programação, frente a isso um conhecimento prévio dos
príncipios de programação facilita a compreensão da condução das
análises aplicadas no software. Entretanto, não é pré-requisito. Neste
capítulo abordaremos os primeiros passos para o emprego da linguagem de
programação R utilizando uma interface ``amigável'' - o software
RStudio. Além disso, serão apresentados os comandos básicos para a
manipulação de dados dentro do RStudio.

\hypertarget{download-e-instalacao-do-r-e-rstudio}{%
\section{Download e instalação do R e
Rstudio}\label{download-e-instalacao-do-r-e-rstudio}}

\textbf{R}: \url{http://www.r-project.org}. Clique em Download (CRAN) -
escolha o link de um repositório - clique no link do sistema operacional
(Linux, Mac ou Windows) - clique em \emph{install R for de first time -
Download}.

\textbf{RStudio}:
\url{http://www.rstudio.com/products/rstudio/download}. Em RStudio
Desktop, escolha a versão \emph{free}, seguidas da opção do sistema
operacional do usuário.

Lembrando que:

\begin{itemize}
\tightlist
\item
  R é o software;
\item
  RStudio é uma ferramenta amigável para o R.
\end{itemize}

\hypertarget{paineis}{%
\section{Painéis}\label{paineis}}

O RStudio é a interface que faz com que seja mais fácil a utilização da
programação em R.

\begin{figure}[H]

{\centering \includegraphics[width=0.8\linewidth]{paineis} 

}

\caption{Painéis do Rstudio}\label{fig:paineis1}
\end{figure}

Fonte: Elaborado pelo(s) autor(es).

\begin{itemize}
\tightlist
\item
  \textbf{Fonte/Editor de Scripts}: se constitui do ambiente onde serão
  abertos os scripts previamente salvos nos mais diversos formatos ou
  mesmo sendo o local de visualização das bases de dados.
\item
  \textbf{Console}: local onde será efetuada a digitação das linhas de
  código que serão interpretadas pelo R.
\item
  \textbf{Ambiente e Histórico}: o ambiente será visualizado os objetos
  criados ou carregados durante a seção e; a aba History retoma os
  scripts digitados no console.
\item
  \textbf{Plots/arquivos/Pacotes}: local onde podem ser acessados os
  arquivos salvos no computador pela aba \emph{files}; a aba
  \emph{Plots} carrega os gráficos e plotagens; a aba \emph{Packages}
  contém os pacotes instalados em seu computador, onde são ativados ou
  instalados novos; em \emph{Help} constam as ajudas e explicações dos
  pacotes e; \emph{Viewer} vizualiza documentos do tipo html.
\end{itemize}

\hypertarget{help}{%
\section{Help}\label{help}}

Acessamos a ajuda do RStudio por meio do comando \texttt{help()},
através da aba ``Help'' ou ao clicar no nome do pacote. Pode-se digitar
a ajuda que usuário necessita (exemplo \texttt{help("summary")}), ou
diretamente no colsole digitamos ? e a função desejada, exemplo:
\texttt{?mean}.

\hypertarget{instalacao-de-pacotes}{%
\section{Instalação de pacotes}\label{instalacao-de-pacotes}}

Em alguns situações, o uso de pacotes pode dar ao trabalho mais
praticidade, e para isso se faz necessário efetuar a sua instalação.
Precisamos ir até o painel dos pacotes em \emph{packages}, selecionar a
opção instalar e inserir o nome do pacote desejado na janela indicada.
Ao selecionar a opção instalar, no console receberemos informações do
procedimento e do sucesso do mesmo.

\begin{figure}[H]

{\centering \includegraphics[width=0.8\linewidth]{pacotes1} 

}

\caption{Instalação de pacotes}\label{fig:pacotes1}
\end{figure}

Fonte: Elaborado pelo(s) autor(es).

\begin{figure}[H]

{\centering \includegraphics[width=0.8\linewidth]{pacotes2} 

}

\caption{Caixa de informação de pacote a ser instalado}\label{fig:pacotes2}
\end{figure}

Fonte: Elaborado pelo(s) autor(es)

A mesma função, para instalação de um pacote, pode ser efetuada
diretamente via console: \texttt{install.packages("pacote")}. É
importante ressaltar a função \texttt{library(nomedopacote)} que é
utilizada no console para informar ao R e ``carregar'' o pacote que o
usuário irá utilizar. Podem ser instalados mais de um pacote ao mesmo
tempo, como no exemplo:

\texttt{install.packages(c("readr",\ "readxl"))}

\hypertarget{abrir-arquivo-de-dados}{%
\section{Abrir arquivo de dados}\label{abrir-arquivo-de-dados}}

Dispondo de um banco de dados em uma planilha eletrônica (LibreOffice
Calc ou EXCEL), neste caso será utilizado o arquivo
\href{https://github.com/Smolski/softwarelivrer/raw/master/basico/arvores.xlsx}{árvores}
como exemplo o banco de dados. Os dados derivam de uma pesquisa com
espécies de árvores registrando as variáveis diâmetro altura do peito
(DAP) e altura. Dados cedidos pela professora Tatiane Chassot.

Pode-se utilizar a linha de comando para carregar os arquivos de dados,
da seguinte forma:

\texttt{library(readxl)}

\texttt{nome.objeto.xls\ =\ read\_excel("d:/arvores.xls")}

Outras opções de arquivos podem ser carregados no RStudio, como por
exemplo arquivos de texto (.txt ou .csv), arquivos derivados do excel
(.xls ou .xlsx), arquivos de dados do SPSS (.sav), do \emph{software}
SAS (.sas7bdat) e do STATA (.dta). A instalação de alguns pacotes é
requerida, dependendo da origem da base de dados, como por exemplo o
\texttt{readxl}, \texttt{readr} e \texttt{haven}, como os exemplos
abaixo:

\texttt{library(readr)}

\texttt{nomeobjeto\ =\ read.csv("d:/arvores.csv")}

\texttt{library(haven)}

\texttt{nomeobjeto\ =\ read\_sav("d:/arvores.sav")}

\texttt{nomeobjeto\ =\ read\_dta("d:/arvores.dta")}

\texttt{nomeobjeto\ =\ read\_sas("d:/arvores.sas7bdat")}

Outras opções podem ser comandadas dentro destes comando para abertura
de arquivos, como por exemplo, um arquivo csv em que esteja separado por
vírgulas pode ser lido como:

\texttt{read.csv("d:/arvores.csv",\ sep=",")}

O comando \texttt{header=TRUE} diz que a primeira linha do arquivo
contém o cabeçalho; \texttt{skip=4} faz com que sejam ignoradas as 4
primeiras linhas.

A função \texttt{load()} (exemplo: \texttt{load("base.RData")}) pode ser
utilizada para carregar as bases de dados salvas com a função
\texttt{save()}, que será descrita no subcapítulo a seguir.

Outra opção é o carregamento das bases de dados manualmente pelo caminho
\emph{Envoirment \(>\) Import Dataset}, escolhendo o tipo de arquivo:

\begin{figure}[H]

{\centering \includegraphics[width=0.8\linewidth]{r3} 

}

\caption{Aba Import Dataset}\label{fig:r3}
\end{figure}

Fonte: Elaborado pelo(s) autor(es).

Na caixa correspondente a File/Url se insere o endereço virtual ou o
local onde se encontra o arquivo. Ao importar os dados, carrega-se um
objeto criado com as informações contidas no arquivo. No nosso exeplo,
carregamos a planilha arvores (arquivo .xls) como mostra a Figura
\ref{fig:r4}, derivado do caminho ``Import Dataset \(>\) From Excel'' do
Environment.

\begin{figure}[H]

{\centering \includegraphics[width=0.8\linewidth]{r4} 

}

\caption{Caixa de informações do Import Data}\label{fig:r4}
\end{figure}

Fonte: Elaborado pelo(s) autor(es).

O campo \emph{Code Preview} mostra o comando que está sendo criado para
a importação destes dados. Em \emph{Import Options}, delimita-se opções
do objeto como o nome (\emph{name}), o número máximo de linhas
(\emph{Max Rows}), quantas linhas serão puladas na importação do arquivo
(\emph{Skip}), o tratamento das células em branco (\emph{NA}) e se a
primeira linha contém os nomes (\emph{Firts Row as Names}).

Com relação à importação de arquivos de texto separado por caracteres
(.csv), ela se dá via ``Import Dataset \(>\) From Text (readr)'' do
Environment. Constam algumas solicitações diferentes a serem
determinadas pelo usuário no campo \emph{Import Options}, conforme
mostra a Figura \ref{fig:r4csv}. Uma questão importante é a opção
\emph{Delimiter}, a qual o pesquisador tem que prestar atenção quando o
arquivo está separado por vírgulas (\emph{Comma}), ponto e vírgula
(\emph{Semicolon}) ou outro tipo de caractere. A opção \emph{Locale
\(>\) Configure\ldots{}} oportuniza determinar os tipos de marca decimal
e codificação de textos, por exemplo.

\begin{figure}[H]

{\centering \includegraphics[width=0.8\linewidth]{r4csv} 

}

\caption{Opções da importação de arquivos .csv}\label{fig:r4csv}
\end{figure}

Fonte: Elaborado pelo(s) autor(es)

Importante mencionar que em ambos os casos de importação, no campo
\emph{Dada Preview} onde constam os dados do arquivo a ser importado, é
possível determinar o tipo de dado que cada ``coluna'' contém. Isto é
extremamente importante, pois campos que possuem números, que serão
posteriormente utilizados em operações aritméticas, por exemplo, devem
ser configurados como tal. No entanto, como será visto adiante, a
alteração do tipo do dado também pode ser feita posteriormente sem
problema algum.

Alguns tipos de dados:

\begin{itemize}
\tightlist
\item
  \textbf{Numeric}: números, valores decimais em geral (\texttt{5.4}).
\item
  \textbf{Integer}: números (\texttt{4}).
\item
  \textbf{Character}: variável de texto, ou \emph{string}
  (\texttt{casa}).
\item
  \textbf{Double}: cria um vetor de precisão dupla, que abarca os
  números.
\item
  \textbf{Logical}: operadores booleanos (\texttt{TRUE,\ FALSE}).
\item
  \textbf{Date}: opção para datas.
\item
  \textbf{Time}: vetor para séries de tempo.
\item
  \textbf{Factor}: variável nominal, inclusive como fator ordenado,
  representam categorias.
\end{itemize}

Ainda, é possível importar objetos utilizando arquivos hospedados em
links da internet, por exemplo o comando
\texttt{source("http://www.openintro.org/stat/data/cdc.R")} utiliza a
função \texttt{source()} para carregar um objeto do R denominado cdc
(``cdc.R'').

\hypertarget{salvar-arquivo-de-dados}{%
\section{Salvar arquivo de dados}\label{salvar-arquivo-de-dados}}

O banco de dados que o R armazena na memória pode ser salvo, junto com
todo o ambiente, usando o ícone de disquete na aba ``Environment''
(salva como arquivo .RData), e depois carregado pelo ícone de pasta
(Abrir dados\ldots{}) na mesma aba. Desta forma, salvará todos os
objetos criados no ambiente de trabalho.

\begin{figure}[H]

{\centering \includegraphics[width=0.8\linewidth]{r6} 

}

\caption{Atalho para abrir e salvar arquivo de dados}\label{fig:r6}
\end{figure}

Fonte: Elaborado pelo(s) autor(es)

Outra opção com mesmo efeito é utilizar o comando a seguir diretamente
no console do RStudio:

\texttt{save("nomeDoObjeto",file="nomeDoArquivo.RData")}

O nome do objeto pode ser uma lista de objetos para salvar mais de um
objeto do ambiente, \texttt{list=("objeto1",\ "objeto2")}. Para carregar
um arquivo RData no ambiente, o comando a ser utilizado pelo usuário é

\texttt{load("arquivo.RData")},

desde que o arquivo esteja no diretório de trabalho do R.

É possível exportar as bases trabalhadas para vários formatos de
arquivos de dados e de texto, como seguem alguns exemplos:

\begin{itemize}
\tightlist
\item
  \texttt{write.csv(nomeobjeto,"file.csv",\ sep=";")}: salvando em
  arquivo csv.
\item
  \texttt{write.foreign(nomeobjeto,"d:/nome.sps")}: arquivos sps.
\item
  \texttt{write.foreign(nomeobjeto,"d:/nome.dta")}: arquivos dta.
\item
  \texttt{write.foreign(nomeobjeto,"d:/nome.sas7bdat")}: arquivos
  sas7bdat.
\end{itemize}

\hypertarget{diretorios-de-trabalho}{%
\section{Diretórios de trabalho}\label{diretorios-de-trabalho}}

Os trabalhos efetudados via Rstudio, incluindo as bases de dados, os
objetos, os resultados das fórmulas, os cálculos aplicados sobre os
vetores e demais arquivos resultantes da utilização do programa podem
ser salvos em seu diretório de arquivos. Após instalado o Rstudio
destina um diretório padrão salvar estes arquivos, o qual pode ser
verificado com o comando \texttt{getwd()}.

Este caminho padrão, por sua vez, pode ser alterado via comando

\texttt{setwd("C://file/path")}

onde o usuário escolhe a pasta desejada que ficará como padrão. O
comando \texttt{dir()} mostra ao usuário os documentos que constam no
diretório padrão ou o escolhido para a consulta.

\hypertarget{operacoes}{%
\section{Operações}\label{operacoes}}

\hypertarget{operacoes-aritmeticas}{%
\subsection{Operações Aritméticas}\label{operacoes-aritmeticas}}

A realização de uma operação aritmética no R acontece da seguinte forma:
onde a resolução das operações segue o padrão, ou seja, primeiro
exponenciações, seguido de multiplicações e divisões, deixando por
ultimo adições e subtrações, de acordo com a ordem que estão dispostas.
Para alterar a prioridade da resolução de operações fazemos o uso do
parenteses para destacar a operação que deve ser prioritária na
resolução. Seguem alguns exemplos efetuados diretamente no console do
RStudio:

\begin{Shaded}
\begin{Highlighting}[]
\CommentTok{# soma}
\DecValTok{19}\OperatorTok{+}\DecValTok{26}
\end{Highlighting}
\end{Shaded}

\begin{verbatim}
[1] 45
\end{verbatim}

\begin{Shaded}
\begin{Highlighting}[]
\CommentTok{# subtração}
\DecValTok{19-26}
\end{Highlighting}
\end{Shaded}

\begin{verbatim}
[1] -7
\end{verbatim}

\begin{Shaded}
\begin{Highlighting}[]
\CommentTok{# divisão}
\DecValTok{4}\OperatorTok{/}\DecValTok{2}
\end{Highlighting}
\end{Shaded}

\begin{verbatim}
[1] 2
\end{verbatim}

\begin{Shaded}
\begin{Highlighting}[]
\CommentTok{# multiplicação }
\DecValTok{4}\OperatorTok{*}\DecValTok{2}
\end{Highlighting}
\end{Shaded}

\begin{verbatim}
[1] 8
\end{verbatim}

\begin{Shaded}
\begin{Highlighting}[]
\CommentTok{# exponenciação}
\DecValTok{4}\OperatorTok{^}\DecValTok{2}
\end{Highlighting}
\end{Shaded}

\begin{verbatim}
[1] 16
\end{verbatim}

\begin{Shaded}
\begin{Highlighting}[]
\CommentTok{# prioridade de resolução}
\DecValTok{19} \OperatorTok{+}\StringTok{ }\DecValTok{26} \OperatorTok{/}\DecValTok{4} \DecValTok{-2} \OperatorTok{*}\DecValTok{10}
\end{Highlighting}
\end{Shaded}

\begin{verbatim}
[1] 5.5
\end{verbatim}

\begin{Shaded}
\begin{Highlighting}[]
\NormalTok{((}\DecValTok{19} \OperatorTok{+}\StringTok{ }\DecValTok{26}\NormalTok{) }\OperatorTok{/}\NormalTok{(}\DecValTok{4} \DecValTok{-2}\NormalTok{))}\OperatorTok{*}\DecValTok{10}
\end{Highlighting}
\end{Shaded}

\begin{verbatim}
[1] 225
\end{verbatim}

\begin{Shaded}
\begin{Highlighting}[]
\CommentTok{# raiz quadrada}
\KeywordTok{sqrt}\NormalTok{(}\DecValTok{16}\NormalTok{)}
\end{Highlighting}
\end{Shaded}

\begin{verbatim}
[1] 4
\end{verbatim}

\begin{Shaded}
\begin{Highlighting}[]
\CommentTok{# Logaritmo }
\KeywordTok{log}\NormalTok{(}\DecValTok{1}\NormalTok{)}
\end{Highlighting}
\end{Shaded}

\begin{verbatim}
[1] 0
\end{verbatim}

\hypertarget{operacoes-logicas}{%
\subsection{Operações Lógicas}\label{operacoes-logicas}}

O ambiente de programação Rstudio trabalha com algumas operações
lógicas, que serão importantes na manipulação de bases de dados:

\begin{itemize}
\tightlist
\item
  \(a == b\) (``a'' é igual a ``b'')
\item
  \(a != b\) (``a'' é diferente a ``b'')
\item
  \(a > b\) (``a'' é maior que ``b'')
\item
  \(a < b\) (``a'' é menor que ``b'')
\item
  \(a >= b\) (``a'' é maior ou igual a ``b'')
\item
  \(a <= b\) (``a'' é menor ou igual a ``b'')
\item
  is.na (``a'' é missing - faltante)
\item
  is.null (``a'' é nulo)
\end{itemize}

Seguem alguns exemplos da aplicação das operações lógicas:

\begin{Shaded}
\begin{Highlighting}[]
\CommentTok{# maior que }
\DecValTok{2} \OperatorTok{>}\StringTok{ }\DecValTok{1}
\end{Highlighting}
\end{Shaded}

\begin{verbatim}
[1] TRUE
\end{verbatim}

\begin{Shaded}
\begin{Highlighting}[]
\DecValTok{1} \OperatorTok{>}\StringTok{ }\DecValTok{2}
\end{Highlighting}
\end{Shaded}

\begin{verbatim}
[1] FALSE
\end{verbatim}

\begin{Shaded}
\begin{Highlighting}[]
\CommentTok{# menor que }
\DecValTok{1} \OperatorTok{<}\StringTok{ }\DecValTok{2}
\end{Highlighting}
\end{Shaded}

\begin{verbatim}
[1] TRUE
\end{verbatim}

\begin{Shaded}
\begin{Highlighting}[]
\CommentTok{# maior ou igual a }
\DecValTok{0} \OperatorTok{>=}\StringTok{ }\NormalTok{(}\DecValTok{2}\OperatorTok{+}\NormalTok{(}\OperatorTok{-}\DecValTok{2}\NormalTok{))}
\end{Highlighting}
\end{Shaded}

\begin{verbatim}
[1] TRUE
\end{verbatim}

\begin{Shaded}
\begin{Highlighting}[]
\CommentTok{# menor ou igual a }
\DecValTok{1} \OperatorTok{<=}\StringTok{ }\DecValTok{3}
\end{Highlighting}
\end{Shaded}

\begin{verbatim}
[1] TRUE
\end{verbatim}

\begin{Shaded}
\begin{Highlighting}[]
\CommentTok{# conjunção}
\DecValTok{9} \OperatorTok{>}\StringTok{ }\DecValTok{11} \OperatorTok{&}\StringTok{ }\DecValTok{0} \OperatorTok{<}\StringTok{ }\DecValTok{1}
\end{Highlighting}
\end{Shaded}

\begin{verbatim}
[1] FALSE
\end{verbatim}

\begin{Shaded}
\begin{Highlighting}[]
\CommentTok{# ou}
\DecValTok{6} \OperatorTok{<}\StringTok{ }\DecValTok{5} \OperatorTok{|}\StringTok{ }\DecValTok{0} \OperatorTok{>}\StringTok{ }\DecValTok{-1}
\end{Highlighting}
\end{Shaded}

\begin{verbatim}
[1] TRUE
\end{verbatim}

\begin{Shaded}
\begin{Highlighting}[]
\CommentTok{# igual a}
\DecValTok{1} \OperatorTok{==}\StringTok{ }\DecValTok{2}\OperatorTok{/}\DecValTok{2}
\end{Highlighting}
\end{Shaded}

\begin{verbatim}
[1] TRUE
\end{verbatim}

\begin{Shaded}
\begin{Highlighting}[]
\CommentTok{# diferente de}
\DecValTok{1} \OperatorTok{!=}\StringTok{ }\DecValTok{2}
\end{Highlighting}
\end{Shaded}

\begin{verbatim}
[1] TRUE
\end{verbatim}

\hypertarget{criacao-de-objetos}{%
\section{Criação de objetos}\label{criacao-de-objetos}}

A linguagem de programação R se configura em uma linguagem orientada a
objetos, ou seja, a todo tempo estamos criando diversos tipos de objetos
e efetuando operações com os mesmos. Por exemplo, a criação de listas,
bases de dados, união de bases de dados, data.frames e até mesmo mapas!

\begin{Shaded}
\begin{Highlighting}[]
\CommentTok{#Criando um objeto simples}
\NormalTok{objeto =}\StringTok{ "meu primeiro objeto"} \CommentTok{#enter}
\CommentTok{#Agora para retomar o objeto criado:}
\NormalTok{objeto }\CommentTok{#enter}
\end{Highlighting}
\end{Shaded}

\begin{verbatim}
[1] "meu primeiro objeto"
\end{verbatim}

\begin{Shaded}
\begin{Highlighting}[]
\CommentTok{#Pode ser efetuada uma operação:}
\NormalTok{a=}\StringTok{ }\DecValTok{2}\OperatorTok{+}\DecValTok{1}
\NormalTok{a}
\end{Highlighting}
\end{Shaded}

\begin{verbatim}
[1] 3
\end{verbatim}

O comando \texttt{ls()} lista todos os objetos que estão criados no
ambiente e \texttt{rm(x)} remove o objeto indicado (x). Para remover
todos os objetos de uma só vez utiliza-se \texttt{rm(list=ls())}.

\begin{Shaded}
\begin{Highlighting}[]
\CommentTok{#Lista objetos do ambiente}
\KeywordTok{ls}\NormalTok{()}
\end{Highlighting}
\end{Shaded}

\begin{verbatim}
[1] "a"      "objeto"
\end{verbatim}

\begin{Shaded}
\begin{Highlighting}[]
\CommentTok{#Remover um banco de dados}
\KeywordTok{rm}\NormalTok{(a)}
\end{Highlighting}
\end{Shaded}

\textbf{Conversão de uma variável}

Para a aplicação de algumas funções é importante que cada variável
esteja corretamente classificada, o que em alguns casos não ocorre
durante o reconhecimento automático do R. Precisamos então reconhecê-la
como variável texto, numérica ou fator. Além disso, a classe ordered se
aplica a variáveis categóricas que podem ser consideradas ordenáveis.

\begin{Shaded}
\begin{Highlighting}[]
\NormalTok{idade=}\KeywordTok{c}\NormalTok{(}\StringTok{'11'}\NormalTok{, }\StringTok{'12'}\NormalTok{, }\StringTok{'31'}\NormalTok{)}
\NormalTok{nomes=}\KeywordTok{c}\NormalTok{(}\StringTok{"Elisa"}\NormalTok{, }\StringTok{"Priscila"}\NormalTok{, }\StringTok{"Carol"}\NormalTok{)}
\NormalTok{cep=}\KeywordTok{c}\NormalTok{(}\DecValTok{98700000}\NormalTok{,}\DecValTok{98701000}\NormalTok{,}\DecValTok{98702000}\NormalTok{)}
\NormalTok{idade=}\StringTok{ }\KeywordTok{as.numeric}\NormalTok{(idade)}
\NormalTok{idade}
\end{Highlighting}
\end{Shaded}

\begin{verbatim}
[1] 11 12 31
\end{verbatim}

\begin{Shaded}
\begin{Highlighting}[]
\NormalTok{cep =}\StringTok{ }\KeywordTok{as.character}\NormalTok{(cep)}
\NormalTok{cep}
\end{Highlighting}
\end{Shaded}

\begin{verbatim}
[1] "98700000" "98701000" "98702000"
\end{verbatim}

\hypertarget{algumas-funcoes-e-comandos-essenciais}{%
\section{Algumas funções e comandos
essenciais}\label{algumas-funcoes-e-comandos-essenciais}}

A função \texttt{head()} mostra as 6 primeiras colunas do arquivo para
se ter uma noção do conteúdo. No caso do mesmo ser um data.frame,
podemos solicitar o número de valores ou linhas a serem mostrados no
console através do parâmetro n ou na ausência deste, todas as linhas
serão impressas, como exemplo \texttt{head(x\ ,n=2)} para ver as duas
primeiras linhas.

O comando \texttt{summary()} efetua o resumo dos dados, se for
qualitativa mostra a frequência absoluta das categorias e se for
quantitativa apresenta as categorias. No exemplo abaixo trabalharemos
com uma base de dados de treinamento denominada ``iris'' que está
acessível no \emph{software} RStudio através do comando que carrega
dados específicos \texttt{data()}:

\begin{Shaded}
\begin{Highlighting}[]
\CommentTok{#Carregando dados da base do RSdudio iris.}
\KeywordTok{data}\NormalTok{(iris)}

\CommentTok{#Visualizando as primeiras 6 colunas}
\KeywordTok{head}\NormalTok{(iris)}
\end{Highlighting}
\end{Shaded}

\begin{verbatim}
  Sepal.Length Sepal.Width Petal.Length Petal.Width Species
1          5.1         3.5          1.4         0.2  setosa
2          4.9         3.0          1.4         0.2  setosa
3          4.7         3.2          1.3         0.2  setosa
4          4.6         3.1          1.5         0.2  setosa
5          5.0         3.6          1.4         0.2  setosa
6          5.4         3.9          1.7         0.4  setosa
\end{verbatim}

\begin{Shaded}
\begin{Highlighting}[]
\CommentTok{#Resumo do objeto}
\KeywordTok{summary}\NormalTok{(iris)}
\end{Highlighting}
\end{Shaded}

\begin{verbatim}
  Sepal.Length   Sepal.Width    Petal.Length   Petal.Width        Species  
 Min.   :4.30   Min.   :2.00   Min.   :1.00   Min.   :0.1   setosa    :50  
 1st Qu.:5.10   1st Qu.:2.80   1st Qu.:1.60   1st Qu.:0.3   versicolor:50  
 Median :5.80   Median :3.00   Median :4.35   Median :1.3   virginica :50  
 Mean   :5.84   Mean   :3.06   Mean   :3.76   Mean   :1.2                  
 3rd Qu.:6.40   3rd Qu.:3.30   3rd Qu.:5.10   3rd Qu.:1.8                  
 Max.   :7.90   Max.   :4.40   Max.   :6.90   Max.   :2.5                  
\end{verbatim}

O comando \texttt{names()} lista os nomes das colunas dos bancos de
dados escolhidos, enquanto \texttt{tail()} mostra as últimas seis
linhas.

\begin{Shaded}
\begin{Highlighting}[]
\CommentTok{#Para visualizar os nomes das colunas dos dados:}
\KeywordTok{names}\NormalTok{(iris)}
\end{Highlighting}
\end{Shaded}

\begin{verbatim}
[1] "Sepal.Length" "Sepal.Width"  "Petal.Length" "Petal.Width"  "Species"     
\end{verbatim}

\begin{Shaded}
\begin{Highlighting}[]
\CommentTok{#vizualizar as ultimas seis linhas do objetos}
\KeywordTok{tail}\NormalTok{(iris)}
\end{Highlighting}
\end{Shaded}

\begin{verbatim}
    Sepal.Length Sepal.Width Petal.Length Petal.Width   Species
145          6.7         3.3          5.7         2.5 virginica
146          6.7         3.0          5.2         2.3 virginica
147          6.3         2.5          5.0         1.9 virginica
148          6.5         3.0          5.2         2.0 virginica
149          6.2         3.4          5.4         2.3 virginica
150          5.9         3.0          5.1         1.8 virginica
\end{verbatim}

Para que o pesquisador conheça melhor as bases de dados em que está
atuando, o comando \texttt{class()} serve para identificar o tipo de
base ou dados da base. Com o exemplo abaixo constata-se que o objeto
``iris'' é um \emph{data frame}, a variável ``Sepal.Length'' é uma
variável numérica e que a variável numérica.

\begin{Shaded}
\begin{Highlighting}[]
\KeywordTok{class}\NormalTok{(iris)}
\end{Highlighting}
\end{Shaded}

\begin{verbatim}
[1] "data.frame"
\end{verbatim}

\begin{Shaded}
\begin{Highlighting}[]
\KeywordTok{class}\NormalTok{(iris}\OperatorTok{$}\NormalTok{Sepal.Length)}
\end{Highlighting}
\end{Shaded}

\begin{verbatim}
[1] "numeric"
\end{verbatim}

\begin{Shaded}
\begin{Highlighting}[]
\KeywordTok{class}\NormalTok{(iris}\OperatorTok{$}\NormalTok{Especie)}
\end{Highlighting}
\end{Shaded}

\begin{verbatim}
[1] "NULL"
\end{verbatim}

Efeito semelhante possui o comando \texttt{ls.str()}:

\begin{Shaded}
\begin{Highlighting}[]
\KeywordTok{ls.str}\NormalTok{(iris)}
\end{Highlighting}
\end{Shaded}

\begin{verbatim}
Petal.Length :  num [1:150] 1.4 1.4 1.3 1.5 1.4 1.7 1.4 1.5 1.4 1.5 ...
Petal.Width :  num [1:150] 0.2 0.2 0.2 0.2 0.2 0.4 0.3 0.2 0.2 0.1 ...
Sepal.Length :  num [1:150] 5.1 4.9 4.7 4.6 5 5.4 4.6 5 4.4 4.9 ...
Sepal.Width :  num [1:150] 3.5 3 3.2 3.1 3.6 3.9 3.4 3.4 2.9 3.1 ...
Species :  Factor w/ 3 levels "setosa","versicolor",..: 1 1 1 1 1 1 1 1 1 1 ...
\end{verbatim}

Os comandos \texttt{ncol()} e \texttt{nrow()} mostram o número de
colunas e o número de linhas do objeto, respectivamente.

\hypertarget{funcoes-view-e-dim}{%
\subsection{\texorpdfstring{Funções \emph{View} e
\emph{dim}}{Funções View e dim}}\label{funcoes-view-e-dim}}

A função \texttt{View()} permite vizualizar os elementos no script do
dataframe requesitado, enquando a função \texttt{dim()} (abreviatura de
dimensões) fornece o número de linhas e de colunas, respectivamente.

\begin{Shaded}
\begin{Highlighting}[]
\KeywordTok{View}\NormalTok{(iris)}
\KeywordTok{dim}\NormalTok{(iris)}
\end{Highlighting}
\end{Shaded}

\begin{verbatim}
[1] 150   5
\end{verbatim}

Para alterar um nome de uma variável pode ser utilizado o comando
colnames. No exemplo acima, vamos alterar o nome da coluna ``Species''
para ``Especie''.

\begin{Shaded}
\begin{Highlighting}[]
\CommentTok{#Alterar o nome da coluna, sendo que o '[5]' indica que está na quinta coluna.}
\KeywordTok{colnames}\NormalTok{(iris)[}\DecValTok{5}\NormalTok{]=}\StringTok{'Especie'}
\end{Highlighting}
\end{Shaded}

Para selecionarmos uma coluna do objeto ``iris'', por exemplo a coluna
``Sepal.Length'', poderíamos digitar no console o comando
\textbf{iris\$Sepal.Length}. O padrão de carregamento da base de dados
nos obriga a dizer ao R qual é a base que quer selecionar (iris),
inserindo o símbolo \texttt{\$} e após o nome da coluna a qual deseja as
informações. Para criar um novo objeto com esta informação, basta dizer
ao R, como já visto acima, por exemplo:
\textbf{novoobjeto=iris\$novacoluna}.

No entanto, para acessar os dados sem o uso do símbolo \texttt{\$},
podemos usar o seguinte comando: \textbf{attach(iris)}. Assim, podemos
efetuar o sumário da coluna ``Petal.Width'':

\begin{Shaded}
\begin{Highlighting}[]
\CommentTok{#Definindo a função attach para o objeto 'dados'.}
\KeywordTok{attach}\NormalTok{(iris)}
\CommentTok{#Efetuando o sumário de 'pop.total'.}
\KeywordTok{summary}\NormalTok{(Petal.Width)}
\end{Highlighting}
\end{Shaded}

\begin{verbatim}
   Min. 1st Qu.  Median    Mean 3rd Qu.    Max. 
    0.1     0.3     1.3     1.2     1.8     2.5 
\end{verbatim}

\begin{Shaded}
\begin{Highlighting}[]
\CommentTok{#Como a coluna 'distrito' é um fator, o sumário será }
\CommentTok{#a contagem da quantidade de cada fator na coluna.}
\KeywordTok{summary}\NormalTok{(Especie)}
\end{Highlighting}
\end{Shaded}

\begin{verbatim}
    setosa versicolor  virginica 
        50         50         50 
\end{verbatim}

\hypertarget{funcao-tapply}{%
\subsection{\texorpdfstring{Função
\emph{tapply}}{Função tapply}}\label{funcao-tapply}}

O comando \texttt{tapply()} agrega os dados pelos níveis das variáveis
qualitativas. Note que a coluna ``Especie'' possui dados em forma de
fatores. Assim, para filtrarmos a informação (coluna ``Sepal.Length'')
média por Especie, podemos utilizar:

\begin{Shaded}
\begin{Highlighting}[]
\CommentTok{#Função 'tapply', número médio da população total por distrito.}
\KeywordTok{tapply}\NormalTok{(Sepal.Length, Especie, mean)}
\end{Highlighting}
\end{Shaded}

\begin{verbatim}
    setosa versicolor  virginica 
     5.006      5.936      6.588 
\end{verbatim}

No caso da coluna ``Sepal.Length'', se ela possuir um registro NA
(faltante), para que se efetue a média por este coluna neste quesito, há
que se adicionar o parâmetro \texttt{na.rm=T}, que ignora as células
faltantes para calcular-se a média:

\begin{Shaded}
\begin{Highlighting}[]
\CommentTok{#Função 'tapply' considerando NAs:}
\KeywordTok{tapply}\NormalTok{(Sepal.Length, Especie, mean)}
\end{Highlighting}
\end{Shaded}

\begin{verbatim}
    setosa versicolor  virginica 
     5.006      5.936      6.588 
\end{verbatim}

\begin{Shaded}
\begin{Highlighting}[]
\CommentTok{#Função 'tapply' sem considerar NAs:}
\KeywordTok{tapply}\NormalTok{(Sepal.Length, Especie, mean, }\DataTypeTok{na.rm=}\NormalTok{T)}
\end{Highlighting}
\end{Shaded}

\begin{verbatim}
    setosa versicolor  virginica 
     5.006      5.936      6.588 
\end{verbatim}

\hypertarget{funcao-subset}{%
\subsection{\texorpdfstring{Função
\emph{subset}}{Função subset}}\label{funcao-subset}}

Utiliza-se o comando \texttt{subset()} para formar um subconjunto de
dados o qual desejamos selecionar de um objeto. Por exemplo, se
quisermos criar um novo objeto com somente os dados da ``Especie''
setosa:

\begin{Shaded}
\begin{Highlighting}[]
\NormalTok{dadossetosa=}\KeywordTok{subset}\NormalTok{(iris, Especie}\OperatorTok{==}\StringTok{'setosa'}\NormalTok{)}
\KeywordTok{head}\NormalTok{(dadossetosa)}
\end{Highlighting}
\end{Shaded}

\begin{verbatim}
  Sepal.Length Sepal.Width Petal.Length Petal.Width Especie
1          5.1         3.5          1.4         0.2  setosa
2          4.9         3.0          1.4         0.2  setosa
3          4.7         3.2          1.3         0.2  setosa
4          4.6         3.1          1.5         0.2  setosa
5          5.0         3.6          1.4         0.2  setosa
6          5.4         3.9          1.7         0.4  setosa
\end{verbatim}

Pode ser configurado mais de uma condição para a filtragem dos dados,
por exemplo, além de serem filtrados os dados referentes a Especie
setosa, aquelas na qual o Sepal.Length é superior a 5. Como no exemplo,
criamos um novo objeto com estas condições:

\begin{Shaded}
\begin{Highlighting}[]
\NormalTok{dadossetosa2=}\KeywordTok{subset}\NormalTok{(iris, Especie}\OperatorTok{==}\StringTok{'setosa'}\OperatorTok{&}\StringTok{ }\NormalTok{Sepal.Length}\OperatorTok{>}\DecValTok{5}\NormalTok{)}
\KeywordTok{head}\NormalTok{(dadossetosa2)}
\end{Highlighting}
\end{Shaded}

\begin{verbatim}
   Sepal.Length Sepal.Width Petal.Length Petal.Width Especie
1           5.1         3.5          1.4         0.2  setosa
6           5.4         3.9          1.7         0.4  setosa
11          5.4         3.7          1.5         0.2  setosa
15          5.8         4.0          1.2         0.2  setosa
16          5.7         4.4          1.5         0.4  setosa
17          5.4         3.9          1.3         0.4  setosa
\end{verbatim}

\hypertarget{funcao-table}{%
\subsection{\texorpdfstring{Função
\emph{table}}{Função table}}\label{funcao-table}}

Para contar elementos em cada nível de um fator, usa-se a função
\texttt{table()}. A função pode fazer tabulações cruzadas, gerando uma
tabela de contingência, esse tipo de tabela é usado para registrar
observações independentes de duas ou mais variáveis aleatórias.

Para exemplo da utilização da função \texttt{table} agora com dados
qualitativos (gênero e saúde), vamos utilizar a base de dados
\texttt{cdc}:

\begin{Shaded}
\begin{Highlighting}[]
\CommentTok{# Carregando a base}
\KeywordTok{source}\NormalTok{(}\StringTok{"http://www.openintro.org/stat/data/cdc.R"}\NormalTok{)}
\CommentTok{#Vizualizamos as primeiras linhas}
\KeywordTok{head}\NormalTok{(cdc)}
\end{Highlighting}
\end{Shaded}

\begin{verbatim}
    genhlth exerany hlthplan smoke100 height weight wtdesire age gender
1      good       0        1        0     70    175      175  77      m
2      good       0        1        1     64    125      115  33      f
3      good       1        1        1     60    105      105  49      f
4      good       1        1        0     66    132      124  42      f
5 very good       0        1        0     61    150      130  55      f
6 very good       1        1        0     64    114      114  55      f
\end{verbatim}

\begin{Shaded}
\begin{Highlighting}[]
\CommentTok{# Efetuamos a contagem dos dados qualitativos com a função table}
\KeywordTok{table}\NormalTok{(cdc}\OperatorTok{$}\NormalTok{genhlth,cdc}\OperatorTok{$}\NormalTok{gender)}
\end{Highlighting}
\end{Shaded}

\begin{verbatim}
           
               m    f
  excellent 2298 2359
  very good 3382 3590
  good      2722 2953
  fair       884 1135
  poor       283  394
\end{verbatim}

\hypertarget{estrutura-de-dados}{%
\section{Estrutura de dados}\label{estrutura-de-dados}}

\hypertarget{vetores}{%
\subsection{Vetores}\label{vetores}}

Os fatores são uma classe especial de vetores, que definem variáveis
categóricas de classificação, como os tratamentos em um experimento
fatorial, ou categorias em uma tabela de contingência.

\begin{Shaded}
\begin{Highlighting}[]
\CommentTok{# Criação de um vetor}
\KeywordTok{c}\NormalTok{(}\DecValTok{2}\NormalTok{, }\DecValTok{4}\NormalTok{, }\DecValTok{6}\NormalTok{)}
\end{Highlighting}
\end{Shaded}

\begin{verbatim}
[1] 2 4 6
\end{verbatim}

Os vetores podem ser criados a partir de uma sequência numérica ou mesmo
de um intervalo entre valores:

\begin{Shaded}
\begin{Highlighting}[]
\KeywordTok{c}\NormalTok{(}\DecValTok{2}\OperatorTok{:}\DecValTok{6}\NormalTok{)}
\end{Highlighting}
\end{Shaded}

\begin{verbatim}
[1] 2 3 4 5 6
\end{verbatim}

\begin{Shaded}
\begin{Highlighting}[]
\CommentTok{# Criação de um vetor a partir do intervalo entre cada elemento e valores}
\CommentTok{#mínimo e máximo}
\KeywordTok{seq}\NormalTok{(}\DecValTok{2}\NormalTok{, }\DecValTok{3}\NormalTok{, }\DataTypeTok{by=}\FloatTok{0.5}\NormalTok{)}
\end{Highlighting}
\end{Shaded}

\begin{verbatim}
[1] 2.0 2.5 3.0
\end{verbatim}

Criação de um vetor atráves de uma repetição também é útil em várias
situações. No primeiro exemplo repete o intervalo de 1 a 3 4 vezes e no
segundo exemplo, a cada 3 vezes:

\begin{Shaded}
\begin{Highlighting}[]
\KeywordTok{rep}\NormalTok{(}\DecValTok{1}\OperatorTok{:}\DecValTok{3}\NormalTok{, }\DataTypeTok{times=}\DecValTok{4}\NormalTok{)}
\end{Highlighting}
\end{Shaded}

\begin{verbatim}
 [1] 1 2 3 1 2 3 1 2 3 1 2 3
\end{verbatim}

\begin{Shaded}
\begin{Highlighting}[]
\KeywordTok{rep}\NormalTok{(}\DecValTok{1}\OperatorTok{:}\DecValTok{3}\NormalTok{, }\DataTypeTok{each=}\DecValTok{3}\NormalTok{)}
\end{Highlighting}
\end{Shaded}

\begin{verbatim}
[1] 1 1 1 2 2 2 3 3 3
\end{verbatim}

A função factor cria um fator, a partir de um vetor:

\begin{Shaded}
\begin{Highlighting}[]
\NormalTok{sexo<-}\KeywordTok{factor}\NormalTok{(}\KeywordTok{rep}\NormalTok{(}\KeywordTok{c}\NormalTok{(}\StringTok{"F"}\NormalTok{, }\StringTok{"M"}\NormalTok{),}\DataTypeTok{each=}\DecValTok{8}\NormalTok{))}
\NormalTok{sexo}
\end{Highlighting}
\end{Shaded}

\begin{verbatim}
 [1] F F F F F F F F M M M M M M M M
Levels: F M
\end{verbatim}

\begin{Shaded}
\begin{Highlighting}[]
\NormalTok{numeros=}\KeywordTok{rep}\NormalTok{(}\DecValTok{1}\OperatorTok{:}\DecValTok{3}\NormalTok{,}\DataTypeTok{each=}\DecValTok{3}\NormalTok{)}
\NormalTok{numeros}
\end{Highlighting}
\end{Shaded}

\begin{verbatim}
[1] 1 1 1 2 2 2 3 3 3
\end{verbatim}

\begin{Shaded}
\begin{Highlighting}[]
\NormalTok{numeros.f<-}\KeywordTok{factor}\NormalTok{(numeros)}
\NormalTok{numeros.f}
\end{Highlighting}
\end{Shaded}

\begin{verbatim}
[1] 1 1 1 2 2 2 3 3 3
Levels: 1 2 3
\end{verbatim}

Fatores têm um atributo que especifica seus níveis ou categorias
(levels), que seguem ordem alfanumérica crescente, por \emph{default}.
Em muitas análises essa ordem é de fundamental importância e dessa forma
pode ser alterada através do argumento levels, por exemplo, para que
possa ser colocado o controle antes dos tratamentos:

\begin{Shaded}
\begin{Highlighting}[]
\NormalTok{tratamentos=}\KeywordTok{factor}\NormalTok{(}\KeywordTok{rep}\NormalTok{(}\KeywordTok{c}\NormalTok{(}\StringTok{"controle"}\NormalTok{,}\StringTok{"adubo A"}\NormalTok{,}\StringTok{"adubo B"}\NormalTok{), }\DataTypeTok{each=}\DecValTok{4}\NormalTok{))}
\NormalTok{tratamentos}
\end{Highlighting}
\end{Shaded}

\begin{verbatim}
 [1] controle controle controle controle adubo A  adubo A  adubo A  adubo A 
 [9] adubo B  adubo B  adubo B  adubo B 
Levels: adubo A adubo B controle
\end{verbatim}

\begin{Shaded}
\begin{Highlighting}[]
\NormalTok{tratamentos=}\KeywordTok{factor}\NormalTok{(}\KeywordTok{rep}\NormalTok{(}\KeywordTok{c}\NormalTok{(}\StringTok{"controle"}\NormalTok{,}\StringTok{"adubo A"}\NormalTok{,}\StringTok{"adubo B"}\NormalTok{), }\DataTypeTok{each=}\DecValTok{4}\NormalTok{), }
\DataTypeTok{levels=}\KeywordTok{c}\NormalTok{(}\StringTok{"controle"}\NormalTok{, }\StringTok{"adubo A"}\NormalTok{, }\StringTok{"adubo B"}\NormalTok{))}
\NormalTok{tratamentos}
\end{Highlighting}
\end{Shaded}

\begin{verbatim}
 [1] controle controle controle controle adubo A  adubo A  adubo A  adubo A 
 [9] adubo B  adubo B  adubo B  adubo B 
Levels: controle adubo A adubo B
\end{verbatim}

Fatores podem conter níveis não usados (vazios):

\begin{Shaded}
\begin{Highlighting}[]
\NormalTok{participantes=}\KeywordTok{factor}\NormalTok{(}\KeywordTok{rep}\NormalTok{(}\StringTok{"mulheres"}\NormalTok{,}\DecValTok{10}\NormalTok{), }\DataTypeTok{levels=}\KeywordTok{c}\NormalTok{(}\StringTok{"mulheres"}\NormalTok{,}\StringTok{"homens"}\NormalTok{))}
\NormalTok{participantes}
\end{Highlighting}
\end{Shaded}

\begin{verbatim}
 [1] mulheres mulheres mulheres mulheres mulheres mulheres mulheres mulheres
 [9] mulheres mulheres
Levels: mulheres homens
\end{verbatim}

\hypertarget{matrizes}{%
\subsection{Matrizes}\label{matrizes}}

A função matrix tem a finalidade de criar uma matriz com os valores do
argumento data, argumento este que insere as variáveis desejadas na
matriz. O número de linhas é definido pelo argumento nrow e o número de
colunas é definido pelo argumento ncol:

\begin{Shaded}
\begin{Highlighting}[]
\NormalTok{nome.da.matriz=}\StringTok{ }\KeywordTok{matrix}\NormalTok{(}\DataTypeTok{data=}\DecValTok{1}\OperatorTok{:}\DecValTok{12}\NormalTok{,}\DataTypeTok{nrow =} \DecValTok{3}\NormalTok{,}\DataTypeTok{ncol =} \DecValTok{4}\NormalTok{)}
\NormalTok{nome.da.matriz}
\end{Highlighting}
\end{Shaded}

\begin{verbatim}
     [,1] [,2] [,3] [,4]
[1,]    1    4    7   10
[2,]    2    5    8   11
[3,]    3    6    9   12
\end{verbatim}

Por \emph{default} (ação tomada pelo \emph{software}), os valores são
preenchidos por coluna. Para preencher por linha basta instruir o
programa de outra forma, alterando o argumento \texttt{byrow} para TRUE:

\begin{Shaded}
\begin{Highlighting}[]
\NormalTok{nome.da.matriz=}\StringTok{ }\KeywordTok{matrix}\NormalTok{(}\DataTypeTok{data=}\DecValTok{1}\OperatorTok{:}\DecValTok{12}\NormalTok{,}\DataTypeTok{nrow =} \DecValTok{3}\NormalTok{,}\DataTypeTok{ncol =} \DecValTok{4}\NormalTok{, }\DataTypeTok{byrow=}\NormalTok{T)}
\NormalTok{nome.da.matriz}
\end{Highlighting}
\end{Shaded}

\begin{verbatim}
     [,1] [,2] [,3] [,4]
[1,]    1    2    3    4
[2,]    5    6    7    8
[3,]    9   10   11   12
\end{verbatim}

Se a matriz inserida tem menos elementos do que a ordem informada para a
matriz, os são repetidos até preenchê-la:

\begin{Shaded}
\begin{Highlighting}[]
\NormalTok{lista=}\StringTok{ }\KeywordTok{list}\NormalTok{(}\DataTypeTok{matriz=}\KeywordTok{matrix}\NormalTok{(}\KeywordTok{c}\NormalTok{(}\DecValTok{1}\NormalTok{,}\DecValTok{2}\NormalTok{,}\DecValTok{1}\NormalTok{), }\DataTypeTok{nrow=}\DecValTok{3}\NormalTok{, }\DataTypeTok{ncol=}\DecValTok{2}\NormalTok{))}
\NormalTok{lista}
\end{Highlighting}
\end{Shaded}

\begin{verbatim}
$matriz
     [,1] [,2]
[1,]    1    1
[2,]    2    2
[3,]    1    1
\end{verbatim}

\hypertarget{listas}{%
\subsection{Listas}\label{listas}}

As listas podem ser criadas a partir do comando \texttt{list()}.

\begin{itemize}
\tightlist
\item
  \textbf{nrow}: corresponde ao número de linhas;
\item
  \textbf{ncol}: corresponde ao número de colunas.
\end{itemize}

Para ver quais elementos estão em suas listas é só chamar pelo nome que
foi dado para ela, como no exemplo abaixo. Representa uma coleção de
objetos.

\begin{Shaded}
\begin{Highlighting}[]
\NormalTok{lista=}\StringTok{ }\KeywordTok{list}\NormalTok{(}\DataTypeTok{matriz=}\KeywordTok{matrix}\NormalTok{(}\KeywordTok{c}\NormalTok{(}\DecValTok{1}\NormalTok{,}\DecValTok{2}\NormalTok{,}\DecValTok{1}\NormalTok{,}\DecValTok{5}\NormalTok{,}\DecValTok{7}\NormalTok{,}\DecValTok{9}\NormalTok{), }\DataTypeTok{nrow=}\DecValTok{3}\NormalTok{, }\DataTypeTok{ncol=}\DecValTok{2}\NormalTok{),}\DataTypeTok{vetor=}\DecValTok{1}\OperatorTok{:}\DecValTok{6}\NormalTok{)}
\NormalTok{lista}
\end{Highlighting}
\end{Shaded}

\begin{verbatim}
$matriz
     [,1] [,2]
[1,]    1    5
[2,]    2    7
[3,]    1    9

$vetor
[1] 1 2 3 4 5 6
\end{verbatim}

\textbf{Comandos para manipulação de listas}

Para descobrirmos de maneira rápida o números de objetos que há na
lista, utilizamos o comando \texttt{length(nomedalista)}.

\begin{Shaded}
\begin{Highlighting}[]
\NormalTok{lista}
\end{Highlighting}
\end{Shaded}

\begin{verbatim}
$matriz
     [,1] [,2]
[1,]    1    5
[2,]    2    7
[3,]    1    9

$vetor
[1] 1 2 3 4 5 6
\end{verbatim}

\begin{Shaded}
\begin{Highlighting}[]
\KeywordTok{length}\NormalTok{(lista)}
\end{Highlighting}
\end{Shaded}

\begin{verbatim}
[1] 2
\end{verbatim}

O uso do comando \texttt{names(nomedalista)} retorna os nomes dos
objetos que estão presentes na lista.

\begin{Shaded}
\begin{Highlighting}[]
\KeywordTok{names}\NormalTok{(lista)}
\end{Highlighting}
\end{Shaded}

\begin{verbatim}
[1] "matriz" "vetor" 
\end{verbatim}

Para chamar várias listas através usamos o comando da seguinte forma:

\texttt{c(nome1,\ nome2)}

\begin{Shaded}
\begin{Highlighting}[]
\NormalTok{lista}\FloatTok{.1}\NormalTok{=}\StringTok{ }\KeywordTok{list}\NormalTok{(}\DataTypeTok{matriz=}\KeywordTok{matrix}\NormalTok{(}\KeywordTok{c}\NormalTok{(}\DecValTok{1}\NormalTok{,}\DecValTok{2}\NormalTok{,}\DecValTok{1}\NormalTok{,}\DecValTok{5}\NormalTok{,}\DecValTok{7}\NormalTok{,}\DecValTok{9}\NormalTok{), }\DataTypeTok{nrow=}\DecValTok{3}\NormalTok{, }\DataTypeTok{ncol=}\DecValTok{2}\NormalTok{),}
              \DataTypeTok{vetor=}\DecValTok{1}\OperatorTok{:}\DecValTok{6}\NormalTok{)}
\NormalTok{lista}\FloatTok{.2}\NormalTok{=}\StringTok{ }\KeywordTok{list}\NormalTok{(}\DataTypeTok{nomes=}\KeywordTok{c}\NormalTok{(}\StringTok{"Marcelo"}\NormalTok{, }\StringTok{"Fábio"}\NormalTok{, }\StringTok{"Felipe"}\NormalTok{), }
              \DataTypeTok{idade=}\KeywordTok{c}\NormalTok{(}\DecValTok{25}\NormalTok{, }\DecValTok{34}\NormalTok{, }\DecValTok{26}\NormalTok{))}
\KeywordTok{c}\NormalTok{(lista}\FloatTok{.1}\NormalTok{,lista}\FloatTok{.2}\NormalTok{)}
\end{Highlighting}
\end{Shaded}

\begin{verbatim}
$matriz
     [,1] [,2]
[1,]    1    5
[2,]    2    7
[3,]    1    9

$vetor
[1] 1 2 3 4 5 6

$nomes
[1] "Marcelo" "Fábio"   "Felipe" 

$idade
[1] 25 34 26
\end{verbatim}

\hypertarget{data-frames}{%
\subsection{Data frames}\label{data-frames}}

Com a função \texttt{data.frame()} reunimos vetores de mesmo comprimento
em um só objeto. Neste caso são criadas tabelas de dados. Cada
observação é descrita por um conjunto de propriedades. Abaixo podemos
ver como inserir os dados para criar a ``tabela''. Similar como
matrizes, porem diferentes colunas podem possuir elementos de natureza
diferentes .

\begin{Shaded}
\begin{Highlighting}[]
\NormalTok{estudantes=}\StringTok{ }\KeywordTok{c}\NormalTok{(}\StringTok{"Camila"}\NormalTok{, }\StringTok{"Pedro"}\NormalTok{, }\StringTok{"Marcelo"}\NormalTok{,}\StringTok{"Guilherme"}\NormalTok{)}
\NormalTok{idade=}\KeywordTok{c}\NormalTok{(}\DecValTok{21}\NormalTok{,}\DecValTok{17}\NormalTok{,}\DecValTok{17}\NormalTok{,}\DecValTok{18}\NormalTok{)}
\NormalTok{peso=}\KeywordTok{c}\NormalTok{(}\DecValTok{65}\NormalTok{,}\DecValTok{79}\NormalTok{,}\DecValTok{80}\NormalTok{,}\DecValTok{100}\NormalTok{)}
\NormalTok{informacoes=}\KeywordTok{data.frame}\NormalTok{(estudantes,idade,peso)}
\NormalTok{informacoes}
\end{Highlighting}
\end{Shaded}

\begin{verbatim}
  estudantes idade peso
1     Camila    21   65
2      Pedro    17   79
3    Marcelo    17   80
4  Guilherme    18  100
\end{verbatim}

Adiciona-se colunas no \emph{data frame} através do comando a seguir,
pressupondo que a ordem dos dados esteja correta:

\texttt{nomedodata.frame\$variávelaseradicionada}

\begin{Shaded}
\begin{Highlighting}[]
\NormalTok{informacoes}\OperatorTok{$}\NormalTok{cidades=}\KeywordTok{c}\NormalTok{(}\StringTok{"Nova Hartz"}\NormalTok{,}\StringTok{"Gramado"}\NormalTok{,}\StringTok{"Soledade"}\NormalTok{,}
                      \StringTok{"Porto Alegre"}\NormalTok{)}
\NormalTok{informacoes}
\end{Highlighting}
\end{Shaded}

\begin{verbatim}
  estudantes idade peso      cidades
1     Camila    21   65   Nova Hartz
2      Pedro    17   79      Gramado
3    Marcelo    17   80     Soledade
4  Guilherme    18  100 Porto Alegre
\end{verbatim}

É possível fazer uma contagem concatenando com a filtragem do pacote
\texttt{subset}, como no exemplo a contagem dos indivíduos cuja origem é
Soledade.

\begin{Shaded}
\begin{Highlighting}[]
\KeywordTok{length}\NormalTok{(}\KeywordTok{subset}\NormalTok{(informacoes}\OperatorTok{$}\NormalTok{cidades, informacoes}\OperatorTok{$}\NormalTok{cidades}\OperatorTok{==}\StringTok{"Soledade"}\NormalTok{))}
\end{Highlighting}
\end{Shaded}

\begin{verbatim}
[1] 1
\end{verbatim}

\hypertarget{manipulacao-de-banco-de-dados}{%
\section{Manipulação de banco de
dados}\label{manipulacao-de-banco-de-dados}}

A função \texttt{edit()} abre uma interface simples de edição de dados
em formato planilha, e é útil para pequenas modificações. Mas para
salvar as modificações atribua o resultado da função \texttt{edit} a um
objeto.

Utiliza-se o comando da seguinte forma:

\texttt{novonomedabase\ =\ edit(nomeatualdabase)}

\begin{Shaded}
\begin{Highlighting}[]
\NormalTok{informacoes}\FloatTok{.2}\NormalTok{=}\KeywordTok{edit}\NormalTok{(informacoes)}
\end{Highlighting}
\end{Shaded}

\begin{figure}[H]

{\centering \includegraphics[width=0.8\linewidth]{95} 

}

\caption{Editor de dados}\label{fig:95}
\end{figure}

Basta clicar no retângulo correspondente a variável que deseja ser
modificada, excluir ou adicionar novas colunas.

\begin{figure}[H]

{\centering \includegraphics[width=0.8\linewidth]{10} 

}

\caption{Acréscimo de uma nova coluna através do editor de dados}\label{fig:10}
\end{figure}

Logo, chamando o novo banco de dados, teremos:

\begin{Shaded}
\begin{Highlighting}[]
\NormalTok{informacoes}\FloatTok{.2} 
\end{Highlighting}
\end{Shaded}

\begin{verbatim}
  estudantes idade peso      cidades
1     Camila    21   65   Nova Hartz
2      Pedro    17   79      Gramado
3    Marcelo    17   80     Soledade
4  Guilherme    18  100 Porto Alegre
\end{verbatim}

As funções a seguir são aplicáveis a vetores, data.frames e listas, e em
muitos casos trazem praticidade a uma análise estatística. Foram criados
objetos com informações do nome dos estudantes e altura. Segue o
processo de criação do \emph{data frame} com estas informações,
lembrando que esta forma de ``união'' das informações pressupõe que a
ordem dos dados esteja correta:

\begin{Shaded}
\begin{Highlighting}[]
\CommentTok{# Crição do data frame}
\NormalTok{estudantes=}\KeywordTok{c}\NormalTok{(}\StringTok{"Guilherme"}\NormalTok{, }\StringTok{"Marcelo"}\NormalTok{, }\StringTok{"Pedro"}\NormalTok{, }\StringTok{"Camila"}\NormalTok{)}
\NormalTok{altura=}\StringTok{ }\KeywordTok{c}\NormalTok{(}\FloatTok{1.50}\NormalTok{, }\FloatTok{1.9}\NormalTok{, }\FloatTok{1.74}\NormalTok{, }\FloatTok{1.80}\NormalTok{)}
\NormalTok{informacoes}\FloatTok{.3}\NormalTok{=}\KeywordTok{data.frame}\NormalTok{(estudantes, altura)}
\end{Highlighting}
\end{Shaded}

Já o comando \texttt{merge()} serve para juntar dois \emph{data frames}
que possuam uma coluna em comum. Neste caso, unimos o objeto
\texttt{informações.2} com o objeto \texttt{informações.3} utilizando o
nome dos estudantes (informação em comum):

\begin{Shaded}
\begin{Highlighting}[]
\CommentTok{# União de um banco de dados (existencia de uma váriavel em comum)}

\NormalTok{informacoes=}\KeywordTok{merge}\NormalTok{(informacoes}\FloatTok{.2}\NormalTok{,informacoes}\FloatTok{.3}\NormalTok{, }\DataTypeTok{by=}\StringTok{"estudantes"}\NormalTok{)}
\end{Highlighting}
\end{Shaded}

Adicionar um cálculo entre as colunas é muito simples com o RStudio,
neste caso com os dados do peso e altura, pode-se calcular o IMC (Índice
de Massa Corporal) em uma nova coluna:

\begin{Shaded}
\begin{Highlighting}[]
\NormalTok{informacoes}\OperatorTok{$}\NormalTok{Imc=}\KeywordTok{c}\NormalTok{(informacoes}\OperatorTok{$}\NormalTok{peso}\OperatorTok{/}\NormalTok{(informacoes}\OperatorTok{$}\NormalTok{altura}\OperatorTok{^}\DecValTok{2}\NormalTok{))}
\NormalTok{informacoes}
\end{Highlighting}
\end{Shaded}

\begin{verbatim}
  estudantes idade peso      cidades altura   Imc
1     Camila    21   65   Nova Hartz   1.80 20.06
2  Guilherme    18  100 Porto Alegre   1.50 44.44
3    Marcelo    17   80     Soledade   1.90 22.16
4      Pedro    17   79      Gramado   1.74 26.09
\end{verbatim}

Ainda, se houver linhas que tenham pelo menos uma informação faltante
(NA), estas podem ser excluídas com o comando \texttt{na.omit()}, ou
mesmo os NAs serem substituídos por outro caractere (neste caso foi
substituído por zero) com o comando \texttt{is.na}:

\begin{Shaded}
\begin{Highlighting}[]
\CommentTok{# Retirar as linhas que tenham pelo menos um NA:}

\NormalTok{informacoes<-}\StringTok{ }\KeywordTok{na.omit}\NormalTok{(informacoes)}
\NormalTok{informacoes}
\end{Highlighting}
\end{Shaded}

\begin{verbatim}
  estudantes idade peso      cidades altura   Imc
1     Camila    21   65   Nova Hartz   1.80 20.06
2  Guilherme    18  100 Porto Alegre   1.50 44.44
3    Marcelo    17   80     Soledade   1.90 22.16
4      Pedro    17   79      Gramado   1.74 26.09
\end{verbatim}

\begin{Shaded}
\begin{Highlighting}[]
\CommentTok{# Substituir NA's por zero no data.frame}

\NormalTok{informacoes[}\KeywordTok{is.na}\NormalTok{(informacoes)] =}\StringTok{ }\DecValTok{0}
\NormalTok{informacoes}
\end{Highlighting}
\end{Shaded}

\begin{verbatim}
  estudantes idade peso      cidades altura   Imc
1     Camila    21   65   Nova Hartz   1.80 20.06
2  Guilherme    18  100 Porto Alegre   1.50 44.44
3    Marcelo    17   80     Soledade   1.90 22.16
4      Pedro    17   79      Gramado   1.74 26.09
\end{verbatim}

Outro recurso interessante é a substituição de dados em uma coluna, que
pode ser feito de forma automática para uma condição padrão escolhida.
No exemplo abaixo, substituimos aquelas informações de idade igual a 17
pelo número 19:

\begin{Shaded}
\begin{Highlighting}[]
\CommentTok{# Substituir números na coluna}
\NormalTok{informacoes}\OperatorTok{$}\NormalTok{idade[informacoes}\OperatorTok{$}\NormalTok{idade }\OperatorTok{==}\StringTok{ }\DecValTok{17}\NormalTok{] <-}\StringTok{ }\DecValTok{19}
\NormalTok{informacoes}
\end{Highlighting}
\end{Shaded}

\begin{verbatim}
  estudantes idade peso      cidades altura   Imc
1     Camila    21   65   Nova Hartz   1.80 20.06
2  Guilherme    18  100 Porto Alegre   1.50 44.44
3    Marcelo    19   80     Soledade   1.90 22.16
4      Pedro    19   79      Gramado   1.74 26.09
\end{verbatim}

A classificação qualitativa das informações, com base em condições
definidas pelo usuário podem ser facilmente efetuadas pelo comando
\texttt{ifelse}. Para quem não tem intimidade com atributos de
programação, este comando seleciona ``se'' (\emph{if}) uma informação
desejada é atendida, e cria uma rotina (\emph{else}) que será aplicada
``então''.

No nosso exemplo, cria-se um objeto ``classificacao'' e se a coluna IMC
conter dados acima de 25, será marcado como ``peso normal'', sendo que
do contrário, constará como ``excesso de peso''. Após utilizamos o
comando \texttt{cbind()} para unir os dois objetos pelas colunas. caso
não queira utilizar o comando \texttt{cbind()}, poderia ser criado uma
nova coluna com o nome do obetjo sendo ``informacoes\$classificacao''.

\begin{Shaded}
\begin{Highlighting}[]
\CommentTok{# Classificar qualitativamente informações em um determinado intervalo }
\NormalTok{classificacao=}\KeywordTok{ifelse}\NormalTok{(informacoes}\OperatorTok{$}\NormalTok{Imc}\OperatorTok{<}\DecValTok{25}\NormalTok{, }\StringTok{"peso normal"}\NormalTok{, }
                     \StringTok{"excesso de peso"}\NormalTok{)}
\NormalTok{informacoes=}\KeywordTok{cbind}\NormalTok{(informacoes, classificacao)}
\NormalTok{informacoes}
\end{Highlighting}
\end{Shaded}

\begin{verbatim}
  estudantes idade peso      cidades altura   Imc   classificacao
1     Camila    21   65   Nova Hartz   1.80 20.06     peso normal
2  Guilherme    18  100 Porto Alegre   1.50 44.44 excesso de peso
3    Marcelo    19   80     Soledade   1.90 22.16     peso normal
4      Pedro    19   79      Gramado   1.74 26.09 excesso de peso
\end{verbatim}

\begin{table}

\caption{\label{tab:imct}Valores padrão para o IMC}
\centering
\begin{tabular}[t]{l|l}
\hline
Resultado & Significado\\
\hline
Abaixo de 17 & Muito abaixo do peso\\
\hline
Entre 17 e 18,49 & Abaixo do peso\\
\hline
Entre 18,5 e 24,99 & Peso normal\\
\hline
Entre 25 e 29,99 & Acima do peso\\
\hline
Entre 30 e 34,99 & Obesidade I\\
\hline
Entre 35 e 39,99 & Obesidade II (severa)\\
\hline
Acima de 40 & Obesidade III (mórbida)\\
\hline
\end{tabular}
\end{table}

No entanto, o IMC possui várias classificações de acordo com o seu
resultado (Tabela \ref{tab:imct}), sendo que, por exemplo, resultados
abaixo de 17 informam que o indivíduo se encontra como Muito abaixo do
peso, e acima de 40, se encontra em Obesidade III. Para efetuar a
classificação desta maneira utilizando o comando \texttt{ifelse}, ou
seja, com mais de uma condição, pode ser efetuada a estruturação com a
aglutinação do comando:

\begin{Shaded}
\begin{Highlighting}[]
\NormalTok{informacoes}\OperatorTok{$}\NormalTok{tipoimc=}\KeywordTok{ifelse}\NormalTok{(informacoes}\OperatorTok{$}\NormalTok{Imc}\OperatorTok{<}\DecValTok{17}\NormalTok{, }\StringTok{"Muito abaixo do peso"}\NormalTok{,}
\KeywordTok{ifelse}\NormalTok{(informacoes}\OperatorTok{$}\NormalTok{Imc}\OperatorTok{>=}\DecValTok{17}\OperatorTok{&}\NormalTok{informacoes}\OperatorTok{$}\NormalTok{Imc}\OperatorTok{<=}\FloatTok{18.49}\NormalTok{,}\StringTok{"Abaixo do peso"}\NormalTok{,}
\KeywordTok{ifelse}\NormalTok{(informacoes}\OperatorTok{$}\NormalTok{Imc}\OperatorTok{>=}\FloatTok{18.5}\OperatorTok{&}\NormalTok{informacoes}\OperatorTok{$}\NormalTok{Imc}\OperatorTok{<=}\FloatTok{24.99}\NormalTok{,}\StringTok{"Peso Normal"}\NormalTok{,}
\KeywordTok{ifelse}\NormalTok{(informacoes}\OperatorTok{$}\NormalTok{Imc}\OperatorTok{>=}\DecValTok{25}\OperatorTok{&}\NormalTok{informacoes}\OperatorTok{$}\NormalTok{Imc}\OperatorTok{<=}\FloatTok{29.99}\NormalTok{,}\StringTok{"Acima do Peso"}\NormalTok{,}
\KeywordTok{ifelse}\NormalTok{(informacoes}\OperatorTok{$}\NormalTok{Imc}\OperatorTok{>=}\DecValTok{30}\OperatorTok{&}\NormalTok{informacoes}\OperatorTok{$}\NormalTok{Imc}\OperatorTok{<=}\FloatTok{34.99}\NormalTok{,}\StringTok{"Obesidade I"}\NormalTok{,}
\KeywordTok{ifelse}\NormalTok{(informacoes}\OperatorTok{$}\NormalTok{Imc}\OperatorTok{>=}\DecValTok{35}\OperatorTok{&}\NormalTok{informacoes}\OperatorTok{$}\NormalTok{Imc}\OperatorTok{<=}\FloatTok{39.99}\NormalTok{,}\StringTok{"Obesidade II"}\NormalTok{,}
       \StringTok{"Obesidade III"}\NormalTok{))))))}
\NormalTok{informacoes}
\end{Highlighting}
\end{Shaded}

\begin{verbatim}
  estudantes idade peso      cidades altura   Imc   classificacao       tipoimc
1     Camila    21   65   Nova Hartz   1.80 20.06     peso normal   Peso Normal
2  Guilherme    18  100 Porto Alegre   1.50 44.44 excesso de peso Obesidade III
3    Marcelo    19   80     Soledade   1.90 22.16     peso normal   Peso Normal
4      Pedro    19   79      Gramado   1.74 26.09 excesso de peso Acima do Peso
\end{verbatim}

A classificação binária dos dados (0,1) também é relevante para o estudo
da manipulação dos dados trabalhados pelo pesquisador. Neste exemplo,
classificou-se aqueles valores da coluna ``classificacao'' com o ``peso
normal'' iguais a 1, do contrário classificou-se 0 (zero).

\begin{Shaded}
\begin{Highlighting}[]
\CommentTok{# Classificar informações usando o código binário}
\NormalTok{informacoes}\OperatorTok{$}\NormalTok{binario=}\StringTok{ }\KeywordTok{ifelse}\NormalTok{(informacoes}\OperatorTok{$}\NormalTok{classificacao }
                            \OperatorTok{==}\StringTok{ 'peso normal'}\NormalTok{, }\DecValTok{1}\NormalTok{, }\DecValTok{0}\NormalTok{) }
\NormalTok{informacoes}
\end{Highlighting}
\end{Shaded}

\begin{verbatim}
  estudantes idade peso      cidades altura   Imc   classificacao       tipoimc
1     Camila    21   65   Nova Hartz   1.80 20.06     peso normal   Peso Normal
2  Guilherme    18  100 Porto Alegre   1.50 44.44 excesso de peso Obesidade III
3    Marcelo    19   80     Soledade   1.90 22.16     peso normal   Peso Normal
4      Pedro    19   79      Gramado   1.74 26.09 excesso de peso Acima do Peso
  binario
1       1
2       0
3       1
4       0
\end{verbatim}

O comando \texttt{rbind()} é utilizado para incluir linhas novas abaixo
de um objeto já criado pelo pesquisador, sendo que é importante o
cuidado de que estas novas informações tenham os mesmos campos
(colunas). A exemplo, pede-se para incluir uma nova pessoa no \emph{data
frame} informacoes: Francisco, 30 anos de idade, peso 59, natural de
Ijuí, IMC 23,33768, classificado como peso normal. Lembrando de incluir
os campos ``tipoimc'' e ``binario''.

\begin{Shaded}
\begin{Highlighting}[]
\NormalTok{novo1=}\KeywordTok{data.frame}\NormalTok{(}\DataTypeTok{estudantes=}\StringTok{"Francisco"}\NormalTok{, }\DataTypeTok{idade=}\DecValTok{30}\NormalTok{, }\DataTypeTok{peso=}\DecValTok{59}\NormalTok{, }
                 \DataTypeTok{cidades=}\StringTok{"Ijuí"}\NormalTok{, }
                 \DataTypeTok{altura=}\StringTok{"1,59"}\NormalTok{, }
                 \DataTypeTok{Imc=} \FloatTok{23.33768}\NormalTok{, }
                 \DataTypeTok{classificacao=} \StringTok{"peso normal"}\NormalTok{,}
                 \DataTypeTok{tipoimc=}\StringTok{"Peso Normal"}\NormalTok{, }
                 \DataTypeTok{binario=}\DecValTok{1}\NormalTok{)}
\NormalTok{informacoes=}\KeywordTok{rbind}\NormalTok{(informacoes, novo1)}
\NormalTok{informacoes}
\end{Highlighting}
\end{Shaded}

\begin{verbatim}
  estudantes idade peso      cidades altura   Imc   classificacao       tipoimc
1     Camila    21   65   Nova Hartz    1.8 20.06     peso normal   Peso Normal
2  Guilherme    18  100 Porto Alegre    1.5 44.44 excesso de peso Obesidade III
3    Marcelo    19   80     Soledade    1.9 22.16     peso normal   Peso Normal
4      Pedro    19   79      Gramado   1.74 26.09 excesso de peso Acima do Peso
5  Francisco    30   59         Ijuí   1,59 23.34     peso normal   Peso Normal
  binario
1       1
2       0
3       1
4       0
5       1
\end{verbatim}

Outra forma de incluir informações adicionais nos \emph{data frames}
através de atributos é utilizando o pacote \texttt{dplyr}. Decide-se
criar um campo ``faixa etária'', sendo que aqueles indivíduos com idade
acima de 21 chamaremos de ``adulto'' e do contrário ``não adulto''.

\begin{Shaded}
\begin{Highlighting}[]
\KeywordTok{require}\NormalTok{(dplyr)}
\end{Highlighting}
\end{Shaded}

\begin{verbatim}
Carregando pacotes exigidos: dplyr
\end{verbatim}

\begin{verbatim}

Attaching package: 'dplyr'
\end{verbatim}

\begin{verbatim}
The following objects are masked from 'package:stats':

    filter, lag
\end{verbatim}

\begin{verbatim}
The following objects are masked from 'package:base':

    intersect, setdiff, setequal, union
\end{verbatim}

\begin{Shaded}
\begin{Highlighting}[]
\NormalTok{informacoes=}\StringTok{ }\KeywordTok{mutate}\NormalTok{(informacoes, }
                    \StringTok{"faixa etaria"}\NormalTok{=}\StringTok{ }\KeywordTok{ifelse}\NormalTok{(informacoes}\OperatorTok{$}\NormalTok{idade}\OperatorTok{<}\DecValTok{21}\NormalTok{,}
                                           \StringTok{"não adulto"}\NormalTok{, }\StringTok{"adulto"}\NormalTok{))}
\NormalTok{informacoes}
\end{Highlighting}
\end{Shaded}

\begin{verbatim}
  estudantes idade peso      cidades altura   Imc   classificacao       tipoimc
1     Camila    21   65   Nova Hartz    1.8 20.06     peso normal   Peso Normal
2  Guilherme    18  100 Porto Alegre    1.5 44.44 excesso de peso Obesidade III
3    Marcelo    19   80     Soledade    1.9 22.16     peso normal   Peso Normal
4      Pedro    19   79      Gramado   1.74 26.09 excesso de peso Acima do Peso
5  Francisco    30   59         Ijuí   1,59 23.34     peso normal   Peso Normal
  binario faixa etaria
1       1       adulto
2       0   não adulto
3       1   não adulto
4       0   não adulto
5       1       adulto
\end{verbatim}

A (re)ordenação das colunas de um \emph{data frame} pode ser muito útil
em alguns casos, sendo extremamente fácil efetuá-la, cada número
representa o número da respectiva coluna:

\begin{Shaded}
\begin{Highlighting}[]
\CommentTok{# Reordenar colunas}
\NormalTok{informacoes=informacoes[}\KeywordTok{c}\NormalTok{(}\DecValTok{8}\NormalTok{,}\DecValTok{2}\NormalTok{,}\DecValTok{3}\NormalTok{,}\DecValTok{4}\NormalTok{,}\DecValTok{1}\NormalTok{,}\DecValTok{6}\NormalTok{,}\DecValTok{5}\NormalTok{,}\DecValTok{7}\NormalTok{,}\DecValTok{9}\NormalTok{,}\DecValTok{10}\NormalTok{)]}
\end{Highlighting}
\end{Shaded}

Caso se queira a inversão total da ordem das colunas do objeto estudado,
o comando \texttt{rev()} pode ser útil:

\begin{Shaded}
\begin{Highlighting}[]
\CommentTok{# Inversão do posicionamento dos elementos}
\KeywordTok{rev}\NormalTok{(informacoes)}
\end{Highlighting}
\end{Shaded}

\begin{verbatim}
  faixa etaria binario   classificacao altura   Imc estudantes      cidades
1       adulto       1     peso normal    1.8 20.06     Camila   Nova Hartz
2   não adulto       0 excesso de peso    1.5 44.44  Guilherme Porto Alegre
3   não adulto       1     peso normal    1.9 22.16    Marcelo     Soledade
4   não adulto       0 excesso de peso   1.74 26.09      Pedro      Gramado
5       adulto       1     peso normal   1,59 23.34  Francisco         Ijuí
  peso idade       tipoimc
1   65    21   Peso Normal
2  100    18 Obesidade III
3   80    19   Peso Normal
4   79    19 Acima do Peso
5   59    30   Peso Normal
\end{verbatim}

A função \texttt{table()} faz a contagem os dados; já o comando
\texttt{sort()} ordena os objetos em ordem crescente (caso queira no
formato decrescente, informar \texttt{decreasing=TRUE}).

\begin{Shaded}
\begin{Highlighting}[]
\CommentTok{# contagem de objetos}
\KeywordTok{table}\NormalTok{(informacoes}\OperatorTok{$}\NormalTok{classificacao)}
\end{Highlighting}
\end{Shaded}

\begin{verbatim}

excesso de peso     peso normal 
              2               3 
\end{verbatim}

\begin{Shaded}
\begin{Highlighting}[]
\CommentTok{# Ordenar os objetos em ordem crescente}
\KeywordTok{sort}\NormalTok{(informacoes}\OperatorTok{$}\NormalTok{idade)}
\end{Highlighting}
\end{Shaded}

\begin{verbatim}
[1] 18 19 19 21 30
\end{verbatim}

A ordenação de todo o \emph{data frame} a partir de uma variável, pode
ser realizada utilizando o comando \texttt{order}, sendo que pode ser
realizada inclusive com variáveis categóricas (no exemplo abaixo o nome
das cidades).

\begin{Shaded}
\begin{Highlighting}[]
\CommentTok{# Ordem decrescente }
\NormalTok{informacoes[}\KeywordTok{order}\NormalTok{(informacoes}\OperatorTok{$}\NormalTok{idade, }\DataTypeTok{decreasing =} \OtherTok{TRUE}\NormalTok{),]}
\end{Highlighting}
\end{Shaded}

\begin{verbatim}
        tipoimc idade peso      cidades estudantes   Imc altura   classificacao
5   Peso Normal    30   59         Ijuí  Francisco 23.34   1,59     peso normal
1   Peso Normal    21   65   Nova Hartz     Camila 20.06    1.8     peso normal
3   Peso Normal    19   80     Soledade    Marcelo 22.16    1.9     peso normal
4 Acima do Peso    19   79      Gramado      Pedro 26.09   1.74 excesso de peso
2 Obesidade III    18  100 Porto Alegre  Guilherme 44.44    1.5 excesso de peso
  binario faixa etaria
5       1       adulto
1       1       adulto
3       1   não adulto
4       0   não adulto
2       0   não adulto
\end{verbatim}

\begin{Shaded}
\begin{Highlighting}[]
\CommentTok{#ordem crescente}
\NormalTok{informacoes[}\KeywordTok{order}\NormalTok{(informacoes}\OperatorTok{$}\NormalTok{idade, }\DataTypeTok{decreasing =} \OtherTok{FALSE}\NormalTok{),]}
\end{Highlighting}
\end{Shaded}

\begin{verbatim}
        tipoimc idade peso      cidades estudantes   Imc altura   classificacao
2 Obesidade III    18  100 Porto Alegre  Guilherme 44.44    1.5 excesso de peso
3   Peso Normal    19   80     Soledade    Marcelo 22.16    1.9     peso normal
4 Acima do Peso    19   79      Gramado      Pedro 26.09   1.74 excesso de peso
1   Peso Normal    21   65   Nova Hartz     Camila 20.06    1.8     peso normal
5   Peso Normal    30   59         Ijuí  Francisco 23.34   1,59     peso normal
  binario faixa etaria
2       0   não adulto
3       1   não adulto
4       0   não adulto
1       1       adulto
5       1       adulto
\end{verbatim}

\begin{Shaded}
\begin{Highlighting}[]
\CommentTok{#ordem crescente}
\NormalTok{informacoes[}\KeywordTok{order}\NormalTok{(informacoes}\OperatorTok{$}\NormalTok{cidades, }\DataTypeTok{decreasing =} \OtherTok{FALSE}\NormalTok{),]}
\end{Highlighting}
\end{Shaded}

\begin{verbatim}
        tipoimc idade peso      cidades estudantes   Imc altura   classificacao
4 Acima do Peso    19   79      Gramado      Pedro 26.09   1.74 excesso de peso
5   Peso Normal    30   59         Ijuí  Francisco 23.34   1,59     peso normal
1   Peso Normal    21   65   Nova Hartz     Camila 20.06    1.8     peso normal
2 Obesidade III    18  100 Porto Alegre  Guilherme 44.44    1.5 excesso de peso
3   Peso Normal    19   80     Soledade    Marcelo 22.16    1.9     peso normal
  binario faixa etaria
4       0   não adulto
5       1       adulto
1       1       adulto
2       0   não adulto
3       1   não adulto
\end{verbatim}

O comando \texttt{rank()} cria uma ranqueamento crescente das
informações. Se pretende-se, por exemplo, criar uma coluna com o ranking
dos valores do IMC, pode ser utilizado:

\begin{Shaded}
\begin{Highlighting}[]
\NormalTok{informacoes}\OperatorTok{$}\NormalTok{rankingImc=}\KeywordTok{rank}\NormalTok{(informacoes}\OperatorTok{$}\NormalTok{Imc)}
\NormalTok{informacoes}
\end{Highlighting}
\end{Shaded}

\begin{verbatim}
        tipoimc idade peso      cidades estudantes   Imc altura   classificacao
1   Peso Normal    21   65   Nova Hartz     Camila 20.06    1.8     peso normal
2 Obesidade III    18  100 Porto Alegre  Guilherme 44.44    1.5 excesso de peso
3   Peso Normal    19   80     Soledade    Marcelo 22.16    1.9     peso normal
4 Acima do Peso    19   79      Gramado      Pedro 26.09   1.74 excesso de peso
5   Peso Normal    30   59         Ijuí  Francisco 23.34   1,59     peso normal
  binario faixa etaria rankingImc
1       1       adulto          1
2       0   não adulto          5
3       1   não adulto          2
4       0   não adulto          4
5       1       adulto          3
\end{verbatim}

Para utilizar a função \texttt{rank} com os maiores valores em primeiro
lugar:

\begin{Shaded}
\begin{Highlighting}[]
\KeywordTok{rank}\NormalTok{(}\OperatorTok{-}\NormalTok{informacoes}\OperatorTok{$}\NormalTok{Imc)}
\end{Highlighting}
\end{Shaded}

\begin{verbatim}
[1] 5 1 4 2 3
\end{verbatim}

\hypertarget{funcoes-matematicas}{%
\section{Funções Matemáticas}\label{funcoes-matematicas}}

A utilização de funções matemáticasno RStudio contribui para que o
pesquisador possa realizar vários experimentos com seus dados. Os
cálculos podem ser efetuados diretamente no console do programa ou
aplicados aos objetos criados:

\begin{Shaded}
\begin{Highlighting}[]
\KeywordTok{log}\NormalTok{(}\FloatTok{1.5}\NormalTok{)}
\end{Highlighting}
\end{Shaded}

\begin{verbatim}
[1] 0.4055
\end{verbatim}

\begin{Shaded}
\begin{Highlighting}[]
\KeywordTok{exp}\NormalTok{(}\DecValTok{1}\NormalTok{)}
\end{Highlighting}
\end{Shaded}

\begin{verbatim}
[1] 2.718
\end{verbatim}

No caso do \emph{data frame} o qual foi criado acima (``informacoes''),
pode-se buscar as informações dos valores mínimos (função
\texttt{min()}), máximos (\texttt{max()}) da base:

\begin{Shaded}
\begin{Highlighting}[]
\KeywordTok{max}\NormalTok{(informacoes}\OperatorTok{$}\NormalTok{idade)}
\end{Highlighting}
\end{Shaded}

\begin{verbatim}
[1] 30
\end{verbatim}

\begin{Shaded}
\begin{Highlighting}[]
\KeywordTok{min}\NormalTok{(informacoes}\OperatorTok{$}\NormalTok{idade)}
\end{Highlighting}
\end{Shaded}

\begin{verbatim}
[1] 18
\end{verbatim}

Ainda, se o interesse está em descobrir a posição, no *data frame\}, do
peso mínimo e máximo da amostra utiliza-se o comando \texttt{which.min}
e \texttt{which.max}.

\begin{Shaded}
\begin{Highlighting}[]
\CommentTok{# Para descobrir em qual posição se encontra o peso mínimo:}
\KeywordTok{which.min}\NormalTok{(informacoes}\OperatorTok{$}\NormalTok{peso)}
\end{Highlighting}
\end{Shaded}

\begin{verbatim}
[1] 5
\end{verbatim}

\begin{Shaded}
\begin{Highlighting}[]
\KeywordTok{which.max}\NormalTok{(informacoes}\OperatorTok{$}\NormalTok{peso)}
\end{Highlighting}
\end{Shaded}

\begin{verbatim}
[1] 2
\end{verbatim}

Para descobrir qual é o estutande que possui o peso mínimo, por exemplo,
ou o Imc máximo, utiliza-se o seguinte comando (notem que os resultados
trazem a lista de todos os estudantes comparados):

\begin{Shaded}
\begin{Highlighting}[]
\NormalTok{informacoes}\OperatorTok{$}\NormalTok{estudantes[}\KeywordTok{which.min}\NormalTok{(informacoes}\OperatorTok{$}\NormalTok{peso)]}
\end{Highlighting}
\end{Shaded}

\begin{verbatim}
[1] Francisco
Levels: Camila Guilherme Marcelo Pedro Francisco
\end{verbatim}

\begin{Shaded}
\begin{Highlighting}[]
\NormalTok{informacoes}\OperatorTok{$}\NormalTok{estudantes[}\KeywordTok{which.max}\NormalTok{(informacoes}\OperatorTok{$}\NormalTok{Imc)]}
\end{Highlighting}
\end{Shaded}

\begin{verbatim}
[1] Guilherme
Levels: Camila Guilherme Marcelo Pedro Francisco
\end{verbatim}

O arredondamento de valores numéricos pode ser feito utilizando o
comando \texttt{round()}, o qual o pesquisador informa o número de casas
decimais:

\begin{Shaded}
\begin{Highlighting}[]
\CommentTok{# Arredondar para n casas decimais}
\KeywordTok{round}\NormalTok{(informacoes}\OperatorTok{$}\NormalTok{Imc, }\DecValTok{2}\NormalTok{)}
\end{Highlighting}
\end{Shaded}

\begin{verbatim}
[1] 20.06 44.44 22.16 26.09 23.34
\end{verbatim}

Já o comando \texttt{signif()} determina o número de algarismos
significativos da série escolhida, ou seja, ele arredonda para os
valores em seu primeiro argumento com os número de dígitos detemrinados:

\begin{Shaded}
\begin{Highlighting}[]
\NormalTok{x2 <-}\StringTok{ }\NormalTok{pi }\OperatorTok{*}\StringTok{ }\DecValTok{100}\OperatorTok{^}\NormalTok{(}\OperatorTok{-}\DecValTok{1}\OperatorTok{:}\DecValTok{3}\NormalTok{)}
\KeywordTok{round}\NormalTok{(x2, }\DecValTok{3}\NormalTok{)}
\end{Highlighting}
\end{Shaded}

\begin{verbatim}
[1] 3.100e-02 3.142e+00 3.142e+02 3.142e+04 3.142e+06
\end{verbatim}

\begin{Shaded}
\begin{Highlighting}[]
\KeywordTok{signif}\NormalTok{(x2, }\DecValTok{3}\NormalTok{) }
\end{Highlighting}
\end{Shaded}

\begin{verbatim}
[1] 3.14e-02 3.14e+00 3.14e+02 3.14e+04 3.14e+06
\end{verbatim}

A soma do total da coluna idade, o desvio padrão, a variância, a média
aritmética e mediana podem ser encontrados, respectivamente, pelos
comandos \texttt{sum()}, \texttt{sd()}, \texttt{var()}, \texttt{mean()},
\texttt{median()}:

\begin{Shaded}
\begin{Highlighting}[]
\CommentTok{# Realiza a somatória dos valores}
\KeywordTok{sum}\NormalTok{(informacoes}\OperatorTok{$}\NormalTok{idade)}
\end{Highlighting}
\end{Shaded}

\begin{verbatim}
[1] 107
\end{verbatim}

\begin{Shaded}
\begin{Highlighting}[]
\CommentTok{# Desvio padrão}
\KeywordTok{sd}\NormalTok{(informacoes}\OperatorTok{$}\NormalTok{idade)}
\end{Highlighting}
\end{Shaded}

\begin{verbatim}
[1] 4.93
\end{verbatim}

\begin{Shaded}
\begin{Highlighting}[]
\CommentTok{# Variancia}
\KeywordTok{var}\NormalTok{(informacoes}\OperatorTok{$}\NormalTok{idade)}
\end{Highlighting}
\end{Shaded}

\begin{verbatim}
[1] 24.3
\end{verbatim}

\begin{Shaded}
\begin{Highlighting}[]
\CommentTok{# Calcula a média aritmética dos valores}
\KeywordTok{mean}\NormalTok{(informacoes}\OperatorTok{$}\NormalTok{idade)}
\end{Highlighting}
\end{Shaded}

\begin{verbatim}
[1] 21.4
\end{verbatim}

\begin{Shaded}
\begin{Highlighting}[]
\CommentTok{# Informa o valor mediano do conjunto}
\KeywordTok{median}\NormalTok{(informacoes}\OperatorTok{$}\NormalTok{idade)}
\end{Highlighting}
\end{Shaded}

\begin{verbatim}
[1] 19
\end{verbatim}

O comando \texttt{quantile()} oferece a possibilidade de obter os
quartis dos dados de acordo com as probabilidades estabelecidas pelo
pesquisador. No exemplo, explora-se a variável idade:

\begin{Shaded}
\begin{Highlighting}[]
\KeywordTok{quantile}\NormalTok{(informacoes}\OperatorTok{$}\NormalTok{idade,  }\DataTypeTok{probs =} \KeywordTok{c}\NormalTok{(}\FloatTok{0.5}\NormalTok{, }\DecValTok{1}\NormalTok{, }\DecValTok{2}\NormalTok{, }\DecValTok{5}\NormalTok{, }\DecValTok{10}\NormalTok{, }\DecValTok{50}\NormalTok{)}\OperatorTok{/}\DecValTok{100}\NormalTok{)}
\end{Highlighting}
\end{Shaded}

\begin{verbatim}
 0.5%    1%    2%    5%   10%   50% 
18.02 18.04 18.08 18.20 18.40 19.00 
\end{verbatim}

\hypertarget{conversao-de-datas}{%
\section{Conversão de datas}\label{conversao-de-datas}}

A configuração e padronização dos formato de datas no RStudio podem ser
efetuadas pelo pesquisador, primeiramente ao carregar a base de dados no
programa e em um segundo momento durante a manipulação das informações.
Assim, seguem alguns dos procedimentos para a correta alteração dos
padrões de datas:

\begin{Shaded}
\begin{Highlighting}[]
\NormalTok{abertura <-}\StringTok{ }\KeywordTok{c}\NormalTok{(}\StringTok{"03/02/69"}\NormalTok{, }\StringTok{"17/08/67"}\NormalTok{)}
\NormalTok{fechamento <-}\StringTok{ }\KeywordTok{c}\NormalTok{(}\StringTok{"2000-20-01"}\NormalTok{, }\StringTok{"1999-14-08"}\NormalTok{)}
\NormalTok{abertura <-}\StringTok{ }\KeywordTok{as.Date}\NormalTok{(abertura, }\DataTypeTok{format =} \StringTok{"%d/%m/%y"}\NormalTok{)}
\NormalTok{fechamento <-}\StringTok{ }\KeywordTok{as.Date}\NormalTok{(fechamento, }\DataTypeTok{format =} \StringTok{"%Y-%d-%m"}\NormalTok{)}

\CommentTok{# Diferença de dias dos intervalos informados}
\NormalTok{abertura}\OperatorTok{-}\NormalTok{fechamento}
\end{Highlighting}
\end{Shaded}

\begin{verbatim}
Time differences in days
[1] -11308  24840
\end{verbatim}

\hypertarget{exercicios}{%
\section{Exercícios}\label{exercicios}}

\textbf{1.} Baixe o arquivo ``arvores'' que se encontra no endereço
\url{https://smolski.github.io/softwarelivrer/atividades}. Este é um
banco de dados com informações cedido pela professora Tatiane Chassot.
Abra o arquivo no Rstudio tomando os cuidados necessários (importar no
formato correto, prestar atenção nas vírgulas e nomes\ldots{}). Por meio
dos comandos do R, responda as seguintes perguntas, informando o comando
utilizado.

\textbf{1.1.} Qual é a espécie de árvore que possui o maior e menor
diâmetro? E quais são estes valores de diâmetro?

\textbf{1.2.} Qual é a altura média, mínima e média das árvores?

\textbf{1.3.} Encontre o diâmetro médio para cada espécie de árvores.

\textbf{1.4.} Com os comandos do R, verifique a quantidade de dados
referente as variáveis, bem como o nome referente a cada variável.

\textbf{1.5.} Renomeie a primeira coluna para ``espécie''.

\textbf{1.6.} Classifique as árvores quanto ao seu porte, em relação à
altura, em que:

Pequeno porte = árvores com altura inferior a 10 metros.

Grande porte = árvores com altura superior a 10 metros.

\textbf{2.} Baixe o arquivo ``bancodedados1'' que se encontra no
endereço \url{https://smolski.github.io/softwarelivrer/atividades}. Este
é um banco de dados com informações fictícias que usaremos a fim de
aprendizado. Abra o arquivo no Rstudio tomando os cuidados necessários.
Por meio dos comandos do R, responda as seguintes perguntas, informando
o comando utilizado.

\textbf{2.1.} Qual é o vendedor com mais sucesso de vendas? E o vendedor
com menor número de vendas?

\textbf{2.2.} Qual foi o número total de vendas?

\textbf{2.3.} Supondo que um vendedor tenha ficado de fora dos dados,
insira suas informações no banco de dados que já possuímos.

\begin{itemize}
\tightlist
\item
  Vendedor = Silvia; Idade = 48; Setor = 2; N de vendas = 45.
\end{itemize}

\textbf{2.4.} Crie uma nova coluna classificando os vendedores como:

\begin{itemize}
\item
  vendas \(<\) 25 = ``Regular''
\item
  25 \(>\) vendas = ``Ótimo''
\end{itemize}

\textbf{2.5} Renomeie a coluna ``vendas mensais'' para ``vendas
diárias''.

\hypertarget{desc}{%
\chapter{Estatística Descritiva}\label{desc}}

\emph{Denize Ivete Reis}

\begin{flushright}
\emph{}
\end{flushright}

A Estatística é uma ciência cujo campo de aplicação estende-se a
diferentes áreas do conhecimento humano. Tem por objetivo fornecer
métodos e técnicas que permitem lidar, racionalmente, com situações
sujeitas a incertezas. Apresenta um conjunto de técnicas e métodos de
pesquisa que envolvem o planejamento de estudos (experimentais e
observacionais), a coleta e organização de dados, a inferência, a
análise e a disseminação de informação.

Alguns termos extensamente utilizados em estatística, são definidos a
seguir (TRIOLA, \protect\hyperlink{ref-triola1999}{2011}):

\textbf{População}: é uma coleção completa de todos os elementos
(valores, pessoas, medidas etc.) a serem estudados.

\textbf{Censo}: é uma coleção de dados relativos a todos os elementos de
uma população.

\textbf{Amostra}: é uma sub-coleção de elementos extraídos de uma
população. Parâmetro é a medida numérica que descreve uma característica
de uma população.

\textbf{Estatística}: é uma medida numérica que descreve uma
característica de uma amostra.

\hypertarget{natureza-da-medida-das-variaveis}{%
\section{Natureza da medida das
variáveis}\label{natureza-da-medida-das-variaveis}}

Variáveis reporta-se a características ou atributos que podem tomar
diferentes valores ou categorias, o que se opõe ao conceito de constante
(ALMEIDA; FREIRE, \protect\hyperlink{ref-almeida2000}{2000}). Assim,
variável pode ser definida como sendo a característica dos elementos da
amostra ou da população que nos interessa estudar estatisticamente.

Variáveis podem ser classificadas da seguinte forma:

\textbf{Variáveis quantitativas}: consistem em números que representam
contagens ou medidas. Dividem-se em:

\begin{enumerate}
\def\labelenumi{\alph{enumi})}
\item
  Variáveis discretas: resultam em um conjunto finito de valores
  possíveis, ou de um conjunto enumerável desses valores. Ex. número de
  unidades produzidas.
\item
  Variáveis contínuas: resultam de um número infinito de valores
  possíveis que podem ser associados a pontos em uma escala contínua de
  tal maneira que não haja lacunas ou interrupções. Ex. Renda das
  famílias em reais.
\end{enumerate}

\textbf{Variáveis qualitativas}: ou variáveis categóricas, ou atributos
que podem ser separados em diferentes categorias que se distinguem por
alguma característica não-numérica. Divididas em:

\begin{enumerate}
\def\labelenumi{\alph{enumi})}
\item
  Variável nominal: caracterizada por dados que consistem apenas em
  nomes, rótulos ou categorias. Os dados não podem ser dispostos segundo
  um esquema ordenado (como de baixo para cima). Ex. nacionalidade
\item
  Variável ordinal: envolve variáveis representadas por nomes que podem
  ser dispostos em alguma ordem, mas as diferenças entre os valores dos
  dados não podem ser determinadas, ou não tem sentido. Esse nível dá
  informações sobre comparações relativas, mas os graus de diferença não
  servem para cálculos (TRIOLA,
  \protect\hyperlink{ref-triola1999}{2011}). Ex. Grau de escolaridade.
\end{enumerate}

\textbf{Dado}: é o valor assumido por uma variável aleatória em um
experimento.

A Estatística subdivide-se em descritiva e inferencial. A estatística
descritiva se preocupa em descrever os dados. A estatística inferencial,
fundamentada na teoria das probabilidades, se preocupa com a análise
destes dados e sua interpretação.

Informações estatísticas em jornais, relatórios e outras publicações que
consistem de dados reunidos e apresentados de forma clara e resumida, na
forma de tabelas, gráficos ou numéricos, são conhecidos como
estatísticas descritivas (ANDERSON, 2002).

\textbf{Exemplo 1}

Estaremos utilizando como exemplo os dados de uma pesquisa (dados
simulados), cujo banco de dados está intitulado ``Dados\_pesquisa.ods''.
Os dados são referentes aos resultados obtidos por ocasião de uma
pesquisa realizada entre os consumidores a fim de analisar
características associadas ao mercado consumidor de sucos, sendo que a
amostra é composta de 348 entrevistados aleatoriamente selecionados.

\begin{itemize}
\item
  O objetivo primário do estudo foi determinar variáveis que seriam
  úteis para caracterizar os consumidores que já conhecem o suco e a
  possibilidade potencial de futuros consumidores. Há também interesse
  nas relações entre variáveis das características pessoais desses
  consumidores ou futuros consumidores.
\item
  A pesquisa foi realizada, depois que os participantes realizaram uma
  visita técnica às instalações da empresa e puderam conhecer seus
  produtos e processos.
\end{itemize}

Para cada entrevistado foram registrados dados para as seguintes
variáveis:

\textbf{Sexo} -- Gênero sexual;

\textbf{Divulgacao} -- Forma de acesso ao suco ou publicidade do mesmo;

\textbf{Renda\_h} -- Renda por hora do entrevistado;

\textbf{Praticidade} -- Aspectos quanto a oferta do suco, como por ex.
embalagem;

\textbf{Sabor} -- Aspectos relacionados ao sabor;

\textbf{Pessoas\_familia} -- Número de pessoas que compõe o grupo
familiar;

\textbf{Preço} -- como cada entrevistado classificava o preço do
produto;

\textbf{consumo\_anterior} -- Se já consumia o suco antes da visita
técnica;

\textbf{consumo\_pos} -- Se consumia o suco após a visita técnica;

\textbf{Idade} -- Idade dos consumidores;

\textbf{Altura\_(m)} -- Altura dos consumidores;

\textbf{Peso\_(Kg)} -- Peso dos consumidores.

Pede-se:

\begin{enumerate}
\def\labelenumi{\arabic{enumi}.}
\item
  Salvar inicialmente os dados em formato CSV, xlsx ou outro.
\item
  Ler os dados no ``Environment'' pelo ``Import Dataset\ldots{}From
  CSV'' ou outro. No exemplo abaixo foram importados os dados
  diretamente do arquivo hospedado na internet.
\item
  Carregar o banco de dados, com a finalidade de usar os objetos
  (variáveis) diretamente nas funções a serem utilizadas.
\end{enumerate}

\texttt{attach(nome\_da\_planilha)}

\begin{Shaded}
\begin{Highlighting}[]
\KeywordTok{require}\NormalTok{(readxl)}
\end{Highlighting}
\end{Shaded}

\begin{verbatim}
Carregando pacotes exigidos: readxl
\end{verbatim}

\begin{Shaded}
\begin{Highlighting}[]
\NormalTok{url <-}\StringTok{ "https://goo.gl/37Fdzz"}
\NormalTok{destfile <-}\StringTok{ "pesquisa_dados.xlsx"}
\NormalTok{curl}\OperatorTok{::}\KeywordTok{curl_download}\NormalTok{(url, destfile)}
\NormalTok{pesquisa_dados <-}\StringTok{ }\KeywordTok{read_excel}\NormalTok{(destfile)}
\KeywordTok{attach}\NormalTok{(pesquisa_dados)}
\KeywordTok{ls.str}\NormalTok{(pesquisa_dados)}
\end{Highlighting}
\end{Shaded}

\begin{verbatim}
Altura_(m) :  num [1:348] 1.82 1.9 1.69 1.89 1.9 1.76 1.83 1.81 1.67 1.55 ...
Caso :  num [1:348] 1 2 3 4 5 6 7 8 9 10 ...
consumo_anterior :  chr [1:348] "N" "N" "S" "N" "S" "S" "S" "N" "N" "N" "N" "N" "S" "S" "S" ...
consumo_pos :  chr [1:348] "N" "S" "N" "S" "N" "S" "N" "S" "N" "S" "S" "S" "S" "S" "S" ...
Divulgacao :  chr [1:348] "Degustacao" "Radio" "TV" "TV" "Degustacao" "TV" "TV" "Radio" ...
Idade :  num [1:348] 22 21 20 18 16 28 19 19 22 19 ...
Peso_(Kg) :  num [1:348] 78.5 80 54 78 36 82 75 69 58 49 ...
Pessoas_familia :  num [1:348] 4 3 3 7 4 4 3 4 1 4 ...
Praticidade :  chr [1:348] "Pessima" "Otima" "Boa" "Pessima" "Ruim" "Boa" "Regular" ...
Preço :  chr [1:348] "Acima_concorrencia" "Abaixo_concorrencia" ...
Renda_h :  chr [1:348] "1.41" "17.34" "6.86" "2.65" "2.01" "11.32" "6.86" "3.25" ...
Sabor :  chr [1:348] "Otimo" "Pessimo" "Bom" "Otimo" "Otimo" "Regular" "Ruim" "Bom" ...
Sexo :  chr [1:348] "Feminino" "Feminino" "Feminino" "Feminino" "Masculino" ...
\end{verbatim}

\hypertarget{tabelas}{%
\section{Tabelas}\label{tabelas}}

Segundo BARBETTA (\protect\hyperlink{ref-barbetta1988}{2010}), dados
representados em tabelas e gráficos adequados, permitem observar
determinados aspectos relevantes, bem como delinear hipóteses a respeito
da estrutura dos dados em estudo, o que conhecemos como análise
exploratória de dados. Isto pode ser feito inicialmente com a
representação em forma de tabelas.

O comando \texttt{table()} é utilizado para elaborarmos tabelas de
frequências absolutas. Dependendo da variável a ser representada,
podemos usar esse comando de diferentes formas:

\hypertarget{tabela-simples-para-apresentacao-das-frequencias-absolutas}{%
\subsection{Tabela simples para apresentação das frequências
absolutas}\label{tabela-simples-para-apresentacao-das-frequencias-absolutas}}

Uma tabela simples considera quantas vezes ocorre cada categoria (ou
nível).

\texttt{table(nome\_variável)}

Ex. Variável \textbf{Praticidade}

\begin{Shaded}
\begin{Highlighting}[]
\KeywordTok{table}\NormalTok{(Praticidade)}
\end{Highlighting}
\end{Shaded}

\begin{verbatim}
Praticidade
    Boa   Otima Pessima Regular    Ruim 
     82      70      21      80      95 
\end{verbatim}

\hypertarget{tabela-cruzada}{%
\subsection{Tabela cruzada}\label{tabela-cruzada}}

A tabela cruzada, também conhecida como tabela de dupla entrada, para
apresentação das frequências absolutas.

\texttt{table(nome\_variável1,nome\_variável2)}

Ex. Construir uma tabela cruzada apresentando as frequências absolutas
das variáveis \textbf{Sexo} e \textbf{Divulgacao}.

\begin{Shaded}
\begin{Highlighting}[]
\KeywordTok{table}\NormalTok{(pesquisa_dados}\OperatorTok{$}\NormalTok{Sexo,pesquisa_dados}\OperatorTok{$}\NormalTok{Divulgacao)}
\end{Highlighting}
\end{Shaded}

\begin{verbatim}
           
            Degustacao Outro Radio  TV
  Feminino          78     6    61 147
  Masculino         19     1    15  21
\end{verbatim}

\hypertarget{tabela-cruzada-para-apresentacao-das-frequencias-relativas}{%
\subsection{Tabela cruzada para apresentação das frequências
relativas}\label{tabela-cruzada-para-apresentacao-das-frequencias-relativas}}

Com a introdução do comando \texttt{prop.table} é possível gerar,
facilmente, tabelas de frequências relativas para as variáveis de
interesse. As medidas relativas são importantes para comparar
distribuições de frequências (BARBETTA,
\protect\hyperlink{ref-barbetta1988}{2010}).

\texttt{prop.table(table(nome\_variável1,nome\_variável2))}

Ex. Construir uma tabela cruzada apresentando as frequências relativas
das variáveis \textbf{Sexo} e \textbf{Divulgacao}.

\begin{Shaded}
\begin{Highlighting}[]
\KeywordTok{prop.table}\NormalTok{(}\KeywordTok{table}\NormalTok{(Divulgacao,Sexo))}
\end{Highlighting}
\end{Shaded}

\begin{verbatim}
            Sexo
Divulgacao   Feminino Masculino
  Degustacao 0.224138  0.054598
  Outro      0.017241  0.002874
  Radio      0.175287  0.043103
  TV         0.422414  0.060345
\end{verbatim}

A função \texttt{tapply} serve para calcular um valor usando uma
variável categórica como condição, ou seja, aplica uma função qualquer
(como média, por exemplo) a uma variável quantitativa para cada classe
de uma variável categórica. Assim, permite obter em um só comando, a
medida para cada categoria.

\texttt{tapply(var\_quantitativa,var\_categórica,\ função\_desejada)}

\texttt{tapply(variavel\_quantitativa,variavel\_qualitativa,\ mean)}

Se um registro possui \texttt{NA}, isto é, dados perdidos: com o
parâmetro na.rm=T, indicamos para o comando ignorar os NAs nos dados e
calcular a média.

\texttt{tapply(variavel\_quanti,\ variavel\_quali,\ mean,\ na.rm=T)}

\hypertarget{graficos}{%
\section{Gráficos}\label{graficos}}

\hypertarget{grafico-de-colunas}{%
\subsection{Gráfico de colunas}\label{grafico-de-colunas}}

As frequências podem ser visualizadas graficamente, usando gráficos de
barras elementares, que se aplicam à descrição de qualquer variável
qualitativa ou quantitativa discreta, vetor de dados ou tabelas.

No entanto, no caso de dados em banco de dados, quando não utilizamos
outros mecanismos de atribuição, precisamos usar o comando table.

\texttt{barplot(table(nome\_variável))}

Ex. Construir um gráfico de colunas para a variável \textbf{Sexo}.

\begin{Shaded}
\begin{Highlighting}[]
\KeywordTok{barplot}\NormalTok{(}\KeywordTok{table}\NormalTok{(Sexo))}
\end{Highlighting}
\end{Shaded}

\begin{figure}[H]

{\centering \includegraphics[width=0.8\linewidth]{index_files/figure-latex/unnamed-chunk-67-1} 

}

\caption{Gráfico de colunas com a variável Sexo}\label{fig:unnamed-chunk-67}
\end{figure}

\textbf{Obs}.: É possível personalizar o gráfico, incluindo o título do
eixo x (xlab), o título do eixoy (ylab), o título do gráfico (main), a
cor da coluna (col) e cor da borda da coluna (border), lembrando que as
cores, assim como os comandos devem ser expressas em inglês.

\texttt{barplot(table(nome\_variável),\ col=c("blue","red"),\ main="Título",\ xlab="Variável\ do\ eixo\ x",\ ylab\ =\ "Informação\ que\ consta\ no\ eixo\ y",border="red")}

\textbf{Ex.1)} Construir um gráfico de colunas para a variável
\textbf{Pessoas\_familia}.

\begin{Shaded}
\begin{Highlighting}[]
\KeywordTok{barplot}\NormalTok{(}\KeywordTok{table}\NormalTok{(}\StringTok{`}\DataTypeTok{Pessoas_familia}\StringTok{`}\NormalTok{), }\DataTypeTok{col=}\KeywordTok{c}\NormalTok{(}\StringTok{"blue"}\NormalTok{), }
        \DataTypeTok{main =} \StringTok{"Frequência de pessoas por família"}\NormalTok{, }
        \DataTypeTok{xlab =} \StringTok{"Frequência", }
\StringTok{        ylab = "}\NormalTok{Pessoas}\StringTok{", }
\StringTok{        border = "}\NormalTok{red}\StringTok{")}
\end{Highlighting}
\end{Shaded}

\begin{figure}[H]

{\centering \includegraphics[width=0.8\linewidth]{index_files/figure-latex/unnamed-chunk-68-1} 

}

\caption{Gráfico de colunas com a variável `Pessoas familia`}\label{fig:unnamed-chunk-68}
\end{figure}

\textbf{Ex.2)} Construir uma tabela de dupla entrada para as variáveis
\textbf{Sexo} e \textbf{Divulgação}.

\begin{Shaded}
\begin{Highlighting}[]
\KeywordTok{barplot}\NormalTok{(}\KeywordTok{table}\NormalTok{(Sexo,Divulgacao), }
        \DataTypeTok{col=}\KeywordTok{c}\NormalTok{(}\StringTok{"blue"}\NormalTok{), }
        \DataTypeTok{main =} \StringTok{"Frequência de pessoas por Sexo e Divulgacao"}\NormalTok{)}
\end{Highlighting}
\end{Shaded}

\begin{figure}[H]

{\centering \includegraphics[width=0.8\linewidth]{index_files/figure-latex/unnamed-chunk-69-1} 

}

\caption{Gráfico de colunas com as variáveis Sexo e Divulgacao}\label{fig:unnamed-chunk-69}
\end{figure}

\textbf{Ex.3)} Na sequência utiliza o sinal de atribuição \textless{}-
para atribuir o nome Resultado para esta tabela (tabela de dupla entrada
obtida em Ex.2).

\begin{Shaded}
\begin{Highlighting}[]
\NormalTok{Resultado<-}\KeywordTok{table}\NormalTok{(Sexo,Divulgacao)}
\end{Highlighting}
\end{Shaded}

\textbf{Ex.4)} Execute o seguinte comando:

\begin{Shaded}
\begin{Highlighting}[]
\KeywordTok{barplot}\NormalTok{(Resultado,}\DataTypeTok{col=}\KeywordTok{c}\NormalTok{(}\StringTok{"blue"}\NormalTok{,}\StringTok{"red"}\NormalTok{),}\DataTypeTok{main=}\StringTok{"Título"}\NormalTok{,}
        \DataTypeTok{xlab=}\StringTok{"Variável do eixo x"}\NormalTok{,}
        \DataTypeTok{ylab=}\StringTok{"Informação que consta no eixo y"}\NormalTok{, }
        \DataTypeTok{border=}\StringTok{'red'}\NormalTok{, }
        \DataTypeTok{beside=}\NormalTok{T,}\DataTypeTok{legend=}\KeywordTok{rownames}\NormalTok{(Resultado),}
        \DataTypeTok{args.legend =} \KeywordTok{list}\NormalTok{(}\DataTypeTok{x =} \StringTok{"topleft"}\NormalTok{))}
\end{Highlighting}
\end{Shaded}

\begin{figure}[H]

{\centering \includegraphics[width=0.8\linewidth]{index_files/figure-latex/unnamed-chunk-71-1} 

}

\caption{Gráfico de colunas com as variáveis Sexo e Divulgacao (2)}\label{fig:unnamed-chunk-71}
\end{figure}

Observe que o uso do argumento \texttt{beside=T} evita que as barras
fiquem empilhadas e o arguemnto \texttt{legend}' insere a legenda
conforme as cores das colunas.

\textbf{Ex.5)} Repita o exercício a partir do Ex.3, invertendo a ordem
entre as variáveis qualitativas.

\hypertarget{setograma-ou-grafico-de-pizza}{%
\subsection{Setograma ou gráfico de
pizza}\label{setograma-ou-grafico-de-pizza}}

Os gráficos em setores são utilizados para ilustrar dados qualitativos
de modo mais compreensível. Quando a variável é ordinal, gráficos de
colunas são mais indicados pelo fato de permitirem manter a ordem das
categorias. Isto também vale para os casos em que se tem muitas
categorias ou quanto se pretende dar mais destaque às categorias mais
frequentes (BARBETTA, \protect\hyperlink{ref-barbetta1988}{2010}).

\texttt{pie(table(nome\_variável),main="nome")}

Ex. Construa um gráfico na forma de Setograma para a variável
\textbf{Sabor}.

\begin{Shaded}
\begin{Highlighting}[]
\CommentTok{# Criar objeto com a tabela de Sabor}
\NormalTok{Sabor1=}\KeywordTok{table}\NormalTok{(Sabor)}

\CommentTok{# Calcular o percentual}
\NormalTok{percent=}\KeywordTok{signif}\NormalTok{(Sabor1}\OperatorTok{/}\KeywordTok{sum}\NormalTok{(Sabor1)}\OperatorTok{*}\DecValTok{100}\NormalTok{,}\DecValTok{3}\NormalTok{)}

\CommentTok{#Criando os nomes da legenda}
\NormalTok{nomesleg=}\KeywordTok{c}\NormalTok{(}\StringTok{"Bom"}\NormalTok{,}\StringTok{"Ótimo"}\NormalTok{,}\StringTok{"Péssimo"}\NormalTok{,}\StringTok{"Regular"}\NormalTok{,}\StringTok{"Ruim"}\NormalTok{)}

\CommentTok{#Plota-se o gráfico de pizza}
\KeywordTok{pie}\NormalTok{(Sabor1, }
    \DataTypeTok{labels =} \KeywordTok{paste}\NormalTok{(percent, }\StringTok{"%"}\NormalTok{, }\DataTypeTok{sep=}\StringTok{""}\NormalTok{), }
    \DataTypeTok{col =} \KeywordTok{terrain.colors}\NormalTok{(}\DecValTok{5}\NormalTok{), }\CommentTok{# Determina cores }
    \DataTypeTok{radius =} \DecValTok{1}\NormalTok{) }
\KeywordTok{legend}\NormalTok{(}\DataTypeTok{x=}\StringTok{"topright"}\NormalTok{, }\CommentTok{# Determina posição da legenda}
       \DataTypeTok{legend=}\NormalTok{nomesleg, }\CommentTok{# Insere nomes da legenda}
       \DataTypeTok{cex =} \FloatTok{0.65}\NormalTok{, }\CommentTok{# Tamanho do texto}
       \DataTypeTok{fill =} \KeywordTok{terrain.colors}\NormalTok{(}\DecValTok{5}\NormalTok{)) }\CommentTok{# Determina cores }

\NormalTok{## Alguns exemplos de paletas de cores:}
\CommentTok{# - rainbow(n)}
\CommentTok{# - heat.colors(n)}
\CommentTok{# - terrain.colors(n) }
\CommentTok{# - topo.colors(n)}
\CommentTok{# - cm.colors(n)}
\end{Highlighting}
\end{Shaded}

\begin{figure}[H]

{\centering \includegraphics[width=0.8\linewidth]{index_files/figure-latex/unnamed-chunk-72-1} 

}

\caption{Gráfico de pizza com a variável Sabor}\label{fig:unnamed-chunk-72}
\end{figure}

\hypertarget{histograma}{%
\subsection{Histograma}\label{histograma}}

No histograma, utilizado em geral quando temos variáveis quantitativas
contínuas, a altura dos retângulos representa a frequência de ocorrência
de valores no intervalo (deve iniciar sempre em zero), devem ter sempre
a mesma largura podendo ser justapostos. O eixo horizontal (dos valores
da variável) pode iniciar próximo ao menor valor da variável (BARBETTA,
\protect\hyperlink{ref-barbetta1988}{2010}). Para confecção do
histograma devemos usar:

\texttt{hist(nome\_variável)}

Ex. Construa um histograma com a variável \textbf{Renda\_h}.

\begin{Shaded}
\begin{Highlighting}[]
\KeywordTok{hist}\NormalTok{(}\KeywordTok{as.numeric}\NormalTok{(}\StringTok{`}\DataTypeTok{Renda_h}\StringTok{`}\NormalTok{))}
\end{Highlighting}
\end{Shaded}

\begin{figure}[H]

{\centering \includegraphics[width=0.8\linewidth]{index_files/figure-latex/unnamed-chunk-73-1} 

}

\caption{Histograma com a variável `Renda h`}\label{fig:unnamed-chunk-73}
\end{figure}

\textbf{Obs}. I: Neste caso também é possível personalizar o gráfico,
incluindo o título do eixo x (xlab), o título do eixoy (ylab), o título
do gráfico (main), a cor da coluna (col) e cor da borda da coluna
(border), lembrando que as cores, assim como os comandos devem ser
expressas em inglês.

\textbf{Obs}. II: Para definir o número de intervalos no Histograma,
usamos:

\texttt{hist(nome\_variável,\ breaks\ =\ 5)}

\begin{Shaded}
\begin{Highlighting}[]
\KeywordTok{hist}\NormalTok{(}\KeywordTok{as.numeric}\NormalTok{(}\StringTok{`}\DataTypeTok{Renda_h}\StringTok{`}\NormalTok{), }\DataTypeTok{breaks=}\DecValTok{5}\NormalTok{)}
\end{Highlighting}
\end{Shaded}

\begin{figure}[H]

{\centering \includegraphics[width=0.8\linewidth]{index_files/figure-latex/unnamed-chunk-74-1} 

}

\caption{Histograma com a variável Renda h com breaks=5}\label{fig:unnamed-chunk-74}
\end{figure}

Use o argumento \texttt{main=NULL} para remover o título.

\hypertarget{boxplot-ou-diagrama-em-caixas}{%
\subsection{Boxplot ou diagrama em
caixas}\label{boxplot-ou-diagrama-em-caixas}}

Os diagramas em caixa são convenientes para revelar tendências centrais,
dispersão, distribuição dos dados e a presença de outliers (valores
extremos). Como as medianas revelam uma tendência central, ao passo que
os quartis indicam a dispersão dos dados, os diagramas em caixa têm a
vantagem de não serem tão sensíveis a valores extremos como outras
medidas baseadas na média e no desvio-padrão. Por outro lado, os
diagramas em caixa (boxplots) não dão informação tão detalhada quanto os
histogramas ou os gráficos ramo-e-folhas, podendo não ser, assim, a
melhor escolha quando lidamos com um único conjunto de dados. Os
diagramas em caixa são, entretanto, mais convenientes na comparação de
dois ou mais conjuntos de dados (TRIOLA,
\protect\hyperlink{ref-triola1999}{2011}).

No diagrama de caixas, torna-se fácil identificar \textbf{outliers} (ou
valores extremos), que são valores extremamente raros, no sentido de que
estão muito afastados da maioria dos dados. Ao explorarmos um conjunto
de dados, não podem deixar de considerar os outliers, porque eles podem
revelar informações importantes (TRIOLA,
\protect\hyperlink{ref-triola1999}{2011}).

Para obter o boxplot para um conjunto de dados:

\texttt{boxplot(variávelA,\ variávelB,\ names=c("A","B"))}

\textbf{Ex.1)} Construir um boxplot da variável \textbf{Idade}.

\begin{Shaded}
\begin{Highlighting}[]
\KeywordTok{boxplot}\NormalTok{(Idade,}\DataTypeTok{horizontal =}\NormalTok{ T)}
\end{Highlighting}
\end{Shaded}

\begin{figure}[H]

{\centering \includegraphics[width=0.8\linewidth]{index_files/figure-latex/unnamed-chunk-75-1} 

}

\caption{Boxplot com a variável Idade}\label{fig:unnamed-chunk-75}
\end{figure}

\textbf{Ex.2)} Construir um boxplot das variáveis \textbf{Peso\_(Kg)} e
\textbf{Altura\_(m)}.

\hypertarget{grafico-ramo-e-folhas}{%
\subsection{Gráfico ramo-e-folhas}\label{grafico-ramo-e-folhas}}

Em um gráfico ramo-e-folhas, classificamos os dados segundo um padrão
que revela a distribuição subjacente. O padrão consiste em separar um
número em duas partes em geral: o ramo consiste nos algarismos mais à
esquerda e as folhas consistem nos algarismos mais à direita.

No gráfico Ramo-e-folhas, podemos ver a distribuição desses dados, que é
uma vantagem do gráfico ramo-e-folhas e ainda conservar toda a
informação da lista original; se necessário, podemos recompor a relação
original de valores. Note que as linhas de algarismos em um gráfico
ramo-e-folhas são análogas, em natureza, às barras de um histograma
(TRIOLA, \protect\hyperlink{ref-triola1999}{2011}).

\texttt{stem(nome\_variável)} - comando que permite obter um gráfico
Ramo e Folhas.

Ou

\texttt{stem(nome\_variável,scale=1)}

O ``scale=1'', que é o padrão, separa os ramos das folhas a partir das
casas decimais.

Caso padrão:

\begin{itemize}
\tightlist
\item
  A ideia do ramo e folhas é separar um número (como 16,0) em duas
  partes. Assim, a primeira parte inteira (16) chamada de ramo e a
  segunda, a parte decimal (0) chamada de folha. O padrão do R é separar
  os números em duas partes (inteira e decimal) e agrupar os números em
  classes de tamanho 2. Por exemplo, o ramo 16 leva em conta os números
  16 e 17.
\end{itemize}

\textbf{Obs.}: Esse padrão vai se alterando, à medida que o conjunto de
dados apresente diferentes casas decimais.

Assim, outras opções podem ser avaliadas:

\begin{enumerate}
\def\labelenumi{\alph{enumi})}
\item
  \texttt{stem(nome\_variável,scale=0.5)}
\item
  \texttt{stem(nome\_variável,scale=2)}
\end{enumerate}

\textbf{Obs.}: Quando uma folha relacionada com certo ramo tem uma
quantidade tão grande de valores que ele sintetiza essa quantidade
usando a denominação +n, e invade a linha seguinte. Isso pode ser
melhorado usando \textbf{width}.

\begin{enumerate}
\def\labelenumi{\alph{enumi})}
\setcounter{enumi}{2}
\tightlist
\item
  \texttt{stem(nome\_variável,scale=0.5,width=120)}
\end{enumerate}

Ex. Construa um gráfico Ramo e Follhas com a variável \textbf{Idade}.

\begin{Shaded}
\begin{Highlighting}[]
\KeywordTok{stem}\NormalTok{(Idade,}\DataTypeTok{scale=}\DecValTok{2}\NormalTok{)}
\end{Highlighting}
\end{Shaded}

\begin{verbatim}

  The decimal point is at the |

  16 | 000
  17 | 000000000
  18 | 0000000000000000000000000000000000000000
  19 | 000000000000000000000000000000000000000000000000
  20 | 0000000000000000000000000000000000000000000000000000000000000000
  21 | 000000000000000000000000000000000000000000000000000000000000
  22 | 0000000000000000000000000000000000000000000
  23 | 000000000000
  24 | 000000000
  25 | 0000
  26 | 00000000000
  27 | 00000000000
  28 | 0000000000000
  29 | 00
  30 | 00000
  31 | 
  32 | 00
  33 | 
  34 | 00
  35 | 00
  36 | 
  37 | 
  38 | 000
  39 | 
  40 | 
  41 | 
  42 | 
  43 | 
  44 | 
  45 | 
  46 | 
  47 | 
  48 | 00000
\end{verbatim}

\hypertarget{graficos-de-dispersao}{%
\subsection{Gráficos de dispersão}\label{graficos-de-dispersao}}

Às vezes temos dados emparelhados de forma que associa cada valor de um
conjunto a um determinado valor de um segundo conjunto. Um diagrama de
dispersão é um gráfico dos dados emparelhados (x, y), com um eixo x
horizontal e um eixo y vertical. O diagrama de dispersão, apresenta no
eixo horizontal os valores da primeira variável e um eixo vertical para
os valores da segunda variável. O padrão dos pontos assim marcados
costuma ajudar a determinar se existe algum relacionamento entre as duas
variáveis A e B.

\texttt{plot(variável\_independente,Variável\_dependente)}

Ou

\texttt{plot(variável\_dependente\textasciitilde{}variável\_independente)}

\hypertarget{grafico-de-linhas}{%
\subsection{Gráfico de linhas}\label{grafico-de-linhas}}

Apresenta a evolução de um dado, geralmente ao longo do tempo. Eixos na
vertical e na horizontal indicam as informações a que se refere e a
linha traçada entre eles, ascendente, descendente constante ou com
vários altos e baixos mostra o percurso de um fenômeno específico.

Ex. Considere os dados que descrevem os valores do número de empresas
fiscalizadas na fiscalização do trabalho na área rural Brasil 1998-2010.

\begin{longtable}[]{@{}cc@{}}
\caption{\label{tab:evolres}Evolução dos resultados da fiscalização do
trabalho na área rural Brasil 1998-2010}\tabularnewline
\toprule
\textbf{Ano} & \textbf{Empresas Fiscalizadas}\tabularnewline
\midrule
\endfirsthead
\toprule
\textbf{Ano} & \textbf{Empresas Fiscalizadas}\tabularnewline
\midrule
\endhead
1998 & 7.042\tabularnewline
1999 & 6.561\tabularnewline
2000 & 8.585\tabularnewline
2001 & 9.641\tabularnewline
2002 & 8.873\tabularnewline
2003 & 9.367\tabularnewline
2004 & 13.856\tabularnewline
2005 & 12.192\tabularnewline
2006 & 13.326\tabularnewline
2007 & 13.390\tabularnewline
2008 & 10.839\tabularnewline
2009 & 13.379\tabularnewline
2010 & 11.978\tabularnewline
\bottomrule
\end{longtable}

Fonte: MTE. SFIT. Elaboração: DIEESE.

Para construir um gráfico de linhas, utilizamos o seguinte comando:

\texttt{plot(x,y,type=\ "Tipo\ de\ símbolo")}

Neste gráfico, podemos utilizar comandos já utilizados anteriormente,
para inserir título, nomes dos eixos, etc. Para escolher o formato das
linhas, com o uso do argumento \texttt{type}, seguem algumas opções:

\begin{itemize}
\tightlist
\item
  \texttt{"p"} para pontos,
\item
  \texttt{"l"} para linhas,
\item
  \texttt{"b"} para pontos e linhas,
\item
  \texttt{"c"} para linhas descontínuas nos pontos,
\item
  \texttt{"o"} para pontos sobre as linhas,
\item
  \texttt{"n"} para nenhum gráfico, apenas a janela.
\end{itemize}

Para o caso de representação no mesmo gráfico, de duas ou mais
variáveis, o processo deverá ser realizado por etapas:

\texttt{plot(x,y1,type="b",main="Título",\ xlab="Nome\_eixo\_x",ylab="Nome\_eixo\_y",\ col="cor\ das\ linhas",ylim=c(yi,ys))}

\begin{Shaded}
\begin{Highlighting}[]
\NormalTok{empfisc=}\KeywordTok{data.frame}\NormalTok{(}\DataTypeTok{ano=}\KeywordTok{c}\NormalTok{(}\DecValTok{1998}\NormalTok{,}\DecValTok{1999}\NormalTok{,}\DecValTok{2000}\NormalTok{,}\DecValTok{2001}\NormalTok{,}\DecValTok{2002}\NormalTok{,}\DecValTok{2003}\NormalTok{,}\DecValTok{2004}\NormalTok{,}\DecValTok{2005}\NormalTok{,}\DecValTok{2006}\NormalTok{,}\DecValTok{2007}\NormalTok{,}
    \DecValTok{2008}\NormalTok{,}\DecValTok{2009}\NormalTok{,}\DecValTok{2010}\NormalTok{), }\DataTypeTok{qtd=}\KeywordTok{c}\NormalTok{(}\DecValTok{7042}\NormalTok{,}\DecValTok{6561}\NormalTok{,}\DecValTok{8585}\NormalTok{,}\DecValTok{9641}\NormalTok{,}\DecValTok{8873}\NormalTok{,}\DecValTok{9367}\NormalTok{,}
\DecValTok{13856}\NormalTok{,}\DecValTok{12192}\NormalTok{,}\DecValTok{13326}\NormalTok{,}\DecValTok{13390}\NormalTok{,}\DecValTok{10839}\NormalTok{,}\DecValTok{13379}\NormalTok{,}\DecValTok{11978}\NormalTok{))}

\KeywordTok{plot}\NormalTok{(empfisc}\OperatorTok{$}\NormalTok{ano,empfisc}\OperatorTok{$}\NormalTok{qtd,}\DataTypeTok{type=}\StringTok{"b"}\NormalTok{,}\DataTypeTok{main=}\StringTok{"Título"}\NormalTok{,}
     \DataTypeTok{xlab=}\StringTok{"Nome_eixo_x"}\NormalTok{,}\DataTypeTok{ylab=}\StringTok{"Nome_eixo_y"}\NormalTok{, }
     \DataTypeTok{col=}\StringTok{"blue"}\NormalTok{,}\DataTypeTok{xlim=}\KeywordTok{c}\NormalTok{(}\DecValTok{1998}\NormalTok{,}\DecValTok{2010}\NormalTok{))}
\end{Highlighting}
\end{Shaded}

\begin{figure}[H]

{\centering \includegraphics[width=0.8\linewidth]{index_files/figure-latex/unnamed-chunk-78-1} 

}

\caption{Gráfico de linha sobre a fiscalização do trabalho na área rural Brasil 1998-2010}\label{fig:unnamed-chunk-78}
\end{figure}

onde, no argumento \texttt{ylim}, devemos indicar o intervalo de
variação dos valores de y, ou seja todo o intervalo que será necessário
para representar todas as variáveis.

Na sequência adicionamos as instruções para as demais variáveis:

\texttt{lines(x,\ y2,col="cor\_desejada",\ type="b")}

Com o argumento \texttt{"legend"} instruímos a formatação da legenda:

\texttt{legend(xp,yp,c("representação\_variável\_1\ na\ legenda",\ "representação\_variável\_2\ na\ legenda"),}col
=c(``Cor1'',``cor2''),pch=Valor entre 0 e 25)`

Obs.: \texttt{pch}= número (entre 0 e 25). No Help do R (buscando com
pch), você encontra a lista completa de símbolos que podem ser
utilizados na representação da legenda. Neste caso, pode ser importante
também alterar o tamanho da fonte da legenda, com o uso do argumento
\texttt{"cex"}.

Exemplo: Segue exemplo de um gráfico de linhas para as temperaturas
registradas durante o dia 11/04/2018, pela Estação Meteorológica de São
Luiz Gonzaga, RS, conforme dados obtidos no site do Inmet.

\begin{Shaded}
\begin{Highlighting}[]
\KeywordTok{library}\NormalTok{(readr)}
\NormalTok{inmet <-}\StringTok{ }\KeywordTok{read_delim}\NormalTok{(}\StringTok{"https://goo.gl/se71v2"}\NormalTok{, }
    \StringTok{";"}\NormalTok{, }\DataTypeTok{escape_double =} \OtherTok{FALSE}\NormalTok{, }
    \DataTypeTok{col_types =} \KeywordTok{cols}\NormalTok{(}\DataTypeTok{data =} \KeywordTok{col_date}\NormalTok{(}\DataTypeTok{format =} \StringTok{"%m/%d/%Y"}\NormalTok{)), }
    \DataTypeTok{trim_ws =} \OtherTok{TRUE}\NormalTok{)}
\KeywordTok{head}\NormalTok{(inmet)}
\end{Highlighting}
\end{Shaded}

\begin{verbatim}
# A tibble: 6 x 6
  codigo_estacao data        hora temp_inst temp_max temp_min
  <chr>          <date>     <int>     <dbl>    <dbl>    <dbl>
1 A852           2018-04-11     0      26.2     27.1     26.2
2 A852           2018-04-11     1      26       26.2     26  
3 A852           2018-04-11     2      25.5     26.1     25.5
4 A852           2018-04-11     3      25.1     25.5     25  
5 A852           2018-04-11     4      24.6     25.2     24.5
6 A852           2018-04-11     5      24.3     24.7     24.2
\end{verbatim}

Segue a sequência de comandos, para obtenção do gráfico de linhas:

\begin{Shaded}
\begin{Highlighting}[]
\KeywordTok{plot}\NormalTok{(inmet}\OperatorTok{$}\NormalTok{hora,inmet}\OperatorTok{$}\NormalTok{temp_inst,}\DataTypeTok{type =} \StringTok{"b"}\NormalTok{, }
  \DataTypeTok{main =} \StringTok{"Temperaturas registradas na estação metereológica}
\StringTok{  de São Luis Gonzaga, 11 de abril de 2018"}\NormalTok{,}
  \DataTypeTok{xlab =} \StringTok{"hora"}\NormalTok{,}\DataTypeTok{ylab =} \StringTok{"temperaturas"}\NormalTok{,}\DataTypeTok{col=}\StringTok{"blue"}\NormalTok{,}
  \DataTypeTok{ylim =} \KeywordTok{c}\NormalTok{(}\DecValTok{20}\NormalTok{,}\DecValTok{40}\NormalTok{))}

\KeywordTok{lines}\NormalTok{(inmet}\OperatorTok{$}\NormalTok{hora,inmet}\OperatorTok{$}\NormalTok{temp_max,}\DataTypeTok{col=}\StringTok{"red"}\NormalTok{,}\DataTypeTok{type =} \StringTok{"b"}\NormalTok{)}

\KeywordTok{lines}\NormalTok{(inmet}\OperatorTok{$}\NormalTok{hora,inmet}\OperatorTok{$}\NormalTok{temp_min,}\DataTypeTok{col=}\StringTok{"green"}\NormalTok{,}\DataTypeTok{type =} \StringTok{"b"}\NormalTok{)}

\KeywordTok{legend}\NormalTok{(}\DecValTok{0}\NormalTok{,}\DecValTok{40}\NormalTok{,}\KeywordTok{c}\NormalTok{(}\StringTok{"temp_inst"}\NormalTok{,}\StringTok{"temp_max"}\NormalTok{,}\StringTok{"temp_min"}\NormalTok{),}
  \DataTypeTok{col =}\KeywordTok{c}\NormalTok{(}\StringTok{"blue"}\NormalTok{,}\StringTok{"red"}\NormalTok{,}\StringTok{"green"}\NormalTok{),}\DataTypeTok{pch=}\FloatTok{4.1}\NormalTok{,}\DataTypeTok{cex =} \FloatTok{0.75}\NormalTok{)}
\end{Highlighting}
\end{Shaded}

\begin{figure}[H]

{\centering \includegraphics[width=0.8\linewidth]{index_files/figure-latex/unnamed-chunk-80-1} 

}

\caption{Gráfico de linha sobre as temperaturas registradas em São Luiz Gonzaga - RS}\label{fig:unnamed-chunk-80}
\end{figure}

\hypertarget{estatisticas-descritivas}{%
\section{Estatísticas Descritivas}\label{estatisticas-descritivas}}

Para determinar o valor máximo de um conjunto de dados, utilizamos:

\texttt{max(nome\_da\_variável)}

Use a variável \textbf{Renda\_h}

\begin{Shaded}
\begin{Highlighting}[]
\CommentTok{#Transforme a variável Renda_h em variável numérica}
\NormalTok{pesquisa_dados}\OperatorTok{$}\NormalTok{Renda_h=}\KeywordTok{as.numeric}\NormalTok{(pesquisa_dados}\OperatorTok{$}\NormalTok{Renda_h)}
\CommentTok{#É preciso repetir o comando attach()}
\KeywordTok{attach}\NormalTok{(pesquisa_dados)}
\KeywordTok{max}\NormalTok{(Renda_h)}
\end{Highlighting}
\end{Shaded}

\begin{verbatim}
[1] 21.83
\end{verbatim}

De forma análoga, para determinar o valor mínimo de um conjunto de
dados, utilizamos:

\texttt{min(nome\_da\_variável)}

Use a variável \textbf{Renda\_h}

\begin{Shaded}
\begin{Highlighting}[]
\KeywordTok{min}\NormalTok{(Renda_h)}
\end{Highlighting}
\end{Shaded}

\begin{verbatim}
[1] 1.02
\end{verbatim}

\textbf{Obs.}: Para determinar a amplitude total de um conjunto de
dados, utilizamos:

\texttt{max(nome\_da\_variável)-min(nome\_da\_variável)}

Use a variável \textbf{Renda\_h}

\begin{Shaded}
\begin{Highlighting}[]
\KeywordTok{max}\NormalTok{(Renda_h)}\OperatorTok{-}\KeywordTok{min}\NormalTok{(Renda_h)}
\end{Highlighting}
\end{Shaded}

\begin{verbatim}
[1] 20.81
\end{verbatim}

Para obter as medidas da estatística descritiva, no caso medidas de
tendência central (mínimo, quartil 1, mediana, média, quartil 3,
máximo):

\texttt{summary(nome\_da\_variável)}

Ex. Use a variável \textbf{Renda\_h}

\begin{Shaded}
\begin{Highlighting}[]
\KeywordTok{summary}\NormalTok{(Renda_h)}
\end{Highlighting}
\end{Shaded}

\begin{verbatim}
   Min. 1st Qu.  Median    Mean 3rd Qu.    Max. 
   1.02    4.64    6.79    7.31    9.51   21.83 
\end{verbatim}

A moda é o valor que tem o maior número de ocorrências em um conjunto de
dados.

O R não tem um padrão de função embutida para calcular a moda. Uma
sugestão é a criação de uma função pelo usuário, que pode ser obtida,
por exemplo por:

\texttt{subset(table(variável),\ table(variável)==max(table(variável)))}

Ex. Use a variável \textbf{Praticidade}

\begin{Shaded}
\begin{Highlighting}[]
\KeywordTok{subset}\NormalTok{(}\KeywordTok{table}\NormalTok{(Praticidade), }
       \KeywordTok{table}\NormalTok{(Praticidade)}\OperatorTok{==}\KeywordTok{max}\NormalTok{(}\KeywordTok{table}\NormalTok{(Praticidade)))}
\end{Highlighting}
\end{Shaded}

\begin{verbatim}
Ruim 
  95 
\end{verbatim}

Ex. Use a variável quantitativa \textbf{Pessoas\_familia}

\begin{Shaded}
\begin{Highlighting}[]
\KeywordTok{table}\NormalTok{(Pessoas_familia)}
\end{Highlighting}
\end{Shaded}

\begin{verbatim}
Pessoas_familia
 0  1  2  3  4  5  6  7  8  9 10 
 2 14 25 62 73 60 64 30 15  2  1 
\end{verbatim}

\textbf{Obs.}: O primeiro valor encontrado, refere-se ao valor da moda
ao passo que o segundo valor representa quantas vezes esse valor foi
verificado.

Comando que permite determinar o percentil, no caso o percentil 10:

\texttt{quantile(nome\_variável,0.1)}

\textbf{Obs.}: Experimente usar o comando:

\texttt{quantile(nome\_variável)}

\textbf{Obs.}: Para a obtenção de quartis e decis, basta realizar a
conversão para o respectivo percentil e assim calcular normalmente.

\begin{Shaded}
\begin{Highlighting}[]
\KeywordTok{quantile}\NormalTok{(Renda_h)}
\end{Highlighting}
\end{Shaded}

\begin{verbatim}
    0%    25%    50%    75%   100% 
 1.020  4.638  6.785  9.512 21.830 
\end{verbatim}

\begin{Shaded}
\begin{Highlighting}[]
\KeywordTok{quantile}\NormalTok{(Renda_h,}\FloatTok{0.1}\NormalTok{)}
\end{Highlighting}
\end{Shaded}

\begin{verbatim}
  10% 
3.244 
\end{verbatim}

Para obter as medidas de variabilidade, no caso, variância e
desvio-padrão, respectivamente:

\texttt{var(nome\_variável)}

\texttt{sd(nome\_variável)}

Ex. Calcule as medidas de variabilidade com a variável
\textbf{Pessoas\_familia}

\begin{Shaded}
\begin{Highlighting}[]
\KeywordTok{var}\NormalTok{(Pessoas_familia)}
\end{Highlighting}
\end{Shaded}

\begin{verbatim}
[1] 3.245
\end{verbatim}

\begin{Shaded}
\begin{Highlighting}[]
\KeywordTok{sd}\NormalTok{(Pessoas_familia)}
\end{Highlighting}
\end{Shaded}

\begin{verbatim}
[1] 1.801
\end{verbatim}

A função \texttt{subset()}:

Com esta função podemos fazer cálculos utilizando filtros,
simultaneamente. A aplicação de filtros é extremamente útil quando
queremos explorar os dados de forma rápida e eficiente.

Exemplos:

Ex. 1) Altura das pessoas do sexo masculino: com a função abaixo o R
gera um subconjunto com as alturas de todas as pessoas do sexo
masculino.

\begin{Shaded}
\begin{Highlighting}[]
\KeywordTok{subset}\NormalTok{(}\StringTok{`}\DataTypeTok{Altura_(m)}\StringTok{`}\NormalTok{, Sexo}\OperatorTok{==}\StringTok{"Masculino"}\NormalTok{)}
\end{Highlighting}
\end{Shaded}

\begin{verbatim}
 [1] 1.90 1.76 1.83 1.81 1.67 1.55 1.60 1.84 1.80 1.60 1.75 1.73 1.68 1.81 1.90
[16] 1.80 1.56 1.65 1.60 1.61 1.59 1.75 1.59 1.89 1.62 1.60 1.50 1.65 1.79 1.65
[31] 1.79 1.67 1.59 1.71 1.60 1.72 1.73 1.65 1.65 1.50 1.57 1.86 1.85 1.80 1.77
[46] 1.81 1.73 1.80 1.66 1.71 1.60 1.72 1.81 1.55 1.60 1.80
\end{verbatim}

Ex. 2) Média das alturas das pessoas do sexo masculino: inserindo o
comando \texttt{mean()} ao subconjunto anterior, teremos como resultado
a média das alturas das pessoas do sexo masculino.

\begin{Shaded}
\begin{Highlighting}[]
\KeywordTok{mean}\NormalTok{(}\KeywordTok{subset}\NormalTok{(}\StringTok{`}\DataTypeTok{Altura_(m)}\StringTok{`}\NormalTok{, Sexo}\OperatorTok{==}\StringTok{"Masculino"}\NormalTok{))}
\end{Highlighting}
\end{Shaded}

\begin{verbatim}
[1] 1.702
\end{verbatim}

Ex. 3) Média das alturas das pessoas do sexo masculino com mais de 26
anos:

\begin{Shaded}
\begin{Highlighting}[]
\KeywordTok{mean}\NormalTok{(}\KeywordTok{subset}\NormalTok{(}\StringTok{`}\DataTypeTok{Altura_(m)}\StringTok{`}\NormalTok{, Sexo}\OperatorTok{==}\StringTok{"Masculino"}\OperatorTok{&}\StringTok{ }\NormalTok{Idade}\OperatorTok{>}\DecValTok{25}\NormalTok{))}
\end{Highlighting}
\end{Shaded}

\begin{verbatim}
[1] 1.654
\end{verbatim}

Ex. 4) Contagem de pessoas do sexo feminino que tenham menos de 60 kg:

\begin{Shaded}
\begin{Highlighting}[]
\KeywordTok{length}\NormalTok{(}\KeywordTok{subset}\NormalTok{(Sexo,Sexo}\OperatorTok{==}\StringTok{"Feminino"} \OperatorTok{&}\StringTok{ `}\DataTypeTok{Peso_(Kg)}\StringTok{`}\OperatorTok{<}\DecValTok{60}\NormalTok{))}
\end{Highlighting}
\end{Shaded}

\begin{verbatim}
[1] 94
\end{verbatim}

Ex. 5) Montando uma tabela para exibir o gênero de pessoas que
classificaram o Sabor como ``Pessimo'':

\begin{Shaded}
\begin{Highlighting}[]
\KeywordTok{table}\NormalTok{(}\KeywordTok{subset}\NormalTok{(Sexo, Sabor}\OperatorTok{==}\StringTok{"Pessimo"}\NormalTok{))}
\end{Highlighting}
\end{Shaded}

\begin{verbatim}

 Feminino Masculino 
        7         3 
\end{verbatim}

Este capítulo não teve a pretensão de esgotar o estudo de todos os
comandos a serem aplicados na estatística descritiva (veja help do R),
nem tampouco os conceitos estatísticos necessários à compreensão. Para
mais detalhes sobre os conceitos de estatística descritiva, você pode
consultar outras referências ou até mesmo as já citadas neste capítulo.

\hypertarget{inf}{%
\chapter{Estatística Inferencial}\label{inf}}

\emph{Tatiane Chassot}

\begin{flushright}
\emph{}
\end{flushright}

A inferência estatística, ou estatística inferencial, tem por objetivo
concluir e tomar decisões, com base em amostras (Figura
\ref{fig:infestat}). Usam-se dados extraídos de uma amostra para
produzir inferência sobre a população (LOPES et al.,
\protect\hyperlink{ref-lopes2008}{2008}).

\begin{figure}[H]

{\centering \includegraphics[width=0.8\linewidth]{infestat} 

}

\caption{Inferência Estatística}\label{fig:infestat}
\end{figure}

Em Estatística, o termo \textbf{população} é definido como conjunto de
indivíduos, ou itens, com pelo menos uma característica em comum,
podendo ser finita ou infinita (LOPES et al.,
\protect\hyperlink{ref-lopes2008}{2008}). Por exemplo, água de um rio,
sangue de uma pessoa, lote de peças produzidas por uma indústria,
eleitores de um município.

A \textbf{amostra} é um subconjunto, necessariamente finito, de uma
população e é selecionada de forma que todos os elementos da população
tenham a mesma chance de serem escolhidos.

\hypertarget{intervalo-de-confianca}{%
\section{Intervalo de Confiança}\label{intervalo-de-confianca}}

Entre as diferentes técnicas de Inferência Estatística, temos a
Estimação de Parâmetros, que consiste na determinação de um
\textbf{Intervalo de Confiança (IC)} para uma média ou proporção
populacional, ao um nível (1 - \(\alpha\))\% de confiança.

O nível de confiança (1 - \(\alpha\))\% normalmente varia de 90\% a
99\%.

\hypertarget{intervalo-de-confianca-para-uma-media-populacional}{%
\subsection{Intervalo de confiança para uma média
populacional}\label{intervalo-de-confianca-para-uma-media-populacional}}

Um \textbf{intervalo de confiança (IC)} é o \textbf{intervalo} estimado
onde a média de um parâmetro tem uma dada probabilidade de ocorrer.
Comumente define-se como o \textbf{intervalo} onde há (1 - \(\alpha\))\%
de probabilidade da média verdadeira da população inteira ocorrer.

IC (limite inferior \(\leq\) \(\mu\) \(\leq\) limite superior) = (1 -
\(\alpha\))\%

No software RStudio, o Intervalo de Confiança pode ser obtido usando o
teste t.

\textbf{Exemplo 1}: Os dados amostrais a seguir representam o número de
horas de estudos semanais para a disciplina de Estatística Básica, de
uma amostra de 10 alunos:

19 18 20 16 18 19 19 17 22 21

Qual é o intervalo de confiança para a média populacional de onde essa
amostra foi retirada?

\begin{Shaded}
\begin{Highlighting}[]
\NormalTok{horasestudo=}\KeywordTok{c}\NormalTok{(}\DecValTok{19}\NormalTok{,}\DecValTok{18}\NormalTok{,}\DecValTok{20}\NormalTok{,}\DecValTok{16}\NormalTok{,}\DecValTok{18}\NormalTok{,}\DecValTok{19}\NormalTok{,}\DecValTok{19}\NormalTok{,}\DecValTok{17}\NormalTok{,}\DecValTok{22}\NormalTok{,}\DecValTok{21}\NormalTok{)}
\KeywordTok{t.test}\NormalTok{(horasestudo)}
\end{Highlighting}
\end{Shaded}

\begin{verbatim}

    One Sample t-test

data:  horasestudo
t = 33, df = 9, p-value = 1e-10
alternative hypothesis: true mean is not equal to 0
95 percent confidence interval:
 17.62 20.18
sample estimates:
mean of x 
     18.9 
\end{verbatim}

IC (17,6 \(\leq\) \(\mu\) \(\leq\) 20,2) = 95\%

Com 95\% de confiança, a média populacional das horas semanais de estudo
para a disciplina de Estatística Básica está entre 17,6 e 20,2 horas. Ou
seja, qualquer aluno (de onde essa amostra foi retirada) estuda em
média, de 17,6 a 20,2 horas por semana.

Se não informarmos o nível de confiança, o software R considera 95\%. No
entanto, para mudar o nível de confiança para 90\%, acrescentamos a
informação \texttt{conf.level\ =\ 0.90} após o nome da variável:

\begin{Shaded}
\begin{Highlighting}[]
\KeywordTok{t.test}\NormalTok{(horasestudo, }\DataTypeTok{conf.level =} \FloatTok{0.90}\NormalTok{)}
\end{Highlighting}
\end{Shaded}

\begin{verbatim}

    One Sample t-test

data:  horasestudo
t = 33, df = 9, p-value = 1e-10
alternative hypothesis: true mean is not equal to 0
90 percent confidence interval:
 17.86 19.94
sample estimates:
mean of x 
     18.9 
\end{verbatim}

IC (17,9 \(\leq\) \(\mu\) \(\leq\) 19,9) = 90\%

Com 90\% de confiança, a média populacional das horas semanais de estudo
para a disciplina de Estatística Básica está entre 17,9 e 19,9 horas. Ou
seja, qualquer aluno (de onde essa amostra foi retirada) estuda em
média, de 17,9 a 19,9 horas por semana.

Para mudar o nível de confiança para 99\%:

\begin{Shaded}
\begin{Highlighting}[]
\KeywordTok{t.test}\NormalTok{(horasestudo, }\DataTypeTok{conf.level =} \FloatTok{0.99}\NormalTok{)}
\end{Highlighting}
\end{Shaded}

\begin{verbatim}

    One Sample t-test

data:  horasestudo
t = 33, df = 9, p-value = 1e-10
alternative hypothesis: true mean is not equal to 0
99 percent confidence interval:
 17.06 20.74
sample estimates:
mean of x 
     18.9 
\end{verbatim}

IC (17,1 \(\leq\) \(\mu\) \(\leq\) 20,7) = 99\%

Com 99\% de confiança, a média populacional das horas semanais de estudo
para a disciplina de Estatística Básica está entre 17,1 e 20,7 horas. Ou
seja, qualquer aluno (de onde essa amostra foi retirada) estuda em
média, de 17,1 a 20,7 horas por semana.

\hypertarget{para-verificar-normalidade-dos-dados}{%
\subsection{Para verificar normalidade dos
dados}\label{para-verificar-normalidade-dos-dados}}

Algumas técnicas de inferência estatística têm como requisitos a
normalidade dos dados. Para verificar se os dados seguem uma
distribuição normal, podemos, inicialmente usar o histograma e depois
confirmar com um teste estatístico para testar normalidade como
Shapiro-Wilk ou Kolmogorov-Smirnov.

Hipóteses do teste:

\begin{itemize}
\tightlist
\item
  \textbf{H0}: os dados seguem uma distribuição normal
\item
  \textbf{H1}: os dados não seguem uma distribuição normal
\end{itemize}

O \textbf{valor p} reflete a plausibilidade de se obter tais resultados
no caso de H0 ser de fato verdadeira.

\begin{figure}[H]

{\centering \includegraphics[width=0.8\linewidth]{testehip1} 

}

\caption{Teste de hipóteses}\label{fig:testehip1}
\end{figure}

\begin{Shaded}
\begin{Highlighting}[]
\KeywordTok{shapiro.test}\NormalTok{(horasestudo)}
\end{Highlighting}
\end{Shaded}

\begin{verbatim}

    Shapiro-Wilk normality test

data:  horasestudo
W = 0.98, p-value = 0.9
\end{verbatim}

Como p \(>\) 0,05, não rejeita-se H0 e conclui-se que os dados seguem
uma distribuição normal.

\hypertarget{intervalo-de-confianca-para-uma-proporcao-populacional}{%
\subsection{Intervalo de confiança para uma proporção
populacional}\label{intervalo-de-confianca-para-uma-proporcao-populacional}}

IC (limite inferior \(\leq\) \(\pi\) \(\leq\) limite superior) = (1 -
\(\alpha\))\%

\textbf{Exemplo 2}: (adaptado de
\url{https://www.passeidireto.com/arquivo/3802950/capitulo7---intervalos-de-confianca})
Entre 500 pessoas entrevistadas a respeito de suas preferências
eleitorais, 260 mostraram-se favoráveis ao candidato B. Qual é a
proporção amostral dos favoráveis ao candidato B? E a proporção
populacional dos favoráveis?

Sintaxe no software RStudio:

\texttt{prop.test(x,n,conf.level=nível\ de\ confiança)}

Em que:

x = número de sucessos

n= tamanho da amostra

nível de confiança = 0,90 a 0,99

\begin{Shaded}
\begin{Highlighting}[]
\KeywordTok{prop.test}\NormalTok{(}\DecValTok{260}\NormalTok{,}\DecValTok{500}\NormalTok{)}
\end{Highlighting}
\end{Shaded}

\begin{verbatim}

    1-sample proportions test with continuity correction

data:  260 out of 500, null probability 0.5
X-squared = 0.72, df = 1, p-value = 0.4
alternative hypothesis: true p is not equal to 0.5
95 percent confidence interval:
 0.4752 0.5645
sample estimates:
   p 
0.52 
\end{verbatim}

A proporção amostral dos eleitores favoráveis ao candidato B é de 0,52.

IC (0,48 \(\leq\) \(\pi\) \(\leq\) 0,56) = 95\%

Com 95\% de confiança, a proporção populacional dos eleitores favoráveis
ao candidato B está entre 0,48 e 0,56.

Para mudar o nível de confiança para 90\%:

\begin{Shaded}
\begin{Highlighting}[]
\KeywordTok{prop.test}\NormalTok{(}\DecValTok{260}\NormalTok{,}\DecValTok{500}\NormalTok{,}\DataTypeTok{conf.level =} \FloatTok{0.90}\NormalTok{)}
\end{Highlighting}
\end{Shaded}

\begin{verbatim}

    1-sample proportions test with continuity correction

data:  260 out of 500, null probability 0.5
X-squared = 0.72, df = 1, p-value = 0.4
alternative hypothesis: true p is not equal to 0.5
90 percent confidence interval:
 0.4822 0.5575
sample estimates:
   p 
0.52 
\end{verbatim}

IC (0,48 \(\leq\) \(\pi\) \(\leq\) 0,56) = 90\%

Com 90\% de confiança, a proporção populacional dos eleitores favoráveis
ao candidato B está entre 0,48 e 0,56.

Para mudar o nível de confiança para 99\%:

\begin{Shaded}
\begin{Highlighting}[]
\KeywordTok{prop.test}\NormalTok{(}\DecValTok{260}\NormalTok{,}\DecValTok{500}\NormalTok{,}\DataTypeTok{conf.level =} \FloatTok{0.99}\NormalTok{)}
\end{Highlighting}
\end{Shaded}

\begin{verbatim}

    1-sample proportions test with continuity correction

data:  260 out of 500, null probability 0.5
X-squared = 0.72, df = 1, p-value = 0.4
alternative hypothesis: true p is not equal to 0.5
99 percent confidence interval:
 0.4616 0.5779
sample estimates:
   p 
0.52 
\end{verbatim}

IC (0,46 \(\leq\) \(\pi\) \(\leq\) 0,58) = 99\%

Com 99\% de confiança, a proporção populacional dos eleitores favoráveis
ao candidato B está entre 0,46 e 0,58.

\hypertarget{teste-de-hipoteses}{%
\section{Teste de hipóteses}\label{teste-de-hipoteses}}

O teste de hipóteses é uma outra forma de fazer inferência estatística.
Formula-se uma hipótese (H0) para um parâmetro populacional e, partir de
uma amostra dessa população, aceita-se ou rejeita-se esta hipótese.

\textbf{H0}: hipótese nula (sempre tem a condição de igualdade)

\textbf{H1}: hipótese alternativa (tem o sinal de \(\neq\), \(>\) ou
\(<\))

\hypertarget{teste-de-hipoteses-para-uma-media-populacional}{%
\subsection{Teste de hipóteses para uma média
populacional}\label{teste-de-hipoteses-para-uma-media-populacional}}

H0: \(\mu\) \(=\) \ldots{}\ldots{}

H1: \(\mu\) \(\neq\) \ldots{}\ldots{}

H0: \(\mu\) \(=\) \ldots{}\ldots{}.

H1: \(\mu\) \(>\) \ldots{}\ldots{}..

H0: \(\mu\) \(=\) \ldots{}\ldots{}

H1: \(\mu\) \(<\) \ldots{}\ldots{}

No software RStudio, usa-se o \texttt{t.test} para a realização do teste
de hipóteses para uma média populacional, levando-se em conta o valor de
p-value para aceitar ou rejeitar H0.

De acordo com as hipóteses, temos variações do \texttt{t.test}, conforme
segue:

sintaxe: \texttt{t.test(amostra,\ opções)}

\begin{itemize}
\tightlist
\item
  \textbf{amostra}: Vetor contendo a amostra da qual se quer testar a
  média populacional.
\item
  \textbf{opções}: alternative: string indicando a hipótese alternativa
  desejada. Valores possíveis: \texttt{"two-sided"}, \texttt{"less"} ou
  \texttt{"greater"}.
\item
  \(\mu\): valor indicando o verdadeiro valor da média populacional.
\end{itemize}

\textbf{Exemplo 3}: (adaptado de
\textless{}www.leg.ufpr.br/\textasciitilde{}paulojus/CE002/pratica/praticase8.xml\textgreater{}
) A precipitação pluviométrica mensal numa certa região nos últimos 9
meses foi a seguinte:

30,5 34,1 27,9 35,0 26,9 30,2 28,3 31,7 25,8

Construa um teste de hipóteses para saber se a média da precipitação
pluviométrica mensal é igual a 30,0 mm.

\textbf{H0}: \(\mu\) \(=\) 30 mm

\textbf{H1}: \(\mu\) \(\neq\) 30 mm

\begin{Shaded}
\begin{Highlighting}[]
\NormalTok{chuva=}\KeywordTok{c}\NormalTok{(}\FloatTok{30.5}\NormalTok{,}\FloatTok{34.1}\NormalTok{,}\FloatTok{27.9}\NormalTok{,}\DecValTok{35}\NormalTok{,}\FloatTok{26.9}\NormalTok{,}\FloatTok{30.2}\NormalTok{,}\FloatTok{28.3}\NormalTok{,}\FloatTok{31.7}\NormalTok{,}\FloatTok{25.8}\NormalTok{)}
\NormalTok{chuva}
\end{Highlighting}
\end{Shaded}

\begin{verbatim}
[1] 30.5 34.1 27.9 35.0 26.9 30.2 28.3 31.7 25.8
\end{verbatim}

\begin{Shaded}
\begin{Highlighting}[]
\KeywordTok{t.test}\NormalTok{(chuva,}\DataTypeTok{alt=}\StringTok{"two.sided"}\NormalTok{,}\DataTypeTok{mu=}\DecValTok{30}\NormalTok{)}
\end{Highlighting}
\end{Shaded}

\begin{verbatim}

    One Sample t-test

data:  chuva
t = 0.042, df = 8, p-value = 1
alternative hypothesis: true mean is not equal to 30
95 percent confidence interval:
 27.62 32.47
sample estimates:
mean of x 
    30.04 
\end{verbatim}

Conclusão: Aceita-se H0 e conclui-se que a precipitação pluviométrica é
igual a 30mm.

\textbf{Exemplo 4}: (adaptado de
\url{https://www.passeidireto.com/arquivo/5533375/lista-eststistica-pronta-p-3-prova-com-respostas/3})
Um empresário desconfia que o tempo médio de espera para atendimento de
seus clientes é superior a 20 minutos. Para testar essa hipótese ele
entrevistou 20 pessoas e questionou quanto tempo demorou para ser
atendido. O resultado dessa pesquisa foi o seguinte:

22 20 21 23 22 20 23 22 20 24 21 20 21 24 22 22 23 22 20 24

Teste a hipótese de que o tempo de espera é superior a 20 minutos.

\textbf{H0}: \(\mu\) \(=\) 20 minutos

\textbf{H1}: \(\mu\) \(>\) 20 minutos

\begin{Shaded}
\begin{Highlighting}[]
\NormalTok{tempo=}\KeywordTok{c}\NormalTok{(}\DecValTok{22}\NormalTok{,}\DecValTok{20}\NormalTok{,}\DecValTok{21}\NormalTok{,}\DecValTok{23}\NormalTok{,}\DecValTok{22}\NormalTok{,}\DecValTok{20}\NormalTok{,}\DecValTok{23}\NormalTok{,}\DecValTok{22}\NormalTok{,}\DecValTok{20}\NormalTok{,}\DecValTok{24}\NormalTok{,}\DecValTok{21}\NormalTok{,}\DecValTok{20}\NormalTok{,}\DecValTok{21}\NormalTok{,}\DecValTok{24}\NormalTok{,}\DecValTok{22}\NormalTok{,}\DecValTok{22}\NormalTok{,}\DecValTok{23}\NormalTok{,}\DecValTok{22}\NormalTok{,}\DecValTok{20}\NormalTok{,}\DecValTok{24}\NormalTok{)}
\NormalTok{tempo}
\end{Highlighting}
\end{Shaded}

\begin{verbatim}
 [1] 22 20 21 23 22 20 23 22 20 24 21 20 21 24 22 22 23 22 20 24
\end{verbatim}

\begin{Shaded}
\begin{Highlighting}[]
\KeywordTok{t.test}\NormalTok{(tempo,}\DataTypeTok{alt=}\StringTok{"greater"}\NormalTok{,}\DataTypeTok{mu=}\DecValTok{20}\NormalTok{)}
\end{Highlighting}
\end{Shaded}

\begin{verbatim}

    One Sample t-test

data:  tempo
t = 5.8, df = 19, p-value = 8e-06
alternative hypothesis: true mean is greater than 20
95 percent confidence interval:
 21.26   Inf
sample estimates:
mean of x 
     21.8 
\end{verbatim}

Conclusão: Rejeita-se H0 com nível de significância de 1\% e conclui-se
que o tempo de espera é superior a 20 minutos.

\textbf{Exemplo 5}: (adaptado de
\url{https://docs.ufpr.br/~vayego/pdf_11_2/pratica_04_zoo.pdf}) Os
resíduos industriais jogados nos rios, muitas vezes, absorvem oxigênio,
reduzindo assim o conteúdo do oxigênio necessário à respiração dos
peixes e outras formas de vida aquática. Uma lei estadual exige um
mínimo de 5 p.p.m. (Partes por milhão) de oxigênio dissolvido, a fim de
que o conteúdo de oxigênio seja suficiente para manter a vida aquática.
Seis amostras de água retiradas de um rio, durante a maré baixa,
revelaram os índices (em partes por milhão) de oxigênio dissolvido:

4,9 5,1 4,9 5,5 5,0 4,7

Estes dados são evidência para afirmar que o conteúdo de oxigênio é
menor que 5 partes por milhão?

\textbf{H0}: \(\mu\) \(=\) 5 ppm

\textbf{H1}: \(\mu\) \(<\) 5 ppm

\begin{Shaded}
\begin{Highlighting}[]
\NormalTok{amostras=}\KeywordTok{c}\NormalTok{(}\FloatTok{4.9}\NormalTok{,}\FloatTok{5.1}\NormalTok{,}\FloatTok{4.9}\NormalTok{,}\FloatTok{5.5}\NormalTok{,}\FloatTok{5.0}\NormalTok{,}\FloatTok{4.7}\NormalTok{)}
\KeywordTok{t.test}\NormalTok{(amostras,}\DataTypeTok{alt=}\StringTok{"less"}\NormalTok{,}\DataTypeTok{mu=}\DecValTok{5}\NormalTok{)}
\end{Highlighting}
\end{Shaded}

\begin{verbatim}

    One Sample t-test

data:  amostras
t = 0.15, df = 5, p-value = 0.6
alternative hypothesis: true mean is less than 5
95 percent confidence interval:
 -Inf 5.24
sample estimates:
mean of x 
    5.017 
\end{verbatim}

Conclusão: Aceita-se H0 e conclui-se que o conteúdo de oxigênio é igual
a 5 ppm.

\hypertarget{teste-de-hipoteses-para-uma-proporcao-populacional}{%
\subsection{Teste de hipóteses para uma proporção
populacional}\label{teste-de-hipoteses-para-uma-proporcao-populacional}}

H0: \(\pi\) \(=\) \ldots{}\ldots{}

H1: \(\pi\) \(\neq\) \ldots{}\ldots{}

H0: \(\pi\) \(=\) \ldots{}\ldots{}.

H1: \(\pi\) \(>\) \ldots{}\ldots{}..

H0: \(\pi\) \(=\) \ldots{}\ldots{}

H1: \(\pi\) \(<\) \ldots{}\ldots{}

No software RStudio, usa-se o prop.test para a realização do teste de
hipóteses para uma proporção populacional, levando-se em conta o valor
de p-value para aceitar ou rejeitar H0.

Sintaxe:

\texttt{prop.test(x,n,p=.....,alt=".....")}

em que:

x = número de sucessos;

n= tamanho da amostra;

p = proporção a ser testada;

alt = \texttt{"two.sided"}, \texttt{"greater"} ou \texttt{"less"}.

\textbf{Exemplo 6}: (adaptado de
\url{https://docs.ufpr.br/~soniaisoldi/TP707/Aula8.pdf}) Uma máquina
está regulada quanto produz 3\% de peças defeituosas. Uma amostra
aleatória de 80 peças selecionadas ao acaso apresentou 3 peças
defeituosas. Teste a hipótese de que a máquina está regulada.

\textbf{H0}: \(\pi\) \(=\) 3\%

\textbf{H1}: \(\pi\) \(\neq\) 3\%

\begin{Shaded}
\begin{Highlighting}[]
\KeywordTok{prop.test}\NormalTok{(}\DecValTok{3}\NormalTok{,}\DecValTok{80}\NormalTok{,}\DataTypeTok{p=}\FloatTok{0.03}\NormalTok{,}\DataTypeTok{alt=}\StringTok{"two.sided"}\NormalTok{)}
\end{Highlighting}
\end{Shaded}

\begin{verbatim}

    1-sample proportions test with continuity correction

data:  3 out of 80, null probability 0.03
X-squared = 0.0043, df = 1, p-value = 0.9
alternative hypothesis: true p is not equal to 0.03
95 percent confidence interval:
 0.009735 0.113171
sample estimates:
     p 
0.0375 
\end{verbatim}

Conclusão: Aceita-se H0 e conclui-se que a máquina produz 3\% de peças
defeituosas, ou seja, a máquina está regulada.

\textbf{Exemplo 7}: (adaptado de
\textless{}www.ebah.com.br/content/ABAAAAdLkAI/metodos-estatistico-und-v-lista-resolvida\textgreater{})
As condições de mortalidade de uma região são tais que a proporção de
nascidos que sobrevivem até 60 anos é de 0,6. Testar essa hipótese se em
1.000 nascimentos amostrados aleatoriamente, verificou-se 530
sobreviventes até 60 anos.

\textbf{H0}: \(\pi\) \(=\) 0,6

\textbf{H1}: \(\pi\) \(\neq\) 0,6

\begin{Shaded}
\begin{Highlighting}[]
\KeywordTok{prop.test}\NormalTok{(}\DecValTok{530}\NormalTok{,}\DecValTok{1000}\NormalTok{,}\DataTypeTok{p=}\FloatTok{0.6}\NormalTok{,}\DataTypeTok{alt=}\StringTok{"two.sided"}\NormalTok{)}
\end{Highlighting}
\end{Shaded}

\begin{verbatim}

    1-sample proportions test with continuity correction

data:  530 out of 1000, null probability 0.6
X-squared = 20, df = 1, p-value = 7e-06
alternative hypothesis: true p is not equal to 0.6
95 percent confidence interval:
 0.4985 0.5613
sample estimates:
   p 
0.53 
\end{verbatim}

Conclusão: Rejeita-se H0 com nível de significância de 1\% e conclui-se
que a proporção de nascidos que sobrevivem até os 60 anos é diferente de
0,6.

\textbf{Exemplo 8}: (adaptado de
\url{https://docs.ufpr.br/~jomarc/intervaloeteste.pdf}) Uma empresa
retira periodicamente amostras aleatórias de 500 peças de sua linha de
produção para análise da qualidade. As peças da amostra são
classificadas como defeituosas ou não, sendo que a política da empresa
exige que o processo produtivo seja revisto se houver evidência de mais
de 1,5\% de peças defeituosas. Na última amostra, foram encontradas nove
peças defeituosas. O processo precisa ser revisto?

\textbf{H0}: \(\pi\) \(=\) 1,5\%

\textbf{H1}: \(\pi\) \(>\) 1,5\%

\begin{Shaded}
\begin{Highlighting}[]
\KeywordTok{prop.test}\NormalTok{(}\DecValTok{9}\NormalTok{,}\DecValTok{500}\NormalTok{,}\DataTypeTok{p=}\FloatTok{0.015}\NormalTok{,}\DataTypeTok{alt=}\StringTok{"greater"}\NormalTok{)}
\end{Highlighting}
\end{Shaded}

\begin{verbatim}

    1-sample proportions test with continuity correction

data:  9 out of 500, null probability 0.015
X-squared = 0.14, df = 1, p-value = 0.4
alternative hypothesis: true p is greater than 0.015
95 percent confidence interval:
 0.009766 1.000000
sample estimates:
    p 
0.018 
\end{verbatim}

Conclusão: Não rejeita H0 e conclui-se que a proporção de peças
defeituosas é igual a 1,5\%, ou seja, o processo não precisa ser
revisto.

\textbf{Exemplo 9}: (adaptado de
\url{https://www.passeidireto.com/arquivo/25297344/aula-19---testes-para-proporcao})
Uma pesquisa conclui que 90\% dos médicos recomendam aspirina a
pacientes que têm filhos. Teste a afirmação contra a alternativa de que
a percentagem é inferior a 90\%, se numa amostra aleatória de 100
médicos, 80 recomendam aspirina.

\textbf{H0}: \(\pi\) \(=\) 90\%

\textbf{H1}: \(\pi\) \(<\) 90\%

\begin{Shaded}
\begin{Highlighting}[]
\KeywordTok{prop.test}\NormalTok{(}\DecValTok{80}\NormalTok{,}\DecValTok{100}\NormalTok{,}\DataTypeTok{p=}\FloatTok{0.90}\NormalTok{,}\DataTypeTok{alt=}\StringTok{"less"}\NormalTok{)}
\end{Highlighting}
\end{Shaded}

\begin{verbatim}

    1-sample proportions test with continuity correction

data:  80 out of 100, null probability 0.9
X-squared = 10, df = 1, p-value = 8e-04
alternative hypothesis: true p is less than 0.9
95 percent confidence interval:
 0.0000 0.8618
sample estimates:
  p 
0.8 
\end{verbatim}

Conclusão: Rejeita-se H0 com nível de significância de 1\% e conclui-se
que a proporção de médicos que recomendam aaspirina é inferior a 90\%.

\hypertarget{teste-de-hipotese-para-duas-medias}{%
\subsection{Teste de hipótese para duas
médias}\label{teste-de-hipotese-para-duas-medias}}

O teste de hipótese para duas médias aplica-se quando se deseja comparar
dois grupos:

\begin{figure}[H]

{\centering \includegraphics[width=0.8\linewidth]{testehip2} 

}

\caption{Teste de hipótese para dois grupos}\label{fig:testehip2}
\end{figure}

Podemos comparar duas médias de duas amostras dependentes, também
chamadas de pareadas, ou médias de duas amostras independentes.

\hypertarget{teste-de-hipoteses-duas-amostras-dependentes}{%
\subsubsection{Teste de hipóteses duas amostras
dependentes}\label{teste-de-hipoteses-duas-amostras-dependentes}}

\textbf{Exemplo 10}: Foi obtido o peso de seis indivíduos antes e após
um treinamento de exercício físico. Teste a hipótese de que a média
antes do treinamento é diferente da média após o treinamento.

\begin{table}

\caption{\label{tab:unnamed-chunk-108}Amostras dependentes}
\centering
\begin{tabular}[t]{l|r|r|r|r|r|r}
\hline
Indivíduo & A & B & C & D & E & F\\
\hline
Peso antes do treinamento & 99 & 62 & 74 & 59 & 70 & 73\\
\hline
Peso depois do treinamento & 94 & 62 & 66 & 58 & 70 & 76\\
\hline
\end{tabular}
\end{table}

No software RStudio, usa-se o \texttt{t.test} para a realização do teste
de hipóteses para uma média populacional, levando-se em conta o valor de
p-value para aceitar ou rejeitar H0.

Hipóteses:

\textbf{H0}: média antes \(=\) média depois

\textbf{H1}: média antes \(\neq\) média depois

\begin{Shaded}
\begin{Highlighting}[]
\NormalTok{antes=}\KeywordTok{c}\NormalTok{(}\DecValTok{99}\NormalTok{,}\DecValTok{62}\NormalTok{,}\DecValTok{74}\NormalTok{,}\DecValTok{59}\NormalTok{,}\DecValTok{70}\NormalTok{,}\DecValTok{73}\NormalTok{)}
\NormalTok{depois=}\KeywordTok{c}\NormalTok{(}\DecValTok{94}\NormalTok{,}\DecValTok{62}\NormalTok{,}\DecValTok{66}\NormalTok{,}\DecValTok{58}\NormalTok{,}\DecValTok{70}\NormalTok{,}\DecValTok{76}\NormalTok{)}
\KeywordTok{t.test}\NormalTok{(antes,depois,}\DataTypeTok{paired=}\OtherTok{TRUE}\NormalTok{)}
\end{Highlighting}
\end{Shaded}

\begin{verbatim}

    Paired t-test

data:  antes and depois
t = 1.1, df = 5, p-value = 0.3
alternative hypothesis: true difference in means is not equal to 0
95 percent confidence interval:
 -2.334  6.000
sample estimates:
mean of the differences 
                  1.833 
\end{verbatim}

Conclusão: Não rejeita-se H0 e conclui-se que a média de peso antes do
treinamento é igual à média de peso depois do treinamento.

\textbf{Exemplo 11}: (adaptado de
\textless{}www.inf.ufsc.br/\textasciitilde{}marcelo/testes2.html\textgreater{})
Dez cobaias foram submetidas ao tratamento de engorda com certa ração.
Os pesos em gramas, antes e após o teste são dados a seguir. Podemos
concluir que o uso da ração contribuiu para o aumento do peso médio dos
animais?

\begin{table}

\caption{\label{tab:unnamed-chunk-110}Amostras dependentes - caso 2}
\centering
\begin{tabular}[t]{l|r|r|r|r|r|r|r|r|r|r}
\hline
Cobaia & 1 & 2 & 3 & 4 & 5 & 6 & 7 & 8 & 9 & 10\\
\hline
Antes & 635 & 704 & 662 & 560 & 603 & 745 & 698 & 575 & 633 & 669\\
\hline
Depois & 640 & 712 & 681 & 558 & 610 & 740 & 707 & 585 & 635 & 682\\
\hline
\end{tabular}
\end{table}

\textbf{H0}: média antes \(=\) média depois

\textbf{H1}: média antes \(\neq\) média depois

\begin{Shaded}
\begin{Highlighting}[]
\NormalTok{cobaiaantes=}\KeywordTok{c}\NormalTok{(}\DecValTok{635}\NormalTok{,}\DecValTok{704}\NormalTok{,}\DecValTok{662}\NormalTok{,}\DecValTok{560}\NormalTok{,}\DecValTok{603}\NormalTok{,}\DecValTok{745}\NormalTok{,}\DecValTok{698}\NormalTok{,}\DecValTok{575}\NormalTok{,}\DecValTok{633}\NormalTok{,}\DecValTok{669}\NormalTok{)}
\NormalTok{cobaiadepois=}\KeywordTok{c}\NormalTok{(}\DecValTok{640}\NormalTok{,}\DecValTok{712}\NormalTok{,}\DecValTok{681}\NormalTok{,}\DecValTok{558}\NormalTok{,}\DecValTok{610}\NormalTok{,}\DecValTok{740}\NormalTok{,}\DecValTok{707}\NormalTok{,}\DecValTok{585}\NormalTok{,}\DecValTok{635}\NormalTok{,}\DecValTok{682}\NormalTok{)}
\KeywordTok{t.test}\NormalTok{(cobaiaantes,cobaiadepois,}\DataTypeTok{paired=}\OtherTok{TRUE}\NormalTok{)}
\end{Highlighting}
\end{Shaded}

\begin{verbatim}

    Paired t-test

data:  cobaiaantes and cobaiadepois
t = -3, df = 9, p-value = 0.02
alternative hypothesis: true difference in means is not equal to 0
95 percent confidence interval:
 -11.638  -1.562
sample estimates:
mean of the differences 
                   -6.6 
\end{verbatim}

Conclusão: Rejeita-se H0 com nível de significância de 5\% e conclui-se
que a média antes da engorda é diferente da média depois da engorda.

\hypertarget{teste-de-hipoteses-duas-amostras-independentes}{%
\subsubsection{Teste de hipóteses duas amostras
independentes}\label{teste-de-hipoteses-duas-amostras-independentes}}

Primeiramente precisamos saber se existe homogeneidade de variâncias
populacionais, a qual poderá ser verificada por meio de um teste de
homogeneidade de variâncias utilizando os dados das duas amostras.

\hypertarget{teste-para-verificar-homogeneidade-de-variancias}{%
\subsubsection{Teste para verificar homogeneidade de
variâncias}\label{teste-para-verificar-homogeneidade-de-variancias}}

\textbf{Exemplo 12}: (adaptado de
\url{https://www.ime.unicamp.br/~hildete/Aula_p12.pdf}) Dois tipos
diferentes de tecido devem ser comparados. Uma máquina de testes pode
comparar duas amostras ao mesmo tempo. O peso (em miligramas) para sete
experimentos foram:

\begin{table}

\caption{\label{tab:unnamed-chunk-112}Comparação de dois tipos diferentes de tecidos}
\centering
\begin{tabular}[t]{l|l|l|l|l|l|l|l}
\hline
Tecido A & 36 & 26 & 31 & 38 & 28 & 20 & 37\\
\hline
Tecido B & 39 & 27 & 35 & 42 & 31 & 39 & 22\\
\hline
\end{tabular}
\end{table}

Teste se um tecido é mais pesado que o outro.

\textbf{H0}: as variâncias são homogêneas

\textbf{H1}: as variâncias são heterogêneas

\begin{Shaded}
\begin{Highlighting}[]
\NormalTok{tecidoa=}\KeywordTok{c}\NormalTok{(}\DecValTok{36}\NormalTok{,}\DecValTok{26}\NormalTok{,}\DecValTok{31}\NormalTok{,}\DecValTok{38}\NormalTok{,}\DecValTok{28}\NormalTok{,}\DecValTok{20}\NormalTok{,}\DecValTok{37}\NormalTok{)}
\NormalTok{tecidob=}\KeywordTok{c}\NormalTok{(}\DecValTok{39}\NormalTok{,}\DecValTok{27}\NormalTok{,}\DecValTok{35}\NormalTok{,}\DecValTok{42}\NormalTok{,}\DecValTok{31}\NormalTok{,}\DecValTok{39}\NormalTok{,}\DecValTok{22}\NormalTok{)}
\KeywordTok{var.test}\NormalTok{(tecidoa,tecidob)}
\end{Highlighting}
\end{Shaded}

\begin{verbatim}

    F test to compare two variances

data:  tecidoa and tecidob
F = 0.84, num df = 6, denom df = 6, p-value = 0.8
alternative hypothesis: true ratio of variances is not equal to 1
95 percent confidence interval:
 0.1441 4.8823
sample estimates:
ratio of variances 
            0.8389 
\end{verbatim}

Conclusão: Não rejeita-se H0 e conclui-se que as variâncias são
homogêneas.

Agora podemos realizar o teste de comparação de duas amostras
independentes.

\textbf{H0}: média tecido A \(=\) média tecido B

\textbf{H1}: média tecido A \(\neq\) média tecido B

\begin{Shaded}
\begin{Highlighting}[]
\KeywordTok{t.test}\NormalTok{(tecidoa, tecidob, }\DataTypeTok{var.equal =} \OtherTok{TRUE}\NormalTok{, }\DataTypeTok{paired=}\OtherTok{FALSE}\NormalTok{)}
\end{Highlighting}
\end{Shaded}

\begin{verbatim}

    Two Sample t-test

data:  tecidoa and tecidob
t = -0.73, df = 12, p-value = 0.5
alternative hypothesis: true difference in means is not equal to 0
95 percent confidence interval:
 -10.815   5.386
sample estimates:
mean of x mean of y 
    30.86     33.57 
\end{verbatim}

Conclusão: Não rejeita-se H0 e conclui-se que a média de peso do tecido
A é igual à média de peso do tecido B.

\hypertarget{qui}{%
\chapter{Teste de Qui-Quadrado}\label{qui}}

\emph{Iara Denise Endruweit Battisti}

\begin{flushright}
\emph{}
\end{flushright}

Quando existem duas variáveis de interesse, a representação tabular das
frequências observadas pode ser feita através de uma tabela de
contingência, também chamada de tabela cruzada ou tabela de dupla
entrada. Cada interseção de uma linha com uma coluna é chamada de casela
e o valor que aparece em cada casela é a frequência observada, nomeada
como \(O_{ij}\), em que i corresponde a linha e j corresponde a coluna.

\hypertarget{teste-de-qui-quadrado-para-verificar-associacao-entre-duas-variaveis-qualitativas}{%
\section{Teste de qui-quadrado para verificar associação entre duas
variáveis
qualitativas}\label{teste-de-qui-quadrado-para-verificar-associacao-entre-duas-variaveis-qualitativas}}

\textbf{Exemplo 1}: Uma pesquisa sobre ``a exposição a agrotóxicos entre
trabalhadores rurais no município de Cerro Largo/RS'' foi desenvolvida
por Letiane Peccin Ristow, no ano de 2017 (dissertação e mestrado no
Programa de Pós-Graduação em Desenvolvimento e Políticas Públicas da
UFFS, Campus Cerro Largo. Na Tabela \ref{tab:tamprop} são apresentados
os resultados do ``tamanho da propriedade'' e ``armazenamento seguro do
EPI''. Para verificar a existência de associação significativa entre
essas duas variáveis utilizamos o teste de qui-quadrado, dado que são
duas variáveis qualitativas: variável 1 - tamanho da propriedade (até
25ha; 26ha ou mais) e variável 2 -- armazenamento seguro (sim; não).

Primeiramente definimos as seguintes hipóteses estatísticas:

H0: não existe associação entre tamanho da propriedade e armazenamento
seguro (as variáveis são independentes)

H1: existe associação entre tamanho da propriedade e armazenamento
seguro (as variáveis são dependentes)

\begin{longtable}[]{@{}lll@{}}
\caption{\label{tab:tamprop}Tamanho da propriedade e armazenamento seguro
dos agrotóxicos, agricultores de Cerro Largo, RS, 2017.}\tabularnewline
\toprule
\begin{minipage}[b]{0.35\columnwidth}\raggedright
\strut
\end{minipage} & \begin{minipage}[b]{0.32\columnwidth}\raggedright
\textbf{Armazenamento seguro}\strut
\end{minipage} & \begin{minipage}[b]{0.15\columnwidth}\raggedright
\strut
\end{minipage}\tabularnewline
\midrule
\endfirsthead
\toprule
\begin{minipage}[b]{0.35\columnwidth}\raggedright
\strut
\end{minipage} & \begin{minipage}[b]{0.32\columnwidth}\raggedright
\textbf{Armazenamento seguro}\strut
\end{minipage} & \begin{minipage}[b]{0.15\columnwidth}\raggedright
\strut
\end{minipage}\tabularnewline
\midrule
\endhead
\begin{minipage}[t]{0.35\columnwidth}\raggedright
\textbf{Tamanho da propriedade}\strut
\end{minipage} & \begin{minipage}[t]{0.32\columnwidth}\raggedright
Não\strut
\end{minipage} & \begin{minipage}[t]{0.15\columnwidth}\raggedright
Sim\strut
\end{minipage}\tabularnewline
\begin{minipage}[t]{0.35\columnwidth}\raggedright
Até 25 ha\strut
\end{minipage} & \begin{minipage}[t]{0.32\columnwidth}\raggedright
59\strut
\end{minipage} & \begin{minipage}[t]{0.15\columnwidth}\raggedright
8\strut
\end{minipage}\tabularnewline
\begin{minipage}[t]{0.35\columnwidth}\raggedright
26 ha ou mais\strut
\end{minipage} & \begin{minipage}[t]{0.32\columnwidth}\raggedright
31\strut
\end{minipage} & \begin{minipage}[t]{0.15\columnwidth}\raggedright
14\strut
\end{minipage}\tabularnewline
\bottomrule
\end{longtable}

Fonte: RISTOW (\protect\hyperlink{ref-Ristow2017}{2017}).

A estatística de teste para testar as hipóteses apresentadas é o
\(\chi^2\) (qui-quadrado):

\[
\chi^2_{cal}=\sum_{i=1}^{l}\sum_{j=1}^{c}\frac{(O_{ij}-E_{ij})^2}{E_{ij}}
\] em que:

\(l\): número de linhas

\(c\): número de colunas

\(O_{ij}\): frequência observada na linha i e coluna j

\(E_{ij}\): frequência esperada na linha i e coluna j

com grau de liberdade = \(gl = (c-1)(l-1)\).

A frequência esperada de uma casela é obtida pela multiplicação do total
da linha pelo total da coluna dividido pelo total geral. Por exemplo, a
frequência esperada é igual ao total da coluna 1 multiplicada pelo total
da linha 1 dividido pelo total geral, ou seja, (68x90)/112.

Porém, é importante conhecermos as pressuposições do teste de
qui-quadrado de Pearson. Para auxiliar no encaminhamento do teste
adequado para verificar a relação de duas variáveis qualitativas,
seguimos o seguinte check-list.

\hypertarget{check-list-para-escolher-o-teste-adequado-para-verificar-a-relacao-entre-duas-variaveis-qualitativas}{%
\section{Check list para escolher o teste adequado para verificar a
relação entre duas variáveis
qualitativas}\label{check-list-para-escolher-o-teste-adequado-para-verificar-a-relacao-entre-duas-variaveis-qualitativas}}

\begin{itemize}
\item
  O cálculo do teste de qui-quadrado deve ser somente com valores
  absolutos. Quando temos uma tabela 2x2, isto é, duas linhas e duas
  colunas, devemos utilizar o teste de qui-quadrado com correção de
  continuidade (correção de Yates). O motivo é que a distribuição de
  frequências observadas é discreta e está sendo aproximada pela
  distribuição qui-quadrado, que é contínua (BARBETTA,
  \protect\hyperlink{ref-barbetta1988}{2010}).
\item
  Não devemos aplicar o teste de qui-quadrado quando a frequência
  esperada em qualquer casela for menor que 5. Neste caso, devemos usar
  o teste exato de Fisher, para garantir o grau de certeza do teste.
\item
  Quando temos duas amostras pareadas (duas amostras dependentes),
  utilizamos o teste de McNemar.
\item
  Caso tenhamos interesse em avaliar a força da associação entre as duas
  variáveis, devemos utilizar algumas medidas de magnitude dessa força,
  como por exemplo, coeficiente de contingência, razão de prevalência,
  risco relativo e razão de chances (\emph{odds ratio}). Porém, essas
  medidas de magnitude são dependentes do tipo de delineamento do
  estudo.
\end{itemize}

Para aplicar o teste de qui-quadrado ou um alternativo no software R,
primeiramente precisamos informar os dados, podemos fazer isso de duas
formas:

\begin{enumerate}
\def\labelenumi{(\alph{enumi})}
\item
  incluindo os valores no formatado de tabela;
\item
  acessando os valores no banco de dados.
\end{enumerate}

\hypertarget{exemplo-utilizando-os-recursos-do-software-r}{%
\section{Exemplo utilizando os recursos do software
R}\label{exemplo-utilizando-os-recursos-do-software-r}}

Realizar o teste de associação para os dados da Tabela
\ref{tab:tamprop}, para isso, digitar os dados da tabela cruzada (tabela
de contingência) no formato de uma matriz, valor ij, considerando
i=linha e j=coluna, em sequência por coluna (por exemplo, digita-se
todos os valores da primeira coluna, depois digita-se todos os valores
da segunda coluna e assim sucessivamente).

Sintaxe no software R para incluir os valores no formato de tabela:

\begin{Shaded}
\begin{Highlighting}[]
\NormalTok{quiquadrado1<-}\KeywordTok{matrix}\NormalTok{(}\KeywordTok{c}\NormalTok{(}\DecValTok{59}\NormalTok{,}\DecValTok{31}\NormalTok{,}\DecValTok{8}\NormalTok{,}\DecValTok{14}\NormalTok{),}\DataTypeTok{nc=}\DecValTok{2}\NormalTok{)}
\NormalTok{quiquadrado1}
\end{Highlighting}
\end{Shaded}

\begin{verbatim}
     [,1] [,2]
[1,]   59    8
[2,]   31   14
\end{verbatim}

O comando \texttt{matrix} indica que os dados serão organizados em uma
matriz, \texttt{nc} indica o número de colunas da tabela, o operador
\texttt{\textless{}-} atribui os valores digitados no nome informado
pelo usuário que neste caso é \texttt{quiquadrado1}.

O segundo comando \texttt{quiquadrado1}, mostra a matriz elaborada, que
neste caso representa uma tabela cruzada de duas linhas e duas colunas,
conforme a Tabela \ref{tab:tamprop}.

Primeiramente, deve-se verificar a existência de alguma casela com
frequência esperada menor que 5.

\begin{Shaded}
\begin{Highlighting}[]
\KeywordTok{chisq.test}\NormalTok{(quiquadrado1)}\OperatorTok{$}\NormalTok{expected}
\end{Highlighting}
\end{Shaded}

\begin{verbatim}
      [,1]   [,2]
[1,] 53.84 13.161
[2,] 36.16  8.839
\end{verbatim}

Caso não exista, utiliza-se o teste de qui-quadrado com o comando
\texttt{chisq.test}.

\begin{Shaded}
\begin{Highlighting}[]
\KeywordTok{chisq.test}\NormalTok{(quiquadrado1)}
\end{Highlighting}
\end{Shaded}

\begin{verbatim}

    Pearson's Chi-squared test with Yates' continuity correction

data:  quiquadrado1
X-squared = 5.1, df = 1, p-value = 0.02
\end{verbatim}

Observa-se que o software R identificou a tabela 2x2 e aplicou a
correção de continuidade. Porém, podemos informar isso na linha de
comando, incluindo opção \texttt{correct\ =\ TRUE}:

\begin{Shaded}
\begin{Highlighting}[]
\KeywordTok{chisq.test}\NormalTok{(quiquadrado1, }\DataTypeTok{correct=}\OtherTok{TRUE}\NormalTok{)}
\end{Highlighting}
\end{Shaded}

\begin{verbatim}

    Pearson's Chi-squared test with Yates' continuity correction

data:  quiquadrado1
X-squared = 5.1, df = 1, p-value = 0.02
\end{verbatim}

Então devemos concluir pela rejeição ou não da H0 e interpretar esse
resultados.

Caso pelo menos uma casela tenha frequência esperada menor que 5 como
por exemplo na tabela abaixo , utilizamos o teste exato de Fisher.

\begin{longtable}[]{@{}lll@{}}
\caption{\label{tab:tamprop1}Tamanho da propriedade e devolução das
embalagens vazias de agrotóxico, agricultores de Cerro Largo, RS,
2017.}\tabularnewline
\toprule
\begin{minipage}[b]{0.37\columnwidth}\raggedright
\strut
\end{minipage} & \begin{minipage}[b]{0.34\columnwidth}\raggedright
\textbf{Devolução}\strut
\end{minipage} & \begin{minipage}[b]{0.14\columnwidth}\raggedright
\strut
\end{minipage}\tabularnewline
\midrule
\endfirsthead
\toprule
\begin{minipage}[b]{0.37\columnwidth}\raggedright
\strut
\end{minipage} & \begin{minipage}[b]{0.34\columnwidth}\raggedright
\textbf{Devolução}\strut
\end{minipage} & \begin{minipage}[b]{0.14\columnwidth}\raggedright
\strut
\end{minipage}\tabularnewline
\midrule
\endhead
\begin{minipage}[t]{0.37\columnwidth}\raggedright
\textbf{Tamanho da propriedade}\strut
\end{minipage} & \begin{minipage}[t]{0.34\columnwidth}\raggedright
Não\strut
\end{minipage} & \begin{minipage}[t]{0.14\columnwidth}\raggedright
Sim\strut
\end{minipage}\tabularnewline
\begin{minipage}[t]{0.37\columnwidth}\raggedright
Até 25 ha\strut
\end{minipage} & \begin{minipage}[t]{0.34\columnwidth}\raggedright
8\strut
\end{minipage} & \begin{minipage}[t]{0.14\columnwidth}\raggedright
59\strut
\end{minipage}\tabularnewline
\begin{minipage}[t]{0.37\columnwidth}\raggedright
26 ha ou mais\strut
\end{minipage} & \begin{minipage}[t]{0.34\columnwidth}\raggedright
3\strut
\end{minipage} & \begin{minipage}[t]{0.14\columnwidth}\raggedright
43\strut
\end{minipage}\tabularnewline
\bottomrule
\end{longtable}

Fonte: RISTOW (\protect\hyperlink{ref-Ristow2017}{2017}).

Definindo as hipóteses estatísticas:

H0: não existe associação entre tamanho da propriedade e devolução das
embalagens (as variáveis são independentes);

H1: existe associação entre tamanho da propriedade e devolução das
embalagens (as variáveis são dependentes).

Incluindo os valores:

\begin{Shaded}
\begin{Highlighting}[]
\NormalTok{quiquadrado2<-}\KeywordTok{matrix}\NormalTok{(}\KeywordTok{c}\NormalTok{(}\DecValTok{8}\NormalTok{,}\DecValTok{3}\NormalTok{,}\DecValTok{59}\NormalTok{,}\DecValTok{43}\NormalTok{),}\DataTypeTok{nc=}\DecValTok{2}\NormalTok{)}
\NormalTok{quiquadrado2}
\end{Highlighting}
\end{Shaded}

\begin{verbatim}
     [,1] [,2]
[1,]    8   59
[2,]    3   43
\end{verbatim}

Verificando se todas frequências esperadas são maiores ou iguais a 5.

\begin{Shaded}
\begin{Highlighting}[]
\KeywordTok{chisq.test}\NormalTok{(quiquadrado2)}\OperatorTok{$}\NormalTok{expected}
\end{Highlighting}
\end{Shaded}

\begin{verbatim}
      [,1]  [,2]
[1,] 6.522 60.48
[2,] 4.478 41.52
\end{verbatim}

Neste caso, o software R apresenta um ``aviso'' pois observa-se uma
frequência esperada menor que 5. Então devemos optar pelo teste exato de
Fisher.

\begin{Shaded}
\begin{Highlighting}[]
\KeywordTok{fisher.test}\NormalTok{(quiquadrado2)}
\end{Highlighting}
\end{Shaded}

\begin{verbatim}

    Fisher's Exact Test for Count Data

data:  quiquadrado2
p-value = 0.5
alternative hypothesis: true odds ratio is not equal to 1
95 percent confidence interval:
  0.4316 11.9646
sample estimates:
odds ratio 
     1.933 
\end{verbatim}

Então devemos concluir, através do valor p, pela rejeição ou não da H0 e
interpretar esse resultados.

\hypertarget{teste-de-associacao-com-duas-amostras-dependentes}{%
\section{Teste de associação com duas amostras
dependentes}\label{teste-de-associacao-com-duas-amostras-dependentes}}

No caso de amostras pareadas (dependentes), utiliza-se o teste de
McNemar para testar a associação.

\begin{Shaded}
\begin{Highlighting}[]
\NormalTok{dados1=}\KeywordTok{matrix}\NormalTok{(}\KeywordTok{c}\NormalTok{(}\DecValTok{5}\NormalTok{,}\DecValTok{10}\NormalTok{,}\DecValTok{12}\NormalTok{,}\DecValTok{8}\NormalTok{),}\DataTypeTok{nc=}\DecValTok{2}\NormalTok{)}
\NormalTok{dados1}
\end{Highlighting}
\end{Shaded}

\begin{verbatim}
     [,1] [,2]
[1,]    5   12
[2,]   10    8
\end{verbatim}

\begin{Shaded}
\begin{Highlighting}[]
\KeywordTok{mcnemar.test}\NormalTok{(dados1)}
\end{Highlighting}
\end{Shaded}

\begin{verbatim}

    McNemar's Chi-squared test with continuity correction

data:  dados1
McNemar's chi-squared = 0.045, df = 1, p-value = 0.8
\end{verbatim}

Importante observar que para executar o teste de McNemar: no software R
os dados na matriz (tabela de contingência) devem ser distribuídos da
mesma maneira tanto nas linhas quanto nas colunas. Isto é, ``a'' e ``d''
devem expressar o mesmo comportamento. Por exemplo: aprovado,
desaprovado, aprovado, desaprovado.

\begin{longtable}[]{@{}lll@{}}
\caption{\label{tab:tabcont}Tabela de Contingência.}\tabularnewline
\toprule
\begin{minipage}[b]{0.19\columnwidth}\raggedright
\strut
\end{minipage} & \begin{minipage}[b]{0.16\columnwidth}\raggedright
\textbf{Depois}\strut
\end{minipage} & \begin{minipage}[b]{0.16\columnwidth}\raggedright
\strut
\end{minipage}\tabularnewline
\midrule
\endfirsthead
\toprule
\begin{minipage}[b]{0.19\columnwidth}\raggedright
\strut
\end{minipage} & \begin{minipage}[b]{0.16\columnwidth}\raggedright
\textbf{Depois}\strut
\end{minipage} & \begin{minipage}[b]{0.16\columnwidth}\raggedright
\strut
\end{minipage}\tabularnewline
\midrule
\endhead
\begin{minipage}[t]{0.19\columnwidth}\raggedright
\textbf{Antes}\strut
\end{minipage} & \begin{minipage}[t]{0.16\columnwidth}\raggedright
Aprovado\strut
\end{minipage} & \begin{minipage}[t]{0.16\columnwidth}\raggedright
Desaprovado\strut
\end{minipage}\tabularnewline
\begin{minipage}[t]{0.19\columnwidth}\raggedright
Aprovado\strut
\end{minipage} & \begin{minipage}[t]{0.16\columnwidth}\raggedright
a\strut
\end{minipage} & \begin{minipage}[t]{0.16\columnwidth}\raggedright
b\strut
\end{minipage}\tabularnewline
\begin{minipage}[t]{0.19\columnwidth}\raggedright
Desaprovado\strut
\end{minipage} & \begin{minipage}[t]{0.16\columnwidth}\raggedright
c\strut
\end{minipage} & \begin{minipage}[t]{0.16\columnwidth}\raggedright
d\strut
\end{minipage}\tabularnewline
\bottomrule
\end{longtable}

Fonte: Dados simulados.

\textbf{Exemplo 2}: Uma pesquisa foi realizada para verificar o efeito
de um medicamento para perda de peso. O estudo foi realizado com 45
cobaias com características semelhantes. Na Tabela abaixo são
apresentadas a situação do peso antes e após a intervenção (utilização
do medicamento).

Como trata-se de duas amostras dependentes (antes e após) não podemos
aplicar o teste de qui-quadrado. O teste adequado é McNemar.

\begin{longtable}[]{@{}lll@{}}
\caption{\label{tab:sitcob}Situação do peso de cobaias do estudo antes e
após a intervenção.}\tabularnewline
\toprule
\begin{minipage}[b]{0.35\columnwidth}\raggedright
\strut
\end{minipage} & \begin{minipage}[b]{0.25\columnwidth}\raggedright
\textbf{Peso Após}\strut
\end{minipage} & \begin{minipage}[b]{0.22\columnwidth}\raggedright
\strut
\end{minipage}\tabularnewline
\midrule
\endfirsthead
\toprule
\begin{minipage}[b]{0.35\columnwidth}\raggedright
\strut
\end{minipage} & \begin{minipage}[b]{0.25\columnwidth}\raggedright
\textbf{Peso Após}\strut
\end{minipage} & \begin{minipage}[b]{0.22\columnwidth}\raggedright
\strut
\end{minipage}\tabularnewline
\midrule
\endhead
\begin{minipage}[t]{0.35\columnwidth}\raggedright
\textbf{Peso Antes}\strut
\end{minipage} & \begin{minipage}[t]{0.25\columnwidth}\raggedright
Adequado\strut
\end{minipage} & \begin{minipage}[t]{0.22\columnwidth}\raggedright
Sobrepeso\strut
\end{minipage}\tabularnewline
\begin{minipage}[t]{0.35\columnwidth}\raggedright
Aprovado\strut
\end{minipage} & \begin{minipage}[t]{0.25\columnwidth}\raggedright
15\strut
\end{minipage} & \begin{minipage}[t]{0.22\columnwidth}\raggedright
5\strut
\end{minipage}\tabularnewline
\begin{minipage}[t]{0.35\columnwidth}\raggedright
Desaprovado\strut
\end{minipage} & \begin{minipage}[t]{0.25\columnwidth}\raggedright
18\strut
\end{minipage} & \begin{minipage}[t]{0.22\columnwidth}\raggedright
7\strut
\end{minipage}\tabularnewline
\bottomrule
\end{longtable}

Fonte: Dados simulados.

Hipóteses estatísticas:

H0: As frequências das diferentes categorias ocorrem na mesma proporção
(Frequências b e c ocorrem na mesma proporção);

H1: As frequências b e c ocorrem em proporções diferentes, ou seja, as
mudanças são significativas.

\begin{Shaded}
\begin{Highlighting}[]
\NormalTok{mcnemar=}\KeywordTok{matrix}\NormalTok{(}\KeywordTok{c}\NormalTok{(}\DecValTok{15}\NormalTok{,}\DecValTok{18}\NormalTok{,}\DecValTok{5}\NormalTok{,}\DecValTok{7}\NormalTok{),}\DataTypeTok{nc=}\DecValTok{2}\NormalTok{)}
\NormalTok{mcnemar}
\end{Highlighting}
\end{Shaded}

\begin{verbatim}
     [,1] [,2]
[1,]   15    5
[2,]   18    7
\end{verbatim}

\begin{Shaded}
\begin{Highlighting}[]
\KeywordTok{chisq.test}\NormalTok{(mcnemar)}\OperatorTok{$}\NormalTok{expected}
\end{Highlighting}
\end{Shaded}

\begin{verbatim}
      [,1]  [,2]
[1,] 14.67 5.333
[2,] 18.33 6.667
\end{verbatim}

\begin{Shaded}
\begin{Highlighting}[]
\KeywordTok{mcnemar.test}\NormalTok{(mcnemar)}
\end{Highlighting}
\end{Shaded}

\begin{verbatim}

    McNemar's Chi-squared test with continuity correction

data:  mcnemar
McNemar's chi-squared = 6.3, df = 1, p-value = 0.01
\end{verbatim}

\hypertarget{teste-de-qui-quadrado-para-verificar-aderencia-a-uma-distribuicao}{%
\section{Teste de qui-quadrado para verificar aderência a uma
distribuição}\label{teste-de-qui-quadrado-para-verificar-aderencia-a-uma-distribuicao}}

Neste caso usamos o teste de qui-quadrado para verificar se o conjunto
de dados segue uma distribuição teórica especificada.

\textbf{Exemplo 3}: Deseja-se verificar se o número de borrachudos é o
mesmo em diferentes pontos da margem de um rio. O número de borrachudos
observados para cada ponto (local) é apresentado na Tabela
\ref{tab:borrach}.

\begin{table}

\caption{\label{tab:borrach}Número de borrachudos nos diferentes pontos}
\centering
\begin{tabular}[t]{l|r}
\hline
Ponto & Borrachudos\\
\hline
Ponto 1 & 19\\
\hline
Ponto 2 & 12\\
\hline
Ponto 3 & 10\\
\hline
Ponto 4 & 17\\
\hline
Ponto 5 & 25\\
\hline
Ponto 6 & 22\\
\hline
Ponto 7 & 15\\
\hline
\end{tabular}
\end{table}

Fonte: Dados simulados.

Para um nível de 5\% de significância, as hipóteses a serem testadas:

H0: O número de borrachudos não muda conforme o ponto;

H1: Pelo menos um dos pontos tem número de borrachudos diferente dos
demais.

\begin{Shaded}
\begin{Highlighting}[]
\NormalTok{borrach<-}\KeywordTok{c}\NormalTok{(}\DecValTok{20}\NormalTok{,}\DecValTok{12}\NormalTok{,}\DecValTok{10}\NormalTok{,}\DecValTok{17}\NormalTok{,}\DecValTok{30}\NormalTok{,}\DecValTok{22}\NormalTok{,}\DecValTok{35}\NormalTok{)}
\KeywordTok{chisq.test}\NormalTok{(borrach)}\OperatorTok{$}\NormalTok{expected}
\end{Highlighting}
\end{Shaded}

\begin{verbatim}
[1] 20.86 20.86 20.86 20.86 20.86 20.86 20.86
\end{verbatim}

\begin{Shaded}
\begin{Highlighting}[]
\KeywordTok{chisq.test}\NormalTok{(borrach)}
\end{Highlighting}
\end{Shaded}

\begin{verbatim}

    Chi-squared test for given probabilities

data:  borrach
X-squared = 24, df = 6, p-value = 6e-04
\end{verbatim}

\textbf{Exemplo 4}: Suponha que desejamos verificar se o número de
borrachudos segue uma distribuição específica, informado em ``dist''.
Lembrando que os valores no vetor ``dist'' devem estar no formato de
proporção (por exemplo, 0,35).

H0: O número de borrachudos segue a distribuição teórica informada;

H1: O número de borrachudos não segue a distribuição teórica informada.

\begin{Shaded}
\begin{Highlighting}[]
\NormalTok{borrachudos<-}\KeywordTok{c}\NormalTok{(}\DecValTok{20}\NormalTok{,}\DecValTok{12}\NormalTok{,}\DecValTok{10}\NormalTok{,}\DecValTok{17}\NormalTok{,}\DecValTok{30}\NormalTok{,}\DecValTok{22}\NormalTok{,}\DecValTok{35}\NormalTok{)}
\NormalTok{dist<-}\KeywordTok{c}\NormalTok{(}\FloatTok{0.10}\NormalTok{,}\FloatTok{0.10}\NormalTok{,}\FloatTok{0.10}\NormalTok{,}\FloatTok{0.15}\NormalTok{,}\FloatTok{0.15}\NormalTok{,}\FloatTok{0.15}\NormalTok{,}\FloatTok{0.25}\NormalTok{)}
\KeywordTok{chisq.test}\NormalTok{(borrachudos)}\OperatorTok{$}\NormalTok{expected}
\end{Highlighting}
\end{Shaded}

\begin{verbatim}
[1] 20.86 20.86 20.86 20.86 20.86 20.86 20.86
\end{verbatim}

\begin{Shaded}
\begin{Highlighting}[]
\KeywordTok{chisq.test}\NormalTok{(borrachudos, }\DataTypeTok{p=}\NormalTok{dist)}
\end{Highlighting}
\end{Shaded}

\begin{verbatim}

    Chi-squared test for given probabilities

data:  borrachudos
X-squared = 8.1, df = 6, p-value = 0.2
\end{verbatim}

\hypertarget{reg}{%
\chapter{Modelos de Regressão}\label{reg}}

\emph{Iara Denise Endruweit Battisti}

\emph{Erikson Kaszubowski}

\begin{flushright}
\emph{}

\emph{}
\end{flushright}

Muitas vezes há a necessidade de estudar duas ou mais variáveis ao mesmo
tempo com o objetivo de predizer uma variável em função da(s) outra(s).
Por exemplo, verificar se sólidos removidos de um material relaciona-se
com o tempo de secagem e qual é a forma dessa relação. Outros exemplos:
relação entre tempo de estudo e desempenho a uma avaliação; relação
entre investimento em comunicação e vendas; entre outros.

A análise de correlação permite verificar a relação entre duas variáveis
quantitativas. Os modelos de regressão permitem demonstrar a forma da
relação entre duas ou mais variáveis. Estudaremos os modelos de
regressão linear na qual a variáveis resposta (\(Y\)) é quantitativa e
as variáveis preditoras (\(X_i\)) são quantitativas ou qualitativas.

\hypertarget{correlacao-linear}{%
\section{Correlação linear}\label{correlacao-linear}}

É a técnica mais simples para estudar a relação entre duas variáveis. Os
dados compõem uma única amostra de pares de valores (\(x_i, y_i\)),
correspondendo aos valores das variáveis X e Y, respectivamente, feitas
em cada elemento da amostra. Para analisar a existência de relação entre
as duas variáveis, primeiramente pode-se fazer o Diagrama de Dispersão.

\hypertarget{diagrama-de-dispersao}{%
\section{Diagrama de dispersão}\label{diagrama-de-dispersao}}

É um gráfico para verificar a existência de relação entre as variáveis X
e Y. É composto por pontos, os quais correspondem aos pares de valores
(\(xi, y_i\)), sendo a variável X representada no eixo horizontal e a
variável Y representada no eixo vertical.

O diagrama de disperção fornece uma visualização gráfica do
comportamento conjunto das duas variávei em estudo. Na Figura
\ref{fig:diag} a percebe-se uma correlação (relação) linear positiva
entre as variáveis X e Y, ou seja, os valores das duas variaveis crescem
conjuntamente, já na Figura \ref{fig:diag}b percebe-se uma correlação
linear negativa entre as variáveis X e Y, neste caso, os valores de uma
variável crescem enquanto os valores da outra variável decrescem. A
Figura \ref{fig:diag}c informa a ausência de relação entre as duas
variáveis e, a Figura \ref{fig:diag}d mostra uma relação não linear, a
qual não será objeto de estudo nesta publicação.

\begin{figure}[H]

{\centering \includegraphics[width=0.8\linewidth]{correlacao1} 

}

\caption{Diagramas de Dispersão}\label{fig:diag}
\end{figure}

Fonte: Elaborado pelo(s) autor(es).

\textbf{Exemplo}: Suponha que 15 alunos foram selecionados
aleatoriamente na turma de Estatística, sendo registrado o tempo de
estudo e nota da atividade avaliativa. O objetivo da pesquisa é
verificar se existe relação entre tempo de estudo e nota.

\begin{longtable}[]{@{}lllllllllllllll@{}}
\caption{\label{tab:reg1}Relação entre o tempo de estudo e a
nota.}\tabularnewline
\toprule
\begin{minipage}[b]{0.10\columnwidth}\raggedright
\textbf{Tempo}\strut
\end{minipage} & \begin{minipage}[b]{0.04\columnwidth}\raggedright
4,0\strut
\end{minipage} & \begin{minipage}[b]{0.04\columnwidth}\raggedright
6,0\strut
\end{minipage} & \begin{minipage}[b]{0.04\columnwidth}\raggedright
5,5\strut
\end{minipage} & \begin{minipage}[b]{0.04\columnwidth}\raggedright
5,0\strut
\end{minipage} & \begin{minipage}[b]{0.04\columnwidth}\raggedright
6,8\strut
\end{minipage} & \begin{minipage}[b]{0.04\columnwidth}\raggedright
6,5\strut
\end{minipage} & \begin{minipage}[b]{0.04\columnwidth}\raggedright
3,5\strut
\end{minipage} & \begin{minipage}[b]{0.04\columnwidth}\raggedright
4,5\strut
\end{minipage} & \begin{minipage}[b]{0.04\columnwidth}\raggedright
7,5\strut
\end{minipage} & \begin{minipage}[b]{0.04\columnwidth}\raggedright
8,0\strut
\end{minipage} & \begin{minipage}[b]{0.04\columnwidth}\raggedright
5,4\strut
\end{minipage} & \begin{minipage}[b]{0.04\columnwidth}\raggedright
6,5\strut
\end{minipage} & \begin{minipage}[b]{0.04\columnwidth}\raggedright
7,7\strut
\end{minipage} & \begin{minipage}[b]{0.04\columnwidth}\raggedright
7,5\strut
\end{minipage}\tabularnewline
\midrule
\endfirsthead
\toprule
\begin{minipage}[b]{0.10\columnwidth}\raggedright
\textbf{Tempo}\strut
\end{minipage} & \begin{minipage}[b]{0.04\columnwidth}\raggedright
4,0\strut
\end{minipage} & \begin{minipage}[b]{0.04\columnwidth}\raggedright
6,0\strut
\end{minipage} & \begin{minipage}[b]{0.04\columnwidth}\raggedright
5,5\strut
\end{minipage} & \begin{minipage}[b]{0.04\columnwidth}\raggedright
5,0\strut
\end{minipage} & \begin{minipage}[b]{0.04\columnwidth}\raggedright
6,8\strut
\end{minipage} & \begin{minipage}[b]{0.04\columnwidth}\raggedright
6,5\strut
\end{minipage} & \begin{minipage}[b]{0.04\columnwidth}\raggedright
3,5\strut
\end{minipage} & \begin{minipage}[b]{0.04\columnwidth}\raggedright
4,5\strut
\end{minipage} & \begin{minipage}[b]{0.04\columnwidth}\raggedright
7,5\strut
\end{minipage} & \begin{minipage}[b]{0.04\columnwidth}\raggedright
8,0\strut
\end{minipage} & \begin{minipage}[b]{0.04\columnwidth}\raggedright
5,4\strut
\end{minipage} & \begin{minipage}[b]{0.04\columnwidth}\raggedright
6,5\strut
\end{minipage} & \begin{minipage}[b]{0.04\columnwidth}\raggedright
7,7\strut
\end{minipage} & \begin{minipage}[b]{0.04\columnwidth}\raggedright
7,5\strut
\end{minipage}\tabularnewline
\midrule
\endhead
\begin{minipage}[t]{0.10\columnwidth}\raggedright
\textbf{Nota}\strut
\end{minipage} & \begin{minipage}[t]{0.04\columnwidth}\raggedright
5,5\strut
\end{minipage} & \begin{minipage}[t]{0.04\columnwidth}\raggedright
7,5\strut
\end{minipage} & \begin{minipage}[t]{0.04\columnwidth}\raggedright
8,0\strut
\end{minipage} & \begin{minipage}[t]{0.04\columnwidth}\raggedright
7,0\strut
\end{minipage} & \begin{minipage}[t]{0.04\columnwidth}\raggedright
8,1\strut
\end{minipage} & \begin{minipage}[t]{0.04\columnwidth}\raggedright
8,6\strut
\end{minipage} & \begin{minipage}[t]{0.04\columnwidth}\raggedright
4,7\strut
\end{minipage} & \begin{minipage}[t]{0.04\columnwidth}\raggedright
7,5\strut
\end{minipage} & \begin{minipage}[t]{0.04\columnwidth}\raggedright
9,5\strut
\end{minipage} & \begin{minipage}[t]{0.04\columnwidth}\raggedright
9,5\strut
\end{minipage} & \begin{minipage}[t]{0.04\columnwidth}\raggedright
7,8\strut
\end{minipage} & \begin{minipage}[t]{0.04\columnwidth}\raggedright
8,0\strut
\end{minipage} & \begin{minipage}[t]{0.04\columnwidth}\raggedright
9,1\strut
\end{minipage} & \begin{minipage}[t]{0.04\columnwidth}\raggedright
8,0\strut
\end{minipage}\tabularnewline
\bottomrule
\end{longtable}

Fonte: Dados simulados.

Sintaxe no software R:

\texttt{plot(x,y)}

\begin{Shaded}
\begin{Highlighting}[]
\NormalTok{tempo=}\KeywordTok{c}\NormalTok{(}\DecValTok{4}\NormalTok{,}\DecValTok{6}\NormalTok{,}\FloatTok{5.5}\NormalTok{,}\DecValTok{5}\NormalTok{,}\FloatTok{6.8}\NormalTok{,}\FloatTok{6.5}\NormalTok{,}\FloatTok{3.5}\NormalTok{,}\FloatTok{4.5}\NormalTok{,}\DecValTok{7}\NormalTok{,}\DecValTok{8}\NormalTok{,}\FloatTok{5.4}\NormalTok{,}\FloatTok{6.5}\NormalTok{,}\FloatTok{7.7}\NormalTok{,}\FloatTok{7.5}\NormalTok{,}\FloatTok{5.8}\NormalTok{)}
\NormalTok{nota=}\KeywordTok{c}\NormalTok{(}\FloatTok{5.5}\NormalTok{,}\FloatTok{7.5}\NormalTok{,}\DecValTok{8}\NormalTok{,}\DecValTok{7}\NormalTok{,}\FloatTok{8.1}\NormalTok{,}\FloatTok{8.6}\NormalTok{,}\FloatTok{4.7}\NormalTok{,}\FloatTok{7.5}\NormalTok{,}\FloatTok{9.5}\NormalTok{,}\FloatTok{9.5}\NormalTok{,}\FloatTok{7.8}\NormalTok{,}\DecValTok{8}\NormalTok{,}\FloatTok{9.1}\NormalTok{,}\FloatTok{9.5}\NormalTok{,}\DecValTok{8}\NormalTok{)}
\end{Highlighting}
\end{Shaded}

O diagrama de dispersão do exemplo está representado abaixo.

\begin{Shaded}
\begin{Highlighting}[]
\KeywordTok{plot}\NormalTok{(tempo,nota)}
\end{Highlighting}
\end{Shaded}

\begin{figure}[H]

{\centering \includegraphics[width=0.8\linewidth]{index_files/figure-latex/unnamed-chunk-128-1} 

}

\caption{Diagrama de dispersão da nota em relação ao tempo de estudo dos participantes do estudo}\label{fig:unnamed-chunk-128}
\end{figure}

Fonte: Elaborado pelo(s) autor(es).

\hypertarget{coeficiente-de-correlacao-linear-de-pearson}{%
\section{Coeficiente de Correlação Linear de
Pearson}\label{coeficiente-de-correlacao-linear-de-pearson}}

O coeficiente de correlação linear de Pearson (Karl Pearson 1857-1936)
mede o grau de relacionamento linear entre os valores pareados \(x_i\) e
\(y_i\) em uma amostra. O coeficiente linear de Pearson é obtido da
seguinte forma:

\[
r=\frac{n\sum xy-(\sum x)(\sum y)}{\sqrt{n(\sum x^2)-(\sum x)^2} \sqrt{(\sum  y^2)-(\sum y)^2}}
\]

em que:

\begin{itemize}
\tightlist
\item
  n = número de pares na amostra
\item
  x: valores da variável x
\item
  y: valores da variável y
\end{itemize}

O coeficiente de correlação linear (r) é uma estatística amostral,
representando a magnitude da relação entre duas variáveis na amostra. O
parâmetro populacional é representado por \(\rho\). O coeficiente de
correlação linear assume valores entre -1 e +1, inclusive. Se o valor de
r está próximo de 0, conclui-se que não há correlação linear entre as
variáveis X e Y. Seo valor de r está próximo de -1 ou +1, conclui-se
pela existência de correlação linear significativa entre as variáveis X
e Y, sendo que o sinal indica uma relação linear positiva (direta) ou
negativa (inversa).

Sintaxe no software R:

\texttt{cor(x,y)}

Obs: x e y são numéricos.

\begin{Shaded}
\begin{Highlighting}[]
\KeywordTok{cor}\NormalTok{(tempo,nota)}
\end{Highlighting}
\end{Shaded}

\begin{verbatim}
[1] 0.9224
\end{verbatim}

\hypertarget{modelo-de-regressao}{%
\section{Modelo de Regressão}\label{modelo-de-regressao}}

O estudo de regressão refere-se aos casos em que se pretende estabelecer
uma relação entre uma variável Y considerada dependente (variável
resposta ou desfecho) e uma ou mais variáveis \(x_1, x_2,\cdots, x_k\)
(variáveis explicativas ou preditoras) consideradas independentes.

O objetivo da análise de regressão é ajustar uma equação que permita
explicar o comportamento da variável resposta de maneira que o valor
previsto possa estar próximo do que seria observado. A forma do modelo
de regressão depende da relação entre as variáveis, expressa visualmente
pelo diagrama de dispersão, conforme Figura \ref{fig:diag}.

A análise de regressão é uma técnica muito utilizada em variáveis
quantitativas, como por exemplo:

\begin{itemize}
\item
  Vendas em função do investimento em comunicação;
\item
  Altura de crianças em função da idade;
\item
  Nota obtida em função de horas de estudo;
\item
  Produtividade de uma cultura em relação a quantidade de adubação.
\end{itemize}

Na Figura \ref{fig:regress} é apresentada a variação explicada e não
explicada na análise por modelo regressão.

\begin{figure}[H]

{\centering \includegraphics[width=0.8\linewidth]{regress1} 

}

\caption{Variação explicada e não explicada na análise de regressão}\label{fig:regress}
\end{figure}

Fonte: Elaborado pelo(s) autor(es).

Observa-se na Figura \ref{fig:regress}, uma identidade na regressão,
conforme a seguinte expressão:

\begin{figure}[H]

{\centering \includegraphics[width=0.8\linewidth]{regress2} 

}

\caption{Identidade da Regressão}\label{fig:regress2}
\end{figure}

Fonte: Elaborado pelo(s) autor(es).

Assim, a partir da expressão apresentada que o modelo de regressão será
mais adequado na medida em que a proporção de ``Soma de Quadrados de
Regressão'' é mais alta em relação à ``Soma de Quadrado Total'' do que a
``Soma de Quadrado do Resíduo''.

\hypertarget{modelo-de-regressao-linear-simples}{%
\section{Modelo de Regressão Linear
Simples}\label{modelo-de-regressao-linear-simples}}

O modelo de regressão linear simples é usado quando a resposta da
variável dependente se expressa de forma linear (Figura
\ref{fig:regress} e neste caso com apenas uma variável explicativa,
expresso da seguinte maneira (HOFFMANN; OTHERS,
\protect\hyperlink{ref-hoffmann1998}{1998}):

\[
y_i=\beta_0+\beta_1x_i+\varepsilon _i
\]

Em que:

\(y_i\): valores da variável resposta (dependente, desfecho),
\(i = 1,2,...,n\) observações;

\(x_i\): valores da variável explicativa (independente, preditora),
\(i = 1,2,...,n\) observações;

\(\beta_0\): coeficiente linear (intercepto). Interpretado como o valor
da variável dependente quando a variável independente é igual a 0;

\(\beta_1\): coeficiente angular (inclinação). Interpretado como
acréscimo/decréscimo na variável dependente para a variação de uma
unidade na variável independente;

\(\varepsilon_i\): erros aleatórios supostamente de uma população
normal, com média 0 e variância constante
\(\begin{bmatrix}\varepsilon_i N(0, \sigma^2)\end{bmatrix}\).

\hypertarget{metodo-dos-minimos-quadrados}{%
\section{Método dos Mínimos
Quadrados}\label{metodo-dos-minimos-quadrados}}

O método dos mínimos quadrados (MMQ) é utilizado para a obtenção dos
coeficientes linear e angular. Consiste em minimizar a Soma de Quadrados
de Resíduos, ou seja, minimizar:

\[
\sum (y_i-\hat y_i)^2=\sum (y_i-b_0-b_1x_i^2)
\]

As expressões para os coeficientes, que minimizam SQResíduos são obtidas
pela derivadas desta soma de quadrados em relação a \(b_0\) e em relação
a \(b_1\) e podem ser descritas por (HOFFMANN; OTHERS,
\protect\hyperlink{ref-hoffmann1998}{1998}):

\[
b_1=\frac{\sum xy-\frac{\sum x \sum y}{n}}{\sum x^2 - \frac{(\sum x)^2}{n}}
\]

em que:

\textbf{n}: número de pares na amostra;

\textbf{x}: valores da variável x;

\textbf{y}: valores da variável y.

e

\[
b_0=\bar{y}-b_1\bar{x}
\]

em que:

\(\bar{x}\): média aritmética dos valores de x;

\(\bar{y}\): média aritmética dos valores de y;

\(b_1\): valor calculado do coeficiente angular.

Obtendo-se a seguinte equação de regressão linear simples estimada:

\[
\hat{y}=b_0-b_1{x}
\]

em que:

\(b_0\): coeficiente linear estimado;

\(b_1\): coeficiente angular estimado;

\(x\): valores da variável explicativa.

Esta equação refere-se a reta de regressão, se \(b_1\) é um valor
positivo a reta é crescente, demonstrando uma relação positiva entre as
variáveis e se \(b_1\) é um o valor negativo, a reta é decrescente,
demonstrando uma relação inversa entre as variáveis.

Sintaxe no software R:

\texttt{regressao=lm(y\textasciitilde{}x)}

Obs: y são valores numéricos da variável resposta e x são valores
numéricos da variável preditora.

Por exemplo:

\begin{Shaded}
\begin{Highlighting}[]
\NormalTok{regressao=}\KeywordTok{lm}\NormalTok{(nota}\OperatorTok{~}\NormalTok{tempo)}
\NormalTok{regressao}
\end{Highlighting}
\end{Shaded}

\begin{verbatim}

Call:
lm(formula = nota ~ tempo)

Coefficients:
(Intercept)        tempo  
      2.221        0.947  
\end{verbatim}

\hypertarget{analise-de-variancia}{%
\section{Análise de Variância}\label{analise-de-variancia}}

A análise de variância (técnica introduzida por Fisher, na década de 20)
testa o ajuste da equação como um todo, ou seja, um teste para verificar
se a equação de regressão obtida é significativa ou não. No caso de
regressão linear simples, a análise de variância é definida como
apresentada na Tabela \ref{tab:varian}.

As hipóteses testadas na Análise de Variância da Regressão são:

\[
H_0:\beta_1=0 \textrm{(a regressao não é significativa)} 
\] \[
H_1:\beta_1 \neq 0 \textrm{(a regressão é significativa)}
\]

\begin{longtable}[]{@{}lllll@{}}
\caption{\label{tab:varian}Análise de variância para a regressão
linear.}\tabularnewline
\toprule
\begin{minipage}[b]{0.19\columnwidth}\raggedright
\textbf{FV}\strut
\end{minipage} & \begin{minipage}[b]{0.10\columnwidth}\raggedright
\textbf{GL}\strut
\end{minipage} & \begin{minipage}[b]{0.16\columnwidth}\raggedright
\textbf{SQ}\strut
\end{minipage} & \begin{minipage}[b]{0.17\columnwidth}\raggedright
\textbf{QM}\strut
\end{minipage} & \begin{minipage}[b]{0.07\columnwidth}\raggedright
\textbf{F}\strut
\end{minipage}\tabularnewline
\midrule
\endfirsthead
\toprule
\begin{minipage}[b]{0.19\columnwidth}\raggedright
\textbf{FV}\strut
\end{minipage} & \begin{minipage}[b]{0.10\columnwidth}\raggedright
\textbf{GL}\strut
\end{minipage} & \begin{minipage}[b]{0.16\columnwidth}\raggedright
\textbf{SQ}\strut
\end{minipage} & \begin{minipage}[b]{0.17\columnwidth}\raggedright
\textbf{QM}\strut
\end{minipage} & \begin{minipage}[b]{0.07\columnwidth}\raggedright
\textbf{F}\strut
\end{minipage}\tabularnewline
\midrule
\endhead
\begin{minipage}[t]{0.19\columnwidth}\raggedright
Regressão\strut
\end{minipage} & \begin{minipage}[t]{0.10\columnwidth}\raggedright
1\strut
\end{minipage} & \begin{minipage}[t]{0.16\columnwidth}\raggedright
SQRegressão\strut
\end{minipage} & \begin{minipage}[t]{0.17\columnwidth}\raggedright
QMRegressão\strut
\end{minipage} & \begin{minipage}[t]{0.07\columnwidth}\raggedright
Fc\strut
\end{minipage}\tabularnewline
\begin{minipage}[t]{0.19\columnwidth}\raggedright
Desvios\strut
\end{minipage} & \begin{minipage}[t]{0.10\columnwidth}\raggedright
n-2\strut
\end{minipage} & \begin{minipage}[t]{0.16\columnwidth}\raggedright
SQResíduos\strut
\end{minipage} & \begin{minipage}[t]{0.17\columnwidth}\raggedright
QMResíduos\strut
\end{minipage} & \begin{minipage}[t]{0.07\columnwidth}\raggedright
-\strut
\end{minipage}\tabularnewline
\begin{minipage}[t]{0.19\columnwidth}\raggedright
Total\strut
\end{minipage} & \begin{minipage}[t]{0.10\columnwidth}\raggedright
n-1\strut
\end{minipage} & \begin{minipage}[t]{0.16\columnwidth}\raggedright
SQTotal\strut
\end{minipage} & \begin{minipage}[t]{0.17\columnwidth}\raggedright
-\strut
\end{minipage} & \begin{minipage}[t]{0.07\columnwidth}\raggedright
-\strut
\end{minipage}\tabularnewline
\bottomrule
\end{longtable}

Fonte: Elaborado pelo(s) autor(es).

em que:

\[
SQ \textrm{Regressao} = \frac{(\sum xy - \frac{(\sum x \sum y)^2}{n})}{\sum x^2 - \frac{(\sum x)^2}{n}}
\]

\[
SQ \textrm{Total} = \sum y^2 - \frac{(\sum y)^2}{n}
\]

SQResíduo = SQTotal - SQRegressão

QMRegressão = SQRegressão \(/\) GLregressão

QMResíduo = SQResíduo \(/\) GLresíduo

Fc = QMRegressão \(/\) QMResíduo

Espera-se que o QMResíduo seja mínimo, assim o modelo de regressão
estará bem ajustado.

A distribuição de probabilidade para a razão de duas variâncias é
conhecida como a distribuição F. Se a hipótese nula for rejeitada ao
nível de signicância \(\alpha\), rejeita-se H0, portanto a regressão é
significativa.

Sintaze no software R:

\texttt{anova(regressao)}

Obs: regressao é o nome dado ao modelo de regressão.

Por exemplo:

\begin{Shaded}
\begin{Highlighting}[]
\KeywordTok{anova}\NormalTok{(regressao)}
\end{Highlighting}
\end{Shaded}

\begin{verbatim}
Analysis of Variance Table

Response: nota
          Df Sum Sq Mean Sq F value  Pr(>F)    
tempo      1   22.8   22.82    74.2 9.9e-07 ***
Residuals 13    4.0    0.31                    
---
Signif. codes:  0 '***' 0.001 '**' 0.01 '*' 0.05 '.' 0.1 ' ' 1
\end{verbatim}

\hypertarget{coeficiente-de-determinacao}{%
\section{Coeficiente de
Determinação}\label{coeficiente-de-determinacao}}

Representa o percentual de variação total que é explicada pela equação
de regressão, sendo obtido da seguinte forma:

\[
R^2 = \frac{\textrm{SQRegressao}}{SQTotal}
\]

Quanto mais próximo de 1 (ou 100\%), melhor será o ajuste da equação de
regressão. Também utiliza-se o coeficiente de determinação ajustado
(R\(^2\) ajustado), o qual considera o número de variáveis e o tamanho
da amostra, sendo este o mais indicado para regressão múltipla.

Sintaxe no software R:

\texttt{summary(regressao)}

Obs: regressao é o nome dado ao modelo de regressão.

Por exemplo:

\begin{Shaded}
\begin{Highlighting}[]
\KeywordTok{summary}\NormalTok{(regressao)}
\end{Highlighting}
\end{Shaded}

\begin{verbatim}

Call:
lm(formula = nota ~ tempo)

Residuals:
    Min      1Q  Median      3Q     Max 
-0.8372 -0.4109  0.0418  0.3733  1.0154 

Coefficients:
            Estimate Std. Error t value Pr(>|t|)    
(Intercept)    2.221      0.673    3.30   0.0057 ** 
tempo          0.947      0.110    8.61  9.9e-07 ***
---
Signif. codes:  0 '***' 0.001 '**' 0.01 '*' 0.05 '.' 0.1 ' ' 1

Residual standard error: 0.555 on 13 degrees of freedom
Multiple R-squared:  0.851, Adjusted R-squared:  0.839 
F-statistic: 74.2 on 1 and 13 DF,  p-value: 9.88e-07
\end{verbatim}

Para traçar a reta de regressão no diagrama de dispersão, utiliza-se o
seguinte comando:

Sintaxe no software R:

\texttt{abline(regressao)}

Obs: regressao é o nome dado ao modelo de regressão.

Para o exemplo:

\begin{Shaded}
\begin{Highlighting}[]
\KeywordTok{plot}\NormalTok{(nota}\OperatorTok{~}\NormalTok{tempo)}
\KeywordTok{abline}\NormalTok{(}\KeywordTok{coef}\NormalTok{(regressao))}
\end{Highlighting}
\end{Shaded}

\begin{figure}[H]

{\centering \includegraphics[width=0.8\linewidth]{index_files/figure-latex/unnamed-chunk-132-1} 

}

\caption{Reta de regressão ajustada da nota em relação ao tempo de estudo dos participantes da pesquisa}\label{fig:unnamed-chunk-132}
\end{figure}

Fonte: Elaborado pelo(s) autor(es).

O intervalo de 95\% de confiança para os coeficientes de regressão são
obtidos, no software R, da seguinte forma:

Sintaxe no software R:

\texttt{confint(regressao)}

Obs: regressao é o nome dado ao modelo de regressão.

Para o exemplo:

\begin{Shaded}
\begin{Highlighting}[]
\KeywordTok{confint}\NormalTok{(regressao)}
\end{Highlighting}
\end{Shaded}

\begin{verbatim}
             2.5 % 97.5 %
(Intercept) 0.7671  3.676
tempo       0.7097  1.185
\end{verbatim}

\hypertarget{analise-dos-residuos}{%
\section{Análise dos Resíduos}\label{analise-dos-residuos}}

Para a validade dos intervalos de confiança e teste de hipótese torna-se
necessário supor que as observações de Y sejam independentes e o termo
de erro tenha distribuição aproximadamente normal com média 0 e
variância constante.

O método gráfico pode ser utilizado para testar estas suposições,
descrevendo que após a estimação dos parâmetros do modelo, pode-se
calcular os resíduos, através da diferença entre os valores observados y
e os valores preditos \(\hat{y}\), associados a cada x usado na análise.
Faz-se então um gráfico com os pares (\(x,\varepsilon\)), sendo
\(\varepsilon = y -\hat{y}\) (BARBETTA,
\protect\hyperlink{ref-barbetta1988}{2010}).

Se o modelo ajustado for apropriado para os dados, os pontos devem estar
distribuídos de forma aleatória no gráfico dos resíduos, conforme Figura
\ref{fig:residuos}a. Caso a suposição não seja satisfeita, métodos
alternativos podem ser utilizados como: método dos mínimos quadrados
ponderados para o caso de não homocedasticidade; o método dos mínimos
quadrados generalizados para o caso de erros correlacionados; e, métodos
não-paramétricos para o caso de não normalidade.

Além da análise gráfica, existem testes para avaliar a homocedasticidade
como o Teste de Bartlett e para avaliar a normalidade aplicam-se os
testes de Shapiro Wilks ou Kolmogorov-Smirnov.

\begin{figure}[H]

{\centering \includegraphics[width=0.8\linewidth]{residuos1} 

}

\caption{Gráficos para análise de resíduos em regressão}\label{fig:residuos}
\end{figure}

Fonte: Elaborado pelo(s) autor(es).

O primeiro gráfico de resíduos que podemos elaborar é para representar
os valores ajustados pela equação de regressão ajustada no eixo x e os
valores dos resíduos no eixo y, conforme segue.

Sintaxe no software R:

\texttt{plot(fitted(regressao),residuals(regressao),}

\texttt{xlab="Valores\ ajustados",ylab="Resíduos")}

Obs: \texttt{regressao} é o nome dado ao modelo de regressão, fitted
define os valores ajustados no eixo x; \texttt{residuals} define os
valores ajustados no eixo Y; \texttt{xlab} indica o nome do eixo x e
\texttt{ylab} indica o nome do eixo y.

\texttt{abline(h=0)} (obs: adicionar uma linha constante em y=0).

Na Figura \ref{fig:residuos1} é apresentado o gráfico de resíduo, no
qual os resíduos são apresentados no eixo y e os valores ajustados são
apresentados no eixo x.

\begin{Shaded}
\begin{Highlighting}[]
\KeywordTok{plot}\NormalTok{(}\KeywordTok{fitted}\NormalTok{(regressao), }\KeywordTok{residuals}\NormalTok{(regressao),}
\DataTypeTok{xlab=}\StringTok{"Valores ajustados"}\NormalTok{, }\DataTypeTok{ylab=}\StringTok{"Residuos"}\NormalTok{)}
\KeywordTok{abline}\NormalTok{(}\DataTypeTok{h=}\DecValTok{0}\NormalTok{)}
\end{Highlighting}
\end{Shaded}

\begin{figure}[H]

{\centering \includegraphics[width=0.8\linewidth]{index_files/figure-latex/residuos1-1} 

}

\caption{Gráfico dos resíduos em relação aos valores ajustados para os dados do exemplo}\label{fig:residuos1}
\end{figure}

Fonte: Elaborado pelo(s) autor(es).

Outro gráfico de resíduos que é possível elaborar na análise de resíduos
representa a variável preditora (x) no eixo x e o resíduos no eixo Y.

Sintaxe no software R:

\texttt{plot(tempo,residuals(regressao),}

\texttt{xlab="Valores\ independente",\ ylab="Resíduos")}

Obs: \texttt{regressao} é o nome dado ao modelo de regressão; a variável
x define os valores do eixo x e residuals define os valores ajustados no
eixo Y; \texttt{xlab} indica o nome do eixo x e ylab indica o nome do
eixo y.

\texttt{abline(h=0)}

Obs: adicionar uma linha constante em y=0.

Por exemplo:

\begin{Shaded}
\begin{Highlighting}[]
\KeywordTok{plot}\NormalTok{(tempo, }\KeywordTok{residuals}\NormalTok{(regressao), }\DataTypeTok{xlab =} \StringTok{"Valores independentes"}\NormalTok{,}
\DataTypeTok{ylab=}\StringTok{"Residuos"}\NormalTok{)}
\KeywordTok{abline}\NormalTok{(}\DataTypeTok{h=}\DecValTok{0}\NormalTok{)}
\end{Highlighting}
\end{Shaded}

\begin{figure}[H]

{\centering \includegraphics[width=0.8\linewidth]{index_files/figure-latex/residuos2-1} 

}

\caption{Gráfico gerado pelo RStudio para análise dos resíduos com os valores da variável independente}\label{fig:residuos2}
\end{figure}

Fonte: Elaborado pelo(s) autor(es).

Na Figura \ref{fig:residuos2} é apresentado o gráfico de resíduo, em que
no eixo y constam os valores dos resíduos e no eixo x constam os valores
da variável independente.

Considerando os dados do exemplo, suponha que um aluno estudou 6,5 horas
(x=6,5), então o valor ajustado da nota (y ) é dado por
2,2214+0,9474*6,5, resultando em 8,38. Para esse caso, o resíduo é:

Yobservado -- Yestimado =8 --8,38 = -0,38

Para exibir os valores ajustados e os resíduos da equação de regressão
utilizam-se os seguintes comandos:

Sintaxe no software R:

\texttt{regressao\$residuals} (exibe os resíduos do modelo regressao).

\texttt{regressao\$fitted.values} (exibe os valores ajustados do modelo
regressao).

Por exemplo:

\begin{Shaded}
\begin{Highlighting}[]
\NormalTok{regressao}\OperatorTok{$}\NormalTok{residuals}
\end{Highlighting}
\end{Shaded}

\begin{verbatim}
       1        2        3        4        5        6        7        8 
-0.51087 -0.40561  0.56807  0.04176 -0.56351  0.22070 -0.83718  1.01544 
       9       10       11       12       13       14       15 
 0.64701 -0.30036  0.46281 -0.37930 -0.41615  0.17333  0.28386 
\end{verbatim}

Para testar a suposição que os erros aleatórios têm distribuição normal,
pode-se elaborar o gráfico de probabilidade normal, conforme segue:

Sintaxe no software R:

\texttt{qqnorm(residuals(regressao))}

\begin{Shaded}
\begin{Highlighting}[]
\KeywordTok{qqnorm}\NormalTok{(}\KeywordTok{residuals}\NormalTok{(regressao))}
\end{Highlighting}
\end{Shaded}

\begin{figure}[H]

{\centering \includegraphics[width=0.8\linewidth]{index_files/figure-latex/qqnorm-1} 

}

\caption{Gráfico de probabilidade normal para verificar normalidade dos resíduos}\label{fig:qqnorm}
\end{figure}

Fonte: Elaborado pelo(s) autor(es).

Ainda, pode-se construir o gráfico com a distribuiçõa da probabilidade
dos resíduos, através de um histograma, verificando assim se a cauda é
simétrica ou não:

\begin{Shaded}
\begin{Highlighting}[]
\KeywordTok{hist}\NormalTok{(}\DataTypeTok{x =}\NormalTok{ regressao}\OperatorTok{$}\NormalTok{residuals,}
      \DataTypeTok{xlab =} \StringTok{"Resíduos"}\NormalTok{,}
      \DataTypeTok{ylab =} \StringTok{"Densidade"}\NormalTok{,}
      \DataTypeTok{main =} \StringTok{""}\NormalTok{,}
      \DataTypeTok{col =} \StringTok{"lightgreen"}\NormalTok{,}
      \DataTypeTok{probability =} \OtherTok{TRUE}\NormalTok{)}
\KeywordTok{lines}\NormalTok{(}\KeywordTok{density}\NormalTok{(regressao}\OperatorTok{$}\NormalTok{residuals))}
\end{Highlighting}
\end{Shaded}

\begin{figure}[H]

{\centering \includegraphics[width=0.8\linewidth]{index_files/figure-latex/unnamed-chunk-135-1} 

}

\caption{Histograma de distribuição da probabilidade para os resíduos}\label{fig:unnamed-chunk-135}
\end{figure}

Fonte: Elaborado pelo(s) autor(es).

Também, pode-se aplicar o teste de normalidade de Shapiro Wilk para
verificar a normalidade dos dados, confirmando a simetria ou não da
cauda do gráfico acima. O comando utilizado é o seguinte:

\texttt{shapiro.test(residuals(regressao))}

Obs: \texttt{residuals(regressão)} indica os resíduos do modelo de
regressão.

Por exemplo:

\begin{Shaded}
\begin{Highlighting}[]
\KeywordTok{shapiro.test}\NormalTok{(}\KeywordTok{residuals}\NormalTok{(regressao))}
\end{Highlighting}
\end{Shaded}

\begin{verbatim}

    Shapiro-Wilk normality test

data:  residuals(regressao)
W = 0.96, p-value = 0.6
\end{verbatim}

\hypertarget{valores-outliers-na-regressao}{%
\subsection{Valores outliers na
regressão}\label{valores-outliers-na-regressao}}

Para análise dos valores outliers nos resíduos (\emph{residuals
standard} e \emph{residuals studentized}), utilizam-se os seguintes
comandos:

Sintaxe no software R:

\texttt{rstudent(regressao)}

\texttt{rstandard(regressao)}

\begin{Shaded}
\begin{Highlighting}[]
\KeywordTok{rstudent}\NormalTok{(regressao)}
\end{Highlighting}
\end{Shaded}

\begin{verbatim}
       1        2        3        4        5        6        7        8 
-1.04742 -0.74389  1.07142  0.07646 -1.07311  0.40066 -2.01860  2.29138 
       9       10       11       12       13       14       15 
 1.26283 -0.60069  0.86125 -0.69777 -0.81958  0.32859  0.51493 
\end{verbatim}

\begin{Shaded}
\begin{Highlighting}[]
\KeywordTok{rstandard}\NormalTok{(regressao)}
\end{Highlighting}
\end{Shaded}

\begin{verbatim}
       1        2        3        4        5        6        7        8 
-1.04353 -0.75701  1.06538  0.07956 -1.06691  0.41426 -1.81531  1.98916 
       9       10       11       12       13       14       15 
 1.23490 -0.61602  0.86993 -0.71196 -0.83013  0.34048  0.53013 
\end{verbatim}

E o gráfico para verificar valores outliers nos resíduos:

Sintaxe no software R:

\texttt{plot(rstudent(regressao))}

\texttt{plot(rstandard(regressao))}

Os gráficos dos resíduos padronizados (standard) e studentizados
(student) estão apresentados nas Figuras \ref{fig:residpad} e
\ref{fig:residst}, respectivamente.

Para o exemplo:

\begin{Shaded}
\begin{Highlighting}[]
\KeywordTok{plot}\NormalTok{(}\KeywordTok{rstandard}\NormalTok{(regressao))}
\KeywordTok{abline}\NormalTok{(}\DataTypeTok{h=}\DecValTok{2}\NormalTok{,}\DataTypeTok{col=}\StringTok{"red"}\NormalTok{)}
\KeywordTok{abline}\NormalTok{(}\DataTypeTok{h=}\OperatorTok{-}\DecValTok{2}\NormalTok{,}\DataTypeTok{col=}\StringTok{"red"}\NormalTok{)}
\end{Highlighting}
\end{Shaded}

\begin{figure}[H]

{\centering \includegraphics[width=0.8\linewidth]{index_files/figure-latex/residpad-1} 

}

\caption{Resíduos padronizados para o exemplo}\label{fig:residpad}
\end{figure}

Fonte: Elaborado pelo(s) autor(es).

Aqueles valores fora do intervalo (-2,+2) são possíveis outliers.

\begin{Shaded}
\begin{Highlighting}[]
\KeywordTok{plot}\NormalTok{(}\KeywordTok{rstudent}\NormalTok{(regressao)) }
\KeywordTok{abline}\NormalTok{(}\DataTypeTok{h=}\DecValTok{2}\NormalTok{,}\DataTypeTok{col=}\StringTok{"red"}\NormalTok{)}
\KeywordTok{abline}\NormalTok{(}\DataTypeTok{h=}\OperatorTok{-}\DecValTok{2}\NormalTok{,}\DataTypeTok{col=}\StringTok{"red"}\NormalTok{)}
\end{Highlighting}
\end{Shaded}

\begin{figure}[H]

{\centering \includegraphics[width=0.8\linewidth]{index_files/figure-latex/residst-1} 

}

\caption{Resíduos studentizados para o exemplo}\label{fig:residst}
\end{figure}

Fonte: Elaborado pelo(s) autor(es).

\hypertarget{valores-influentes-na-regressao}{%
\subsection{Valores influentes na
regressão}\label{valores-influentes-na-regressao}}

Para análise dos valores influentes, utiliza-se:

Sintaxe no software R:

\texttt{dffits(regressao)}

Para esse exemplo:

\begin{Shaded}
\begin{Highlighting}[]
\KeywordTok{dffits}\NormalTok{(regressao)}
\end{Highlighting}
\end{Shaded}

\begin{verbatim}
       1        2        3        4        5        6        7        8 
-0.55767 -0.19884  0.30669  0.02611 -0.34386  0.11597 -1.34854  0.97320 
       9       10       11       12       13       14       15 
 0.43848 -0.32566  0.25379 -0.20196 -0.38792  0.14210  0.13902 
\end{verbatim}

Aqueles valores maiores que \(2*(p/n)^(1/2)\) são possíveis pontos
influentes. Em que, p = número de parâmetros do modelo e n = tamanho da
amostra.

Para esse exemplo:

\begin{Shaded}
\begin{Highlighting}[]
\DecValTok{2}\OperatorTok{*}\NormalTok{(}\DecValTok{2}\OperatorTok{/}\DecValTok{15}\NormalTok{)}\OperatorTok{^}\NormalTok{(}\DecValTok{1}\OperatorTok{/}\DecValTok{2}\NormalTok{)}
\end{Highlighting}
\end{Shaded}

\begin{verbatim}
[1] 0.7303
\end{verbatim}

O gráfico para detectar pontos influentes pode ser elaborado pelo
comando (o gráfico está apresentado na Figura \ref{fig:ptoinf}:

\begin{Shaded}
\begin{Highlighting}[]
\KeywordTok{plot}\NormalTok{(}\KeywordTok{dffits}\NormalTok{(regressao))}
\KeywordTok{abline}\NormalTok{(}\DataTypeTok{h=}\OperatorTok{-}\FloatTok{0.73}\NormalTok{,}\DataTypeTok{col=}\StringTok{"red"}\NormalTok{)}
\KeywordTok{abline}\NormalTok{(}\DataTypeTok{h=}\FloatTok{0.73}\NormalTok{,}\DataTypeTok{col=}\StringTok{"red"}\NormalTok{)}
\end{Highlighting}
\end{Shaded}

\begin{figure}[H]

{\centering \includegraphics[width=0.8\linewidth]{index_files/figure-latex/ptoinf-1} 

}

\caption{Pontos influentes para o exemplo}\label{fig:ptoinf}
\end{figure}

Fonte: Elaborado pelo(s) autor(es).

O comando \texttt{plot(regressao)} elabora diferentes gráficos para o
diagnóstico do modelo.

\hypertarget{intervalo-de-predicao}{%
\section{Intervalo de Predição}\label{intervalo-de-predicao}}

Após o ajuste da equação de regressão linear simples, verificada a
significância da equação (p \(<\) 0,05) e verificada que a equação
estimada se ajusta bem aos dados pelo valor do coeficiente de
determinação então podemos utilizar a para predizer valores da variável
Y (resposta) a partir de valores da variável X (explicativa). Caso a
regressão não seja significativa a melhor predição para a variável Y é
média dos valores de \(y\), ou seja, \(\hat{y}\).

A predição de valores só tem sentido nos seguintes casos:

\begin{itemize}
\tightlist
\item
  regressão significativa;
\item
  os valores de X devem estar dentro dos limites inferior e superior dos
  dados amostrais;
\item
  as inferências referem-se somente a população de onde a amostra
  aleatória foi extraída;
\item
  as suposições sobre os resíduos devem ser satisfeitas.
\end{itemize}

Quando tem-se um equação estimada do tipo \(\hat{y} = b_0 + b_1x\),
\(\hat{y}\) representa o valor predito da variável Y para um dado valor
da variável X, ou seja, é uma predição pontual, porém esta não informa a
sua precisão, a qual é contemplada no intervalo de predição (da mesma
forma do intervalo de confiança, já visto em inferência estatística).

O intervalo de predição para um determinado Y é dado por:

\[
\hat{y}\pm \varepsilon
\]

em que:

\[
\varepsilon = t_{(n-2;\frac{a}{2})}.S_e. \sqrt{ 1+ \frac{1}{n} +  \frac{n(x_p-\bar{x})^2}{n(\sum x^2)-(\sum x)^2} }
\]

onde:

\(x_p\): o valor dado para x

\(S_e\): o erro padrão da estimativa, definido por:

\[
S_e=\sqrt {\textrm{QMResiduo}}=\sqrt\frac{\sum(y-\hat{y})^2}{n-2}
\]

Assim, obtêm-se o intervalo de predição para um determinado Y, que
também pode ser expresso da seguinte forma:

\[
(\hat{y} - \varepsilon;\hat{y} + \varepsilon)
\]

Sintaxe no software R:

\texttt{x0=data.frame(x=valor\_numérico)}

Obs: x0 recebe o valor de x.

\texttt{predict(regressao,x0,interval="prediction")}

Obs: regressao é o nome dado ao modelo de regressão.

Para o exemplo R:

\begin{Shaded}
\begin{Highlighting}[]
\NormalTok{x0=}\KeywordTok{data.frame}\NormalTok{(}\DataTypeTok{tempo=}\FloatTok{5.5}\NormalTok{)}
\KeywordTok{predict}\NormalTok{(regressao, x0, }\DataTypeTok{interval=}\StringTok{"prediction"}\NormalTok{)}
\end{Highlighting}
\end{Shaded}

\begin{verbatim}
    fit   lwr   upr
1 7.432 6.189 8.675
\end{verbatim}

\hypertarget{rmark}{%
\chapter{RMarkdown}\label{rmark}}

\emph{Felipe Micail da Silva Smolski}

\begin{flushright}
\emph{}
\end{flushright}

\textbf{Markdown} é uma linguagem de marcação de textos utilizada para a
criação de diversos documentos, incluindo artigos, livros e
apresentações. A grande inovação do \textbf{RMarkdown} no RStudio neste
sentido é a utilização desta linguagem por meio do pacote
\texttt{rmarkdown} (arquivos .Rmd) para integrar a criação de documentos
com a análise e manipulação de dados em um único documento (Figura
\ref{fig:rmark}). Desta forma, é possível efetuar pesquisas científicas
que podem ser reproduzidas de forma muito mais fácil.

\begin{figure}[H]

{\centering \includegraphics[width=0.8\linewidth]{rmarkdown} 

}

\caption{Processo de criação de documentos no RMarkdown}\label{fig:rmark}
\end{figure}

Fonte: Adaptado de ALLAIRE et al.
(\protect\hyperlink{ref-R-rmarkdown}{2017}).

Para criação dos documentos é preciso a instalação dos pacotes
\texttt{rmarkdown} e \texttt{knitr} dentro do RStudio, bem como
sugere-se a instalação, no Windows, do programa MiKTeX
(\url{https://miktex.org/download}), que se encarrega de suporte à
configurações da linguagem de marcação de textos LaTeX no caso de
criação dos arquivos PDF.

\hypertarget{criando-o-documento}{%
\section{Criando o documento}\label{criando-o-documento}}

Para criação do documento RMarkdown, no RStudio clique em ``File \(>\)
New File \(>\) R Markdown'', ou mesmo através do atalho para criação de
documentos conforme mostra a Figura \ref{fig:criararq1}. Haverá a
escolha entre a criação de documentos (HTML, PDF e Word/Libre/Open
Office), a criação de uma apresentação (\emph{Presentation}), a criação
de um documento Shiny (documento dinâmico para criação de
\emph{dashboards}) e o carregamento de um modelo de documento
pré-estabelecido (\emph{From Template}).

Neste exemplo será criado um documento em Word, onde são preenchidos os
campos com o título do documento, o nome do autor e escolha o tipo de
documento.

\begin{figure}[H]

{\centering \includegraphics[width=0.8\linewidth]{criararq1} 

}

\caption{Criar documento RMarkdown}\label{fig:criararq1}
\end{figure}

Fonte: Elaborado pelo(s) autor(es).

\hypertarget{compilando-os-resultados-do-arquivo}{%
\section{Compilando os resultados do
arquivo}\label{compilando-os-resultados-do-arquivo}}

O \textbf{RMarkdown} cria um documento incial padrão, contendo alguns
exemplos básicos de inserção de textos e de formatação, que serão vistos
adiante. Para compilação do documento para o formato desejado (neste
caso Word), o usuário deve clicar na aba ``Knit \(>\) Knit to Word'', ou
pelo atalho no teclado CTRL+SHIFT+K.

\begin{figure}[H]

{\centering \includegraphics[width=0.8\linewidth]{compilar} 

}

\caption{Compilado o documento RMarkdown}\label{fig:compil}
\end{figure}

Fonte: Elaborado pelo(s) autor(es).

Caso ocorram erros com relação à codificação do documento, no que diz
respeito aos caracteres de acentuação da língua portuguesa, este pode
ser resolvido salvando o documento criado com a codificação UTF-8. Para
isto, clique em ``File \(>\) Save with Encoding \(>\) UTF-8''. Deve ser
feito este procedimento para cada tipo de arquivo: Word, HTML e PDF.

\begin{figure}[H]

{\centering \includegraphics[width=0.8\linewidth]{errocodif} 

}

\caption{Erro de codificação do documento RMarkdown}\label{fig:errocodif}
\end{figure}

Fonte: Elaborado pelo(s) autor(es).

\hypertarget{elementos-basicos-do-rmarkdown}{%
\section{Elementos básicos do
RMarkdown}\label{elementos-basicos-do-rmarkdown}}

A configuração básica de um arquivo RMarkdown divide-se entre a YAML
Header e o corpo do documento. A YAML (Yet Another Markup Language)
Header, ou metadados, é um cabeçalho onde são inseridas as informações
sobre o arquivo e das opções de compilação. Sempre devem iniciar o
documento, sendo inseridas dentro de dois campos de sinais \(---\).

Já abaixo do YAML, situa-se o local onde o pesquisador digitará o texto,
bem como integrará a inserção de códigos do R e também efetuará as
análises posteriores (análises descritivas, regressões, tabelas,
fórmulas, etc.). Por sua vez, os códigos do R (para manipulação de
dados, como visto até o capítulo anterior deste livro) são ``embutidos''
no texto por meio das \textbf{Code Chunks}. Já o texto é inserido
normalmente em forma de parágrafos (``fora'' dos Chunks), sendo que o
novo parágrafo é iniciado após pressionar a tecla ``Enter'' entre os
textos informados.

\begin{figure}[H]

{\centering \includegraphics[width=0.8\linewidth]{rmark2} 

}

\caption{Tela inicial do arquivo RMarkdown}\label{fig:rmark2}
\end{figure}

Elaborado pelo(s) autor(es).

Desta forma, ao efetuar a compilação do documento, o RStudio ``lê''
todas as informações inseridas no arquivo e cria como resultado um
arquivo escolhido com todas as análises feitas pelo usuário.

No exemplo acima (Figura \ref{fig:rmark2}), a compilação irá gerar um
arquivo em Word, de acordo com o \texttt{output} escolhido, no caso
\texttt{word\_document}. Se o usuário desejar gerar como arquivo de
texto final um documento que pode ser aberto inclusive em software
livre, pode utilizar o formato OpenDocument (.otd). Para isto, basta
substituir o \texttt{output} para \texttt{odt\_document}.

\hypertarget{elementos-basicos-de-formatacao}{%
\section{Elementos básicos de
formatação}\label{elementos-basicos-de-formatacao}}

Dentro do documento \textbf{RMarkdown}, depois dos metadados, começa o
espaço destinado ao texto do documento. Nesta etapa seguem algumas
condições para a formatação do texto, bem como da configuração dos
títulos e fórmulas matemáticas. A linguagem \emph{markdown} preza pela
simplicidade na formatação do texto, a qual posteriormente pode ser
exportada para diversos tipos de documentos de uma só vez. Desta forma,
como visto anteriormente, cria documentos totalmente dinâmicos entre si.

Os níveis de títulos dos documentos RMarkdown são definidos pelo símbolo
\texttt{\#}:

\begin{figure}[H]

{\centering \includegraphics[width=0.8\linewidth]{rmarktit} 

}

\caption{Títulos no RMarkdown}\label{fig:rmarktit}
\end{figure}

Fonte: Elaborado pelo(s) autor(es).

A acentuação das palavras, dentro do texto, é feita normalmente pelo
teclado do usuário. Os caracteres
\texttt{*\#/(){[}{]}\textless{}\textgreater{}} podem ser escritos
normalmente dentro do texto, no entanto os demais (exemplo do cifrão
\texttt{\$}) devem ser escritos precedidos de uma barra:
\texttt{\textbackslash{}\$}. Por outro lado, a formatação em itálico,
negrito, subscrito, sobrescrito, links e demais formatações são feitas
no documento (Figura \ref{fig:rmarkform}).

\begin{figure}[H]

{\centering \includegraphics[width=0.8\linewidth]{rmarkform} 

}

\caption{Formatação no RMarkdown}\label{fig:rmarkform}
\end{figure}

Fonte: Elaborado pelo(s) autor(es).

Como visto, é possível escrever as fórmulas em notação matemática, o que
facilita e muito a vida do pesquisador. No ambiente matemático do
\textbf{RMarkdown}, elas são escritas por meio da linguagem de marcação
de textos LaTeX. Existem muitos manuais sobre esta linguagem, e para
facilitar a escrita, sites como
\url{https://www.codecogs.com/latex/eqneditor.php?lang=pt-br} ajudam o
pesquisador nesta empreitada.

É possível efetuar a inserção de links nos documentos, para páginas
externas ou mesmo internas ao documento (Figura \ref{fig:rmarklinks}).

\begin{figure}[H]

{\centering \includegraphics[width=0.8\linewidth]{rmarklinks} 

}

\caption{Links no RMarkdown}\label{fig:rmarklinks}
\end{figure}

Fonte: Elaborado pelo(s) autor(es).

A inserção de imagens externas no documento, em diversos formatos (aqui
no exemplo .png) é feita a partir do direcionamento do nome da imagem
salva na mesma pasta do arquivo .Rmd criado, ou mesmo pelo link na
internet (Figura \ref{fig:rmarkimg}).

\begin{figure}[H]

{\centering \includegraphics[width=0.8\linewidth]{rmarkimg} 

}

\caption{Imagens no RMarkdown}\label{fig:rmarkimg}
\end{figure}

Fonte: Elaborado pelo(s) autor(es).

A Figura \ref{fig:rmarklist} demonstra algumas formas de criar listas e
itens no decorrer do corpo de texto no \textbf{RMarkdown}.

\begin{figure}[H]

{\centering \includegraphics[width=0.8\linewidth]{rmarklist} 

}

\caption{Listas no RMarkdown}\label{fig:rmarklist}
\end{figure}

Fonte: Elaborado pelo(s) autor(es).

A criação de tabelas simples segue a disposição dos elementos
pré-definidos, sendo que o alinhamento da coluna se dá pelo caractere
``\texttt{:}'' (dois pontos) conforme a Figura \ref{fig:rmarktab}:

\begin{figure}[H]

{\centering \includegraphics[width=0.8\linewidth]{rmarktab} 

}

\caption{Tabelas simples no RMarkdown}\label{fig:rmarktab}
\end{figure}

Fonte: Elaborado pelo(s) autor(es).

As notas de rodapé são inseridas no texto dentro das chaves precedidas
do acento circunflexo \texttt{\^{}{[}\ {]}}. O pesquisador adiciona-os
durante o texto, e o programa enumera automaticamente no documento final
em Word (Figura \ref{fig:rmarkrodape}).

\begin{figure}[H]

{\centering \includegraphics[width=0.8\linewidth]{rmarkrodape} 

}

\caption{Notas de rodapé no RMarkdown}\label{fig:rmarkrodape}
\end{figure}

Fonte: Elaborado pelo(s) autor(es).

\hypertarget{elementos-basicos-do-yaml}{%
\section{Elementos básicos do YAML}\label{elementos-basicos-do-yaml}}

O YAML, ou os metadados do documento, são informações básicas do
documento que podem ser alteradas (Figura \ref{fig:rmarkautor}). Dentre
elas \emph{title} define o título do documento; em \emph{author} é
inserido o autor ou autores e as informações do currículo do pesquisador
são inseridas via nota de rodapé dentro do símbolo
\texttt{\^{}{[}\ \ {]}}; o campo \emph{date} é opcional.

\begin{figure}[H]

{\centering \includegraphics[width=0.8\linewidth]{rmarkautor} 

}

\caption{Configuração do YAML}\label{fig:rmarkautor}
\end{figure}

Fonte: Elaborado pelo(s) autor(es).

Já o campo \emph{output} define a opção de salvamento do arquivo final.
Pode ser informado todos os tipos de arquivos previamente, sendo que no
momento da compilação será utilizado o primeiro tipo de arquivo, no
exemplo, em Word. Para salvar em PDF, é só colocar o campo
\texttt{pdf\_document} em primeiro lugar juntamente com a configuração
dentro deste tipo de arquivo.

Abaixo do tipo de arquivo a ser salvo, constam as opções de salvamento.
No caso do exemplo, abaixo de Word está constando a opção
\texttt{fig\_caption}, que dita se as figuras do documento em Word serão
inseridas com títulos.

Os campos \texttt{fig\_height} e \texttt{fig\_width} determinam a altura
e largura padrão de todas as imagens do documento Word. Abaixo seguem
algumas opções do YAML relacionando-se com a saída do documento em Word:

\begin{itemize}
\item
  \textbf{fig\_caption} - As figuras devem ter título?
\item
  \textbf{fig\_height}, \textbf{fig\_width} - Altura e largura padrão
  das imagens.
\item
  \textbf{highlight} - Estilo de saída pré-definido, inclui ``default'',
  ``tango'', ``pygments'', ``kate'', ``monochrome'', ``espresso'',
  ``zenburn'', e ``haddock''.
\item
  \textbf{keep\_md} - Salva uma cópia em arquivo .md juntamente com os
  outros arquivos.
\item
  \textbf{md\_extensions} - Extensões Markdown a serem incluídas como
  definições padrão no RMarkdown.
\item
  \textbf{pandoc\_args} - Argumentos adicionais para utilizar com o
  pandoc.
\item
  \textbf{reference\_docx} - Arquivo docx com as configurações de
  estilos de texto padrão. Deve ser salvo na mesma pasta do documento
  .rmd criado.
\item
  \textbf{toc} - Adiciona o sumário no início do texto.
\item
  \textbf{toc\_depth} - Determina o menor nível de títulos que será
  exibido no sumário. Exemplo, 1 mostra somente o primeiro nível.
\end{itemize}

Também é possível incluir um campo \texttt{abstract} para o resumo, no
caso de artigo e suas respectivas palavras-chave:

\begin{figure}[H]

{\centering \includegraphics[width=0.8\linewidth]{abstract} 

}

\caption{Abstract no YAML}\label{fig:abstract}
\end{figure}

Fonte: Elaborado pelo(s) autor(es).

\hypertarget{elementos-basicos-dos-chunks}{%
\section{Elementos básicos dos
Chunks}\label{elementos-basicos-dos-chunks}}

Os \textbf{Code Chunks}, como já visto, são espaços destinados à
inclusão de códigos diretamente do RStudio, como se inseríssemos a
informação em seu Console. Desta forma, por exemplo, se efetuarmos uma
operação matemática ou se carregarmos uma base de dados para ser
trabalhada, as rotinas serão efetuadas no momento em que for compilado o
arquivo .Rmd trabalhado.

A criação das Chunks é feita manualmente no corpo do documento .Rmd pela
inclusão do código ??? , ou via plataforma RStudio, no menu ``Insert
\(>\) Insert a new R chunk'', conforme demonstra a Figura
\ref{fig:rmarkchunk1}:

\begin{figure}[H]

{\centering \includegraphics[width=0.8\linewidth]{rmarkchunk1} 

}

\caption{Criação de Chunks}\label{fig:rmarkchunk1}
\end{figure}

Fonte: Elaborado pelo(s) autor(es).

Nota-se que o corpo do documento .Rmd ficou de outra cor, indicando que
está inserida uma Chunk naquele local. Dentro das chaves, a Chunk
divide-se entre uma identificação/nome para aquele campo (é opcional, no
entanto se constar não pode ser repetido no documento) e; as opções da
Chunk.

No exemplo abaixo, o nome da Chunk criada foi ``r nomedochunk''. E no
campo das opções, constaram \texttt{echo=FALSE}, \texttt{fig.height=10}
e \texttt{fig.width=5}. Lembrando que estes campos determinam as opções
somente para este chunk.

A primeira opção, \texttt{echo=FALSE}, informa que no arquivo compilado,
somente será mostrado o resultado da rotina inserida na Chunk (1+1),
portanto será mostrado somente o valor 2. Caso o usuário almejasse
inserir, no arquivo final, o código do R escrito (1+1) juntamente com o
resultado da operação, marcaria \texttt{echo=TRUE}.

\begin{figure}[H]

{\centering \includegraphics[width=0.8\linewidth]{rmarkchunk2} 

}

\caption{Criação de Chunks}\label{fig:rmarkchunk2}
\end{figure}

Fonte: Elaborado pelo(s) autor(es).

As opções \texttt{fig.height} e \texttt{fig.width} referem-se à altura e
largura caso o resultado final da Chunk fosse uma figura ou gráfico
derivado de dados inseridos na mesma. Vale lembrar que somente seriam
determinadas as medidas para esta Chunk.

Para padronizar todas as Chunks para que tenham as mesmas opções, uma
maneira utilizada usualmente é a inserção de uma \texttt{Chunk\ global}.
Ela é incluída no início do texto, sendo que a sua inclusão é
facultativa. No entanto, contribui para padronizar o texto, ao mesmo
tempo que se existir uma Chunk durante o texto que deva ser configurada
de forma diferente (por exemplo, o tamanho da imagem), pode ser efetuado
em cada Chunk individual.

\begin{figure}[H]

{\centering \includegraphics[width=0.8\linewidth]{rmarkchunkopt} 

}

\caption{Chunk global}\label{fig:rmarkchunkopt}
\end{figure}

Fonte: Elaborado pelo(s) autor(es).

Seguem algumas importantes opções das Chunks dos arquivos RMarkdown
(ALLAIRE et al., \protect\hyperlink{ref-R-rmarkdown}{2017}):

\begin{itemize}
\tightlist
\item
  \textbf{echo} - O código da Chunk deve ser incluído no resultado
  final? Padrão = FALSE.
\item
  \textbf{error} - Mostra mensagens de erro no documento (TRUE) ou para
  quando os erros ocorrem.
\item
  \textbf{fig.align} - Alinhamento da figura: ``left'', ``right'' ou
  ``center'' (padrão = ``default'').
\item
  \textbf{fig.height, fig.width} - Tamanho das figuras em polegadas.
\item
  \textbf{include} - Para incluir a Chunk depois de compilar (padrão =
  TRUE).
\item
  \textbf{message} - Mostra as mensagens por ventura existentes no
  documento (padrão = TRUE).
\item
  \textbf{results} - (default=``markup'') ``asis'' - processa os
  resultados na saída do documento; ``hide'' - não mostra os resultados;
  ``hold'' - coloca os resultados abaixo do código.
\item
  \textbf{warning} - Mostra avisos de advertência no documento (padrão =
  TRUE).
\end{itemize}

Como mencionado no início deste capítulo, a grande vantagem do
\textbf{RMarkdown} é a sua versatilidade na criação de documentos
concatenados com as análises estatísticas no RStudio. Desta forma,
dentro das Ckunks, podem ser criadas bases de dados, bem como importados
de sites ou mesmo carregados de arquivos trabalhados previamente no
RStudio.

No exemplo abaixo, foi criado um \emph{data frame} nomeado ``amost''
diretamente no console dentro da Chunk. Em um segundo momento, para
utilizarmos um determinado pacote instalado no RStudio, utiliza-se,
dentro da Chunk, o comando \texttt{require\ ()} juntamente com o pacote
necessário. Podem ser inseridos tantos pacotes quanto forem utilizados
no documento, conforme a Figura \ref{fig:rmarkchunk3}.

\begin{figure}[H]

{\centering \includegraphics[width=0.8\linewidth]{rmarkchunk3} 

}

\caption{Exemplo de criação de Chunk e carregamento de pacote}\label{fig:rmarkchunk3}
\end{figure}

Fonte: Elaborado pelo(s) autor(es).

\hypertarget{inserindo-tabelas-com-as-chunks}{%
\subsection{Inserindo tabelas com as
Chunks}\label{inserindo-tabelas-com-as-chunks}}

Como visto, algumas ações extremamente úteis podem ser efetuadas por
meio das Chunks. Dentre elas, inclui-se a plotagem de tabelas no texto
final, derivadas de objetos criados pelo pesquisador no RStudio. Os
exemplos trazidos abaixo incluem a utilização dos pacotes
\texttt{kable}, \texttt{xtable} e \texttt{flextable} para a criação das
tabelas.

\begin{figure}[H]

{\centering \includegraphics[width=0.8\linewidth]{rmarkchunktab1} 

}

\caption{Exemplo de criação de tabelas com os pacotes kable, xtable e flextable}\label{fig:rmarkchunk31}
\end{figure}

Fonte: Elaborado pelo(s) autor(es).

Além disso, o pacote \texttt{stargazer} é extremamente útil para geração
de tabelas com resultados de regressões com a saída dos documentos em
PDF.

\begin{figure}[H]

{\centering \includegraphics[width=0.8\linewidth]{rmarkchunktab2} 

}

\caption{Exemplo de criação de tabelas com stargazer}\label{fig:rmarkchunk33}
\end{figure}

Fonte: Elaborado pelo(s) autor(es).

Outra forma de passar as tabelas para o Word é criando-a no formato HTML
e copiando para o arquivo em Word (veja em
\url{https://cran.r-project.org/web/packages/kableExtra/vignettes/kableExtra_and_word.html}.).

\hypertarget{inserindo-imagens-com-as-chunks}{%
\subsection{Inserindo imagens com as
Chunks}\label{inserindo-imagens-com-as-chunks}}

Da mesma forma que as tabelas, as imagens também podem ser inseridas com
o auxílio de Chunks. Lembrando que a imagem deve estar na mesma pasta do
arquivo ou na pasta indicada:

\begin{figure}[H]

{\centering \includegraphics[width=0.8\linewidth]{rmarkchunkimg} 

}

\caption{Exemplo de inserção de imagens pelos Chunks}\label{fig:rmarkchunk333}
\end{figure}

Fonte: Elaborado pelo(s) autor(es).

\hypertarget{criacao-de-modelos-para-formatacao-vinculada}{%
\section{Criação de modelos para formatação
vinculada}\label{criacao-de-modelos-para-formatacao-vinculada}}

Para os pesquisadores que trabalham intensamente com o Word ou
Libre/Open Office, a formatação dos resultados decorrentes das análises
compiladas no RMarkdown podem ser incrementadas. Isto porque existe um
recurso de criação de modelos vinculados aos editores de texto, estes
que serão responsáveis pela definição da formatação de todos os itens
(títulos, subtítulos, parágrafos, fontes, etc.), como será visto a
seguir.

\hypertarget{primeiro-passo-criacao-de-um-documento-modelo}{%
\subsection{Primeiro passo: criação de um documento
modelo}\label{primeiro-passo-criacao-de-um-documento-modelo}}

Primeiramente deve-se criar um documento mínimo padrão que será
utilizado como modelo. Crie um novo documento (Rmd), aqui denominaremos
de ``modelo'' (o usuário pode escolher o nome), que será salvo em .Rmd e
gerado o respectivo arquivo Word (ou no formato .odt), na mesma pasta
que o pesquisador salvar arquivos a serem formatados.

Como já visto, para criação de documentos .Rmd clique em ``File \(>\)
New File \(>\) R Markdown''. Escolha o nome e salve na pasta escolhida.
Gere o documento em Word (.docx) ou em outro arquivo de texto (exemplo
.odt) em ``File \(>\) Knit Document''.

\hypertarget{segundo-passo-formatacao-do-modelo}{%
\subsection{Segundo passo: formatação do
modelo}\label{segundo-passo-formatacao-do-modelo}}

Abra o arquivo em Word (denominamos ``modelo.docx''). Atente para a
caixa de seleção de estilos do Word, que será trabalhado nesta etapa
(Figura \ref{fig:rmarkestilos}).

\begin{figure}[H]

{\centering \includegraphics[width=0.8\linewidth]{rmarkestilos} 

}

\caption{Caixa estilos no Word}\label{fig:rmarkestilos}
\end{figure}

Fonte: Elaborado pelo(s) autor(es).

Note que para o resultado desta compilação, o menu estilos traz várias
formatações das diferentes partes do texto, entre elas ``Abstract'',
``Author'', ``Normal'', ``Titulo'', ``Titulo 1'', etc. Estes estilos
serão alterados pelo usuário, para adequar às necessidades do
pesquisador na criação do documento padrão. Clique com o botão direito
nos estilos e em ``Modificar'' para definir a formatação padrão para
cada parte do texto.

\begin{figure}[H]

{\centering \includegraphics[width=0.8\linewidth]{rmarkestilos1} 

}

\caption{Modificação de estilos no Word}\label{fig:rmarkestilos1}
\end{figure}

Fonte: Elaborado pelo(s) autor(es).

\hypertarget{terceiro-passo-vinculacao-do-modelo-ao-arquivo-em-rmarkdown}{%
\subsection{Terceiro passo: vinculação do modelo ao arquivo em
RMarkdown}\label{terceiro-passo-vinculacao-do-modelo-ao-arquivo-em-rmarkdown}}

Após detemrinar as alterações em todos os campos de estilos do documento
modelo no Word, o pesquisador deve vincular este modelo ao documento
.Rmd principal. Além de deixar salvo o modelo em Word na mesma pasta,
deve-se incluir a seguinte informação no YAML mostrada na Figura
\ref{fig:rmarkestilos2} (\texttt{reference\_docx}). Lembrando que para
arquivos em Open/Libre Office, deve ser inserida a opção
\texttt{reference\_odt} seguida do arquivo (.odt) do modelo.

\begin{figure}[H]

{\centering \includegraphics[width=0.8\linewidth]{rmarkestilos2} 

}

\caption{Vinculação do modelo}\label{fig:rmarkestilos2}
\end{figure}

Fonte: Elaborado pelo(s) autor(es).

A partir de então, as compilações do arquivo .Rmd criado pelo
pesquisador seguirão as formatações de estilo que estão determinadas no
arquivo ``modelo.docx''.

\hypertarget{citacoes-e-bibliografias}{%
\section{Citações e bibliografias}\label{citacoes-e-bibliografias}}

Na escrita de trabalhos acadêmicos com o RMarkdown é possível efetuar um
gerenciamento de citações e bibliografias de maneira extremamente
satisfatória e automática. Para isto, o \emph{software} MiKTeX
contribuirá nesta empreitada.\% juntamente com o programa Pandoc.

O exemplo abaixo será utilizado com o formato BibLaTeX (extensão .bib).
Primeiramente crie um documento .bib, que será o local onde o
pesquisador armazenará as bibliografias, que serão posteriormente
utilizadas. Crie um novo arquivo de texto (``File \(>\) New File \(>\)
Text file'') e depois salve-o na mesma pasta do arquivo .Rmd em que
serão inseridas as citações (salve com a extensão ``.bib'' - exemplo:
``bibliografia.bib'').

Dentro deste arquivo serão armazenadas as referências biliográficas, não
deve-se preocupar neste momento com a ordem das referências. Como mostra
a Figura \ref{fig:rmarkbib}, inserimos duas bibliografias a serem
citadas posteriormente.

A primeira (\texttt{@article}), demonstra que é um artigo de uma revista
enquanto a segunda (\texttt{@book}) se trata de um livro. Dentro das
chaves estão os dados das referências, como o título (\texttt{title}),
autores (\texttt{author}) e o ano (\texttt{year}) por exemplo.

\begin{figure}[H]

{\centering \includegraphics[width=0.8\linewidth]{rmarkbib} 

}

\caption{Arquivo .bib}\label{fig:rmarkbib}
\end{figure}

Fonte: Elaborado pelo(s) autor(es).

O BibLateX gerencia todos os tipos de bibliografias sendo que, como
visto acima, as bibliografias possuem campos padrão a serem informados
no arquivo ``.bib''. Por exemplo, a categoria de artigos, possui como
campos obrigatórios \texttt{author}, \texttt{title}, \texttt{journal} e
\texttt{year}. Abaixo seguem algumas especificações dos tipos de
bibliografias (adaptado de LEHMAN; KIME
(\protect\hyperlink{ref-biblatex}{2006})), explicitando os itens
obrigatórios e opcionais que devem ou podem constar em cada registro:

\texttt{@article} - Artigo de revista. \textbf{OBRIGATÓRIOS}: author,
title, journal, year. \textbf{OPCIONAIS}: volume, number, pages, month,
note, key.

\texttt{@book} - Livro. \textbf{OBRIGATÓRIOS}: author/editor, title,
publisher, year. \textbf{OPCIONAIS}: volume/number, series, address,
edition, month, note, key.

\texttt{@inbook} - Parte de livro. \textbf{OBRIGATÓRIOS}: author/editor,
title, chapter/pages, publisher, year. \textbf{OPCIONAIS}:
volume/number, series, type, address, edition, month, note, key.

\texttt{@incollection} - Parte de livro com título próprio.
\textbf{OBRIGATÓRIOS}: author, title, booktitle, publisher, year.
\textbf{OPCIONAIS}: editor, volume/number, series, type, chapter, pages,
address, edition, month, note, key.

\texttt{@inproceedings} - Trabalho de anais de conferência.
\textbf{OBRIGATÓRIOS}: author, title, booktitle, year.
\textbf{OPCIONAIS}: editor, volume/number, series, pages, address,
month, organization, publisher, note, key.

\texttt{@mastersthesis} - Dissertação mestrado. \textbf{OBRIGATÓRIOS}:
author, title, school, year. \textbf{OPCIONAIS}: type, address, month,
note, key.

\texttt{@phdthesis} - Tese de doutorado. \textbf{OBRIGATÓRIOS}: author,
title, school, year. \textbf{OPCIONAIS}: type, address, month, note,
key.

Estas configurações do BibLateX são comuns nos programas de
gerenciamento de bibliografias, como por exemplo no \emph{software}
Mendeley. Os usuários deste programa tem uma facilidade na exportação
para o formato do BibLateX, pois podem copiar as entradas com as
informações de um trabalho e inserí-las dentro do arquivo .bib (Figura
\ref{fig:rmarkmendeley}).

\begin{figure}[H]

{\centering \includegraphics[width=0.8\linewidth]{rmarkmendeley} 

}

\caption{Utilização do Mendeley para exportação de dados de bibliografias}\label{fig:rmarkmendeley}
\end{figure}

Fonte: Elaborado pelo(s) autor(es).

Após escolhidas as bibliografias a serem utilizadas no trabalho, o
pesquisador deve inserir estas entradas como referências dentro do
texto. Para isto, utiliza o nome da bibliografia inserida no arquivo
.bib, no nosso exemplo \texttt{bresser} e \texttt{Forstater2008}, como
mostra a Figura \ref{fig:rmarkcitar}.

\begin{figure}[H]

{\centering \includegraphics[width=0.8\linewidth]{rmarkcitar} 

}

\caption{Inserção de citações no arquivo .Rmd}\label{fig:rmarkcitar}
\end{figure}

Fonte: Elaborado pelo(s) autor(es).

Para que o arquivo que foi criado com as referências bibliográficas
(bibliografia.bib) seja utilizado, o pesquisador deve informar o seu
nome dentro do YAML no campo \texttt{bibliography}. Mas qual norma será
utilizada para as citações e a criação de referências bibliográficas
neste trabalho, já que existem diversas delas? Uma solução é a
utilização de arquivos ``.csl'' (Citation Style Language), que nada mais
são do que arquivos com as descrições de cada estilo das diversas normas
existentes, para ajudar o pesquisador a citar e gerenciar suas
referências.

Estes arquivos podem ser encontrados em diversos locais, como por
exemplo em \url{https://github.com/citation-style-language/styles}
(copie
\href{https://raw.githubusercontent.com/citation-style-language/styles/44808db510152943c5d9dc471a9c8982a3edfbea/associacao-brasileira-de-normas-tecnicas-ipea.csl}{este}
conteúdo para um arquivo ``.txt'' e o renomeie para ``.csl''). Lembrando
que o arquivo ``.csl'' deve ser salvo na mesma pasta do arquivo
``.Rmd''. O arquivo csl aqui utilizado refere-se às normas da ABNT
(Associação Brasileira de Normas Técnicas) utilizados pelo IPEA
(Instituto de Pesquisa Econômica Aplicada). Verifica-se na Figura
\ref{fig:rmarkcitar1} a configuração final do YAML. Neste site
\url{http://editor.citationstyles.org/searchByName/} também são
encontrados arquivos para várias normas bibliográficas.

\begin{figure}[H]

{\centering \includegraphics[width=0.8\linewidth]{rmarkcitar1} 

}

\caption{Configurando YAML para citações e fererências}\label{fig:rmarkcitar1}
\end{figure}

Fonte: Elaborado pelo(s) autor(es).

Por fim, após inserir a citação no texto e informar ao RMarkdown os
arquivos ``.bib'' e ``.csl'' no YAML, basta compilar o arquivo no
formato desejado (atalho no teclado CTRL+SHIFT+K), neste caso Word.
Lembre-se de inserir um título \texttt{\#\ Referências} ou
\texttt{\#\ Referências\ Bibliográficas} ou \texttt{\#\ Bibliografia}
(como preferir), no final do texto, pois serão inseridas as referências
no final do trabalho.

A partir de então fica muito mais fácil alterar a norma necessária para
a produção do trabalho acadêmico, utilizando os mesmos dados de um
artigo ou outro material a ser citado. Isto agiliza a produção acadêmica
e proporciona, como visto neste livro, uma interação muito proveitosa
com a geração das análises por meio do RStudio.

Segue o resultado do arquivo final:

\begin{figure}[H]

{\centering \includegraphics[width=0.8\linewidth]{rmarkcitarf} 

}

\caption{Resultado final das citações e referências com RMarkdown}\label{fig:rmarkcitarf}
\end{figure}

Fonte: Elaborado pelo(s) autor(es).

\hypertarget{sobre-os-autores}{%
\chapter*{Sobre os autores}\label{sobre-os-autores}}
\addcontentsline{toc}{chapter}{Sobre os autores}

\textbf{Denize Ivete Reis}: Possui Licenciatura Plena em Matemática pela
Universidade Regional do Noroeste do Estado do Rio Grande do Sul (1994),
especialização em Estatística Aplicada pela Universidade de Santa Cruz
do Sul (2003), mestrado em Modelagem Matemática pela Universidade
Regional do Noroeste do Estado do Rio Grande do Sul (1997) e doutorado
em Qualidade Ambiental pela Universidade Feevale (2015). Atualmente é
professora adjunta da Universidade Federal da Fronteira Sul, onde atua
na área de Probabilidade e Estatística, Estatística Descritiva e
Inferência Estatística. E-mail:
\href{mailto:denizeir@uffs.edu.br}{\nolinkurl{denizeir@uffs.edu.br}}.

\textbf{Djaina Sibiani Rieger}: Acadêmica do curso de Engenharia
Ambiental e Sanitária da Universidade Federal da Fronteira Sul (UFFS)
Campus Cerro Largo, aluna bolsista de extensão, integrante e conteudista
dos cursos ofertados no Campus sobre o \emph{software} livre R.

\textbf{Erikson Kaszubowski}: Doutor em Psicologia pela Universidade
Federal de Santa Catarina, sob orientação do Prof.~Dr.~Fernando Aguiar,
com a tese ``Modelos de tópicos para associações livres''. Formado em
Psicologia pela Universidade Federal de Santa Catarina, nas graduações
Bacharelado e Formação de Psicólogo (2006), e Licenciatura (2008). Foi
professor de Psicologia da Educação na Universidade Federal da Fronteira
Sul, ministrando as disciplinas da área de Psicologia nos cursos de
Licenciatura e Pós-Graduação. Trabalha atualmente como psicólogo clínico
no Serviço de Atenção Psicológica da UFSC. E-mail:
\href{mailto:erikson84@yahoo.com.br}{\nolinkurl{erikson84@yahoo.com.br}}.

\textbf{Felipe Micail da Silva Smolski}: Possui graduação em Ciências
Econômicas pela Universidade Regional do Noroeste do Estado do Rio
Grande do Sul - UNIJUÍ (2009), pós-graduação em Gestão de Investimentos
pela Faculdade Integrada Grande Fortaleza - FGF (2012), mestrado em
Desenvolvimento e Políticas Públicas pela Universidade Federal da
Fronteira Sul - UFFS, Campus Cerro Largo (2017). E-mail:
\href{mailto:felipesmolski@hotmail.com}{\nolinkurl{felipesmolski@hotmail.com}}.

\textbf{Iara Denise Endruweit Battisti}: Possui graduação em Informática
pela Universidade Regional do Noroeste do Estado do Rio Grande do Sul
(1996), mestrado em Estatística e Experimentação Agropecuária pela
Universidade Federal de Lavras (2001) e doutorado em Epidemiologia pela
Universidade Federal do Rio Grande do Sul (2008). Atualmente é
professora adjunta na Universidade Federal da Fronteira Sul, campus
Cerro Largo (RS). Atua principalmente nos seguintes temas: amostragem
complexa, modelagem multinível, estatística computacional, estatística
aplicada, relação ambiente e saúde utilizando modelagem estatística.
E-mail:
\href{mailto:iara.battisti@uffs.edu.br}{\nolinkurl{iara.battisti@uffs.edu.br}}.

\textbf{Tatiane Chassot}: Possui graduação em Engenharia Florestal pela
Universidade Federal de Santa Maria (2008), mestrado (2009) e doutorado
em Engenharia Florestal também pela Universidade Federal de Santa Maria
(2013). Atualmente é professora adjunta da Universidade Federal da
Fronteira Sul - Campus Cerro Largo onde ministra as disciplinas de
Introdução à Informática, Estatística Básica, Experimentação Agrícola,
Sistemas Agroflorestais, Silvicultura e Práticas Integradoras de Campo.
E-mail:
\href{mailto:tatianechassot@uffs.edu.br}{\nolinkurl{tatianechassot@uffs.edu.br}}.

\hypertarget{referencias}{%
\chapter*{Referências}\label{referencias}}
\addcontentsline{toc}{chapter}{Referências}

\hypertarget{refs}{}
\leavevmode\hypertarget{ref-R-rmarkdown}{}%
ALLAIRE, J. J. et al. \textbf{rmarkdown: Dynamic Documents for R}, 2017.
Disponível em:
\textless{}\url{https://CRAN.R-project.org/package=rmarkdown}\textgreater{}

\leavevmode\hypertarget{ref-almeida2000}{}%
ALMEIDA, L.; FREIRE, T. \textbf{Metodologia da investigação em
psicologia e educação}. Traducao. 2. ed. Braga: Psiquilíbrios, 2000.

\leavevmode\hypertarget{ref-barbetta1988}{}%
BARBETTA, P. A. \textbf{Estatística aplicada às ciências sociais}.
Traducao. 7. ed. Florianópolis: Ed. UFSC, 2010.

\leavevmode\hypertarget{ref-hoffmann1998}{}%
HOFFMANN, R.; OTHERS. \textbf{Análise de regressão: uma introdução à
econometria}. Traducao. São Paulo: Hucitec, 1998.

\leavevmode\hypertarget{ref-biblatex}{}%
LEHMAN, P.; KIME, P. \textbf{BibLATEX -- Sophisticated Bibliographies in
LATEX}, 2006. Disponível em:
\textless{}\url{https://ctan.org/pkg/biblatex}\textgreater{}

\leavevmode\hypertarget{ref-lopes2008}{}%
LOPES, L. F. D. et al. \textbf{Caderno didático: estatística Geral}.
Traducao. 3. ed. Santa Maria: UFSM, 2008. p. 209

\leavevmode\hypertarget{ref-Ristow2017}{}%
RISTOW, L. P. \textbf{Exposição ocupacional de trabalhadores rurais a
agrotóxicos e relação com políticas públicas}. tese de mestrado---Cerro
Largo: Universidade Federal da Fronteira Sul, 2017.

\leavevmode\hypertarget{ref-triola1999}{}%
TRIOLA, M. F. \textbf{Introdução à estatística}. Traducao. 10. ed. Rio
de Janeiro: LTC, 2011.


\end{document}
